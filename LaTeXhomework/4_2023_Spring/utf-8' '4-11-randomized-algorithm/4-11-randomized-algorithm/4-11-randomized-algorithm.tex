% 2-15-rb-tree.tex

%%%%%%%%%%%%%%%%%%%%
\documentclass[a4paper, justified]{tufte-handout}

\input{hw-preamble} % feel free to modify this file
%%%%%%%%%%%%%%%%%%%%
\title{第4-11讲: 随机算法}
\me{林凡琪 }{211240042 }{}{}
\date{\zhtoday} % or like 2019年9月13日
%%%%%%%%%%%%%%%%%%%%
\begin{document}
\maketitle
%%%%%%%%%%%%%%%%%%%%
\noplagiarism % always keep this line
%%%%%%%%%%%%%%%%%%%%
\begin{abstract}
  % \begin{center}{\fcolorbox{blue}{yellow!60}{\parbox{0.65\textwidth}{\large 
  %   \begin{itemize}
  %     \item 
  %   \end{itemize}}}}
  % \end{center}
\end{abstract}
%%%%%%%%%%%%%%%%%%%%
\beginrequired

%%%%%%%%%%%%%%%
\begin{problem}[JH 5.2.2.7]
\end{problem}

\begin{solution}
  (i)\\
  在$[2,n^c]$中,质数的近似数量为$\frac{n^c}{\ln n^c}$,因此$c\log_2{n}$个比特足以实现随机选择。\\
  由于$s\leq p\leq n^c$,所以$|s|\leq c\log_2{n}$。\\
  因此,该协议的通信复杂度为$2c\log_2{n}$。\\
  (ii)\\
  当$x\leq y$时,被判定为相等的概率为
  \[
    \frac{n-1}{n^c/\ln n^c}\leq  \frac{ln n^c}{n^{c-1}}
  \]
  因此Prob ((Rr, Rn) accepts (x, y)) $\geq 1 -\frac{ln n^c}{n^{c-1}} $。
\end{solution}
%%%%%%%%%%%%%%%

%%%%%%%%%%%%%%%
\begin{problem}[JH 5.2.2.8]
\end{problem}

\begin{solution}
  反证法。假设存在一种确定性算法,使得the communication complexity小于$n$。\\
  由假设可推得,必然存在$u,v\in\{0,1\}^{n}$,$u\not\equiv v$,使得$\overline(C_1(u))=\overline(C_1(v))$。\\
  所以$\overline(C_2(\overline(C_1(u)),u))=\overline(C_2(\overline(C_1(v)),u))$。\\
  显然$u\equiv u$,可得$u \equiv v$,这与假设矛盾。\\
  故for every $n\in \mathbb{N}^{+}$, that every deterministic one-way protocol computing $Equality_n$ has a communication complexity of at least n.\\
  (ii)\\
  \begin{figure}[htbp]
    \centering
    \includegraphics[width = 0.90\linewidth]{figs/a}
  \end{figure}


  采用该算法即可。根据书中的证明,可得:\\
  (1)在$x\equiv y$时,$Prob((R_I,R_n) accepted (x,y)) = 1$.\\
  (2)在$x\not\equiv y$时, $Prob((R_I,R_n) accepts (x,y))\leq \frac{\ln n^2}{n}$\\
  故$Prob(A(x)=F(x)) \geq 1-\frac{\ln n^2}{n}\geq \frac{1}{2}+\epsilon$,满足题目要求。\\
  (iii)\\
  反证法。假设存在一种one-sided-error,使得the communication complexity小于$n$。\\
  由假设可推得,必然存在$u,v\in\{0,1\}^{n}$,$u\not\equiv v$,使得$\overline(C_1(u))=\overline(C_1(v))$。\\
  所以$\overline(C_2(\overline(C_1(u)),u))=\overline(C_2(\overline(C_1(v)),u))$。\\
  显然$u\equiv u$,可得$u \equiv v$,这与假设中的one-sided-error矛盾。\\
  故one-sided-error的 communication complexity至少为$n$.\\
\end{solution}
%%%%%%%%%%%%%%%

%%%%%%%%%%%%%%%
%%%%%%%%%%%%%%%%%%%%
\beginoptional
%%%%%%%%%%%%%%%


%%%%%%%%%%%%%%%%%%%%
\beginot
%%%%%%%%%%%%%%%
\begin{ot}[例题5.2.2.5]
  请讲解例题5.2.2.5,并说明,为什么这个随机算法代价好于“任何”确定性算法。
\end{ot}

% \begin{solution}
% \end{solution}
%%%%%%%%%%%%%%%




\vspace{0.50cm}
%%%%%%%%%%%%%%
\begin{ot}[Karger's Algorithm]

  \noindent 参考资料:
  \begin{itemize}
    \item \href{https://en.wikipedia.org/wiki/Karger%27s_algorithm}{ Wikipedia[Karger's algorithm]}
  \end{itemize}
\end{ot}

% \begin{solution}
% \end{solution}
%%%%%%%%%%%%%%%

%%%%%%%%%%%%%%%%%%%%
% 如果没有需要订正的题目,可以把这部分删掉

% \begincorrection
%%%%%%%%%%%%%%%%%%%%

%%%%%%%%%%%%%%%%%%%%
% 如果没有反馈,可以把这部分删掉
\beginfb

% 你可以写
% ~\footnote{优先推荐 \href{problemoverflow.top}{ProblemOverflow}}:
% \begin{itemize}
%   \item 对课程及教师的建议与意见
%   \item 教材中不理解的内容
%   \item 希望深入了解的内容
%   \item $\cdots$
% \end{itemize}
%%%%%%%%%%%%%%%%%%%%
% \bibliography{2-5-solving-recurrence}
% \bibliographystyle{plainnat}
%%%%%%%%%%%%%%%%%%%%
\end{document}