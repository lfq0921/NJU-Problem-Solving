% 2-15-rb-tree.tex

%%%%%%%%%%%%%%%%%%%%
\documentclass[a4paper, justified]{tufte-handout}

\input{hw-preamble} % feel free to modify this file
%%%%%%%%%%%%%%%%%%%%
\title{第4-10讲: 近似算法}
\me{ 林凡琪}{211240042 }{}{}
\date{\zhtoday} % or like 2019年9月13日
%%%%%%%%%%%%%%%%%%%%
\begin{document}
\maketitle
%%%%%%%%%%%%%%%%%%%%
\noplagiarism % always keep this line
%%%%%%%%%%%%%%%%%%%%
\begin{abstract}
  % \begin{center}{\fcolorbox{blue}{yellow!60}{\parbox{0.65\textwidth}{\large 
  %   \begin{itemize}
  %     \item 
  %   \end{itemize}}}}
  % \end{center}
\end{abstract}
%%%%%%%%%%%%%%%%%%%%
\beginrequired

%%%%%%%%%%%%%%%
\begin{problem}[JH 4.2.1.4]
\end{problem}

\begin{solution}
  假设在机器$M_l$上满足有$Time(M_l)=cost(I)$,且其最后一个执行的任务为$p_k$。\\
  根据该任务调度的性质,可得
  \[
    opt(I)\geq\frac{\sum_{i=1}^{n}p_i}{m}
  \]
  \[
    cost(I)-p_k\leq\frac{\sum_{i=1}^{n}[i \neq k]\text{·}p_i}{m}
  \]
  \[
    \frac{\sum_{i=1}^{n}[i \neq k]\text{·}p_i}{m}\leq \frac{\sum_{i=1}^{n}p_i}{m}
  \]
  结合以上,可得
  \[
    opt(I)\geq cost(I) - p_k
  \]
  又有
  \[
    opt(I)\geq p_k
  \]
  于是
  \[
    opt(I)\geq cost(I) - opt(I)
  \]
  即
  \[
    \frac{cost(I)}{opt(I)}\leq 2
  \]
  综上得证。
\end{solution}
%%%%%%%%%%%%%%%

%%%%%%%%%%%%%%%
\begin{problem}[JH 4.2.1.5]
\end{problem}

\begin{solution}
  在GMS问题中,我们可以构造$2m+1$个任务,其中有3个任务的用时为$m$,用时为$m+1$到$2m-1$的各有两个,当有$m$个机器时。\\
  最优的分配方案是:(m,m,m),(m+1,2m-1),(m+2,2m-2),...,(2m-1,m+1),总用时为$3m$\\
  在贪心解法中,(m,m,2m-1)是调度最长的机器上的分配,$cost(I)=4m-1$\\
  因此,近似比为$\frac{4m-1}{3m}$
\end{solution}
%%%%%%%%%%%%%%%

%%%%%%%%%%%%%%%
\begin{problem}[JH 4.2.3.3]
\end{problem}

\begin{solution}
  dist: \\
  对于任意$(G,c)\in L_{\triangle}$,由于三角不等式,$\frac{c(\{u,v\})}{c(\{u,p\})+ {c(\{p,v\})}}<1$,所以$dist(G,c)=0$。\\
  枚举$u,v,p$的时间复杂度为$O(n^3)$,可以在多项式时间内计算。\\
  因此,dist是distance functions。\\
  $dist_{k}$: \\
  对于任意$(G,c)\in L_{\triangle}$,由于三角不等式,$\frac{c(\{u,v\})}{\sum_{i=1}^kc(\{p_i, p_{i+1}\})}<1$,所以$dist_k(G,c)=0$。\\
  枚举路径并判断的时间复杂度为$O(n^k+k)$,可以在多项式时间内计算。\\
  因此,$dist-k$是distance functions。\\
  distance: \\
  对于任意$(G,c)\in L_{\triangle}$,由于$dis_k$的性质,distance(G,c)=0。\\
  枚举$dist_k$中k的复杂度为$O(k)$,且$dist_k$可以在多项式时间内计算,所以distance可以在多项式时间内计算。\\
  因此,distance是distance functions。
\end{solution}
%%%%%%%%%%%%%%%

%%%%%%%%%%%%%%%
\begin{problem}[JH 4.2.3.4]
\end{problem}

\begin{solution}
  (1)\\
  对于任意$(G,c)\in L_{I}$,$h_index(w)=0$\\
  计算the canonical order of words与输入规模同阶,故若输入规模$O(n)$,$h_{index}$必为多项式可计算。\\
  综上,$h_{index}$为distance function。\\
  (2)\\
  对于距离$r$,有
  \[
    \delta_{r,\epsilon}=max\{\delta, R_{A}(I)| I \in BALL_{r,h}(L_I)\}
  \]
  由canonical order的性质,可知上述集合为有限集,是可取max的。\\
  因此其对任意$r$存有$\delta_{r,\epsilon}$近似算法。\\
  故其为稳定
\end{solution}
%%%%%%%%%%%%%%%

%%%%%%%%%%%%%%%
%%%%%%%%%%%%%%%%%%%%
\beginoptional
%%%%%%%%%%%%%%%


%%%%%%%%%%%%%%%%%%%%
\beginot
%%%%%%%%%%%%%%%
\begin{ot}[$\Delta$-TSP]
  找出一个$\Delta$-TSP问题的近似解,并证明这个解的近似度界,说明这个问题是NPO的哪一类问题?
\end{ot}

% \begin{solution}
% \end{solution}
%%%%%%%%%%%%%%%

%%%%%%%%%%%%%%%
\begin{ot}[SCP]
  介绍SCP问题,证明它属于NPO(IV),介绍它的贪心近似算法,并证明它的近似比(JH算法4.3.2.11)
\end{ot}


% \begin{solution}
% \end{solution}
%%%%%%%%%%%%%%%


% \vspace{0.50cm}
%%%%%%%%%%%%%%%
% \begin{ot}[]
% 
%   \noindent 参考资料:
%   \begin{itemize}
%     \item 
%   \end{itemize}
% \end{ot}

% \begin{solution}
% \end{solution}
%%%%%%%%%%%%%%%

%%%%%%%%%%%%%%%%%%%%
% 如果没有需要订正的题目,可以把这部分删掉

% \begincorrection
%%%%%%%%%%%%%%%%%%%%

%%%%%%%%%%%%%%%%%%%%
% 如果没有反馈,可以把这部分删掉
\beginfb

% 你可以写
% ~\footnote{优先推荐 \href{problemoverflow.top}{ProblemOverflow}}:
% \begin{itemize}
%   \item 对课程及教师的建议与意见
%   \item 教材中不理解的内容
%   \item 希望深入了解的内容
%   \item $\cdots$
% \end{itemize}
%%%%%%%%%%%%%%%%%%%%
% \bibliography{2-5-solving-recurrence}
% \bibliographystyle{plainnat}
%%%%%%%%%%%%%%%%%%%%
\end{document}