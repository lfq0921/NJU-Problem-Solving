
% 2-15-rb-tree.tex

%%%%%%%%%%%%%%%%%%%%
\documentclass[a4paper, justified]{tufte-handout}

\input{hw-preamble} % feel free to modify this file
%%%%%%%%%%%%%%%%%%%%
\title{第4-4讲: 数论初步}
\me{林凡琪 }{211240042 }{}{}
\date{\zhtoday} % or like 2019年9月13日
%%%%%%%%%%%%%%%%%%%%
\begin{document}
\maketitle
%%%%%%%%%%%%%%%%%%%%
\noplagiarism % always keep this line
%%%%%%%%%%%%%%%%%%%%
\begin{abstract}
  % \begin{center}{\fcolorbox{blue}{yellow!60}{\parbox{0.65\textwidth}{\large 
  %   \begin{itemize}
  %     \item 
  %   \end{itemize}}}}
  % \end{center}
\end{abstract}
%%%%%%%%%%%%%%%%%%%%
\beginrequired

%%%%%%%%%%%%%%%
\begin{problem}[TJ 2-15(b,f)]
\end{problem}

\begin{solution}
  (b) 234 and 165

  $$gcd(234,165)=3$$

  $$r=12,s=-17$$

  (f)-4357 and 3754

  $$gcd(-4357,3754) = 1$$

  $$r=1463,s=1698$$
\end{solution}
%%%%%%%%%%%%%%%

%%%%%%%%%%%%%%%
\begin{problem}[TJ 2-16]
\end{problem}

\begin{proof}
  令$gcd(a,b)=t$,那么$a=k_1t,b=k_2t,k_1,k_2 \neq 0$,可知

  $$ar+bs=t(k_1r+k_2s)=1$$

  因为$k_1r+k_2s \neq 0$, 所以$t|1$

  可知$t=1$
\end{proof}
%%%%%%%%%%%%%%%

%%%%%%%%%%%%%%%
\begin{problem}[TJ 2-19]
\end{problem}

\begin{proof}
  令
  $$xy=p_1^{2k_1}p_2^{2k_2}...p_t^{2k_t},k_i \geq 0$$

  $$x=p_1^{a_1}p_2^{a_2}...p_t^{a_t},a_i \geq 0$$

  $$y=p_1^{b_1}p_2^{b_2}...p_t^{b_t},b_i \geq 0$$

  所以
  $$gcd(x,y)=p_1^{min(a_1,b_1)}p_2^{min(a_2,b_2)}...p_t^{min(a_t,b_t)}=1$$

  所以
  $$min(a_i,b_i)=0 \Rightarrow a_i=0,b_i=2k_i \text{ or } a_i =2k_i
    ,b_i =0$$

  所以x,y都是perfect squares.

\end{proof}
%%%%%%%%%%%%%%%

%%%%%%%%%%%%%%%
\begin{problem}[TJ 2-29]
\end{problem}

\begin{proof}
  反证法:

  假设有有限的质数 $p_0=5,p_1,p_2,...,p_k$可以用$6n+5$的形式表示.

  令$S=\{p_1,p_2,...,p_k\}$.

  令$P=6p_1p_2...p_k+5$

  当$P$是质数, 与假设矛盾.

  当$P=q_1q_2...q_s$(其中$q_i$是质数),显然$q_i \neq 0,2,3,4(\mod 6)$

  如果$\forall q_i,q_i=1(\mod 6).$那么,$P=q_1q_2...q_s=1(\mod 6)$,这和$P=5(\mod 6)$矛盾

  如果$\exists q_i=p_t=5(\mod 6) \in S$,那么$q_i|P\Rightarrow p_t|P \Rightarrow p_t|6p_1p_2...p_k+5 \Rightarrow p_t|5$.但是与$\forall p_t \in S, p_t >5$矛盾

  如果$\exists q_i=5.$那么$q_i|P\Rightarrow 5|6p_1p_2...p_k+5 \Rightarrow 5|6p_1p_2...p_k \Rightarrow \exists p_t \in S,5|p_t$

  但这和$p_t$是质数矛盾。

  综上得证。

\end{proof}
%%%%%%%%%%%%%%%

%%%%%%%%%%%%%%%
\begin{problem}[TJ 2-30]
\end{problem}

\begin{proof}
  反证法:

  假设有有限的质数$p_0=3,p_1,p_2,...,p_k$可以用$4n-1$的形式表示.

  令$S=\{p_1,p_2,...,p_k\}$.

  令$P=4p_1p_2...p_k+3$

  当$P$是质数, 与假设矛盾.

  当$P=q_1q_2...q_s$(其中$q_i$是质数),显然$q_i \neq 0,2(\mod 4)$

  如果$\forall q_i,q_i=1(\mod 4).$那么,$P=q_1q_2...q_s=1(\mod 4)$,这和$P=3(\mod 4)$矛盾

  如果$\exists q_i=p_t\in S$,那么$q_i|P\Rightarrow p_t|P \Rightarrow p_t|4p_1p_2...p_k-1 \Rightarrow p_t|3$.但是与$\forall p_t \in S, p_t >3$矛盾

  如果$\exists q_i=3.$那么$q_i|P\Rightarrow 3|4p_1p_2...p_k+3 \Rightarrow 3|4p_1p_2...p_k \Rightarrow \exists p_t \in S,3|p_t$

  但这和$p_t$是质数矛盾。

  综上得证。
\end{proof}
%%%%%%%%%%%%%%%

%%%%%%%%%%%%%%%
\begin{problem}[CS 2.2-2]
\end{problem}

\begin{solution}
  能保证$a$有模$m$的逆
  %No, it does not guarantee that $a$ has an inverse mod $m$.

  根据Lemma 2.8,$a$有模$m$的逆的充要条件是$a$和$m$互质。

  而在题目中$a · 133 − 2m · 277 = 1$.

  前提条件有$n\ geq 2$

  可知$$n=m \geq2,y=-544,a^{-1}=133$$

  说明 $gcd(a,m) = 1$,所以$a$有模$m$的逆。

  %A necessary and sufficient condition for $a$ to have an inverse mod $m$ is that $a$ and $m$ are coprime, i.e., their greatest common divisor is 1. In this case, $a · 133 − 2m · 277 = 1$ implies that gcd(a,m) = 1, so a has an inverse mod m. However, if we change the equation to $a · 133 − 2m · 276 = 1$, then $gcd(a,m) = 2$ and a does not have an inverse mod m.

  %If $a$ has an inverse mod $m$, it can be found using the extended Euclidean algorithm, which gives $x$ and $y$ such that $ax + my = gcd(a,m)$. Then $x$ is the inverse of $a \mod m$. For example, if $a = 7$ and $m = 26$, then using the extended Euclidean algorithm we get $x = -11$ and $y = -3$ such that $7x + 26y = -77 + -78 = -1$. Then -11 is the inverse of $7 \mod 26$.
\end{solution}
%%%%%%%%%%%%%%%

%%%%%%%%%%%%%%%
\begin{problem}[CS 2.2-4]
\end{problem}

\begin{solution}
  根据Corallary 2.16可知,

  $gcd(31,32)=1,22$ 在$Z_{31}$里有一个逆

  $gcd(10,2)=2,2$在$Z_{10}$里没有逆
\end{solution}
%%%%%%%%%%%%%%%

%%%%%%%%%%%%%%%
\begin{problem}[CS 2.2-6]
\end{problem}

\begin{solution}
  根据TH 2.15可知,two positive integers $j$ and $k$ have greatest common divisor 1 (and thus are relatively prime) if and only if there are integers $x$ and $y$ such that $jx+ky=1$

  所以$$gcd(a,m)=1$$
\end{solution}
%%%%%%%%%%%%%%%

%%%%%%%%%%%%%%%
\begin{problem}[CS 2.2-8]
\end{problem}

\begin{solution}
  According to TH 2.1, which is exactly Euclid's Division Theorem. Let $j$ be a positive integer. Then for every integer $k$, there exists unique integers $q$ and $r$ and $0\leq r < n$

  According to Lemma 2.13, if $j,k,q$ and $r$ are positive integers such that $k=jq+r$, then $$gcd(j,k)=gcd(r,j)$$

  This means that the greatest common divisor of $q$ and $k$ is equal to the greatest common divisor of $r$ and $q$.
\end{solution}
%%%%%%%%%%%%%%%

%%%%%%%%%%%%%%%
\begin{problem}[CS 2.2-16]
\end{problem}

\begin{solution}
  如果 $m<0,-m=q n+r, r=0$, 那么
  $$
    m=-q n
  $$
  令 $q^{\prime}=-q, r^{\prime}=0$.
  如果 $m<0,-m=q n+r, r>0$, 那么
  $$
    m=-q n-r=-(q+1) n+(n-r)
  $$
  令 $q^{\prime}=-(q+1), r^{\prime}=n-r$.
\end{solution}
%%%%%%%%%%%%%%%

%%%%%%%%%%%%%%%
\begin{problem}[CS 2.2-19]
\end{problem}

\begin{solution}
  $$xy=gcd(x,y)\cdot lcm(x,y)$$

  令$$x=p_1^{a_1}p_2^{a_2}...p_t^{a_t},a_i \geq 0$$

  $$y=p_1^{b_1}p_2^{b_2}...p_t^{b_t},b_i \geq 0$$

  然后
  $$gcd(x,y)=p_1^{min(a_1,b_1)}p_2^{min(a_2,b_2)}...p_t^{min(a_t,b_t)}$$
  $$lcm(x,y)=p_1^{max(a_1,b_1)}p_2^{max(a_2,b_2)}...p_t^{max(a_t,b_t)}$$

  所以
  $$
    \begin{aligned}
      x y & =p_1^{a_1+b_1} p_2^{a_2+b_2} \cdots p_t^{a_t+b_t}                                                                                                                                           \\
          & =p_1^{\min \left(a_1, b_1\right)+\max \left(a_1, b_1\right)} p_2^{\min \left(a_2, b_2\right)+\max \left(a_2, b_2\right)} \cdots p_t^{\min \left(a_t, b_t\right)+\max \left(a_t, b_t\right)} \\
          & =\operatorname{gcd}(x, y) \cdot \operatorname{lcm}(x, y)
    \end{aligned}
  $$
\end{solution}
%%%%%%%%%%%%%%%
%%%%%%%%%%%%%%%%%%%%
\beginoptional


%%%%%%%%%%%%%%%%%%%%
\beginot
%%%%%%%%%%%%%%%
\begin{ot}[Lucas定理]
  \begin{itemize}
    \item 参考资料:\href{https://brilliant.org/wiki/lucas-theorem/}{https://brilliant.org/wiki/lucas-theorem/}
  \end{itemize}
\end{ot}

% \begin{solution}
% \end{solution}
%%%%%%%%%%%%%%%

%%%%%%%%%%%%%%%
\begin{ot}[Miller-Rabin Algorithm]
\end{ot}


% \begin{solution}
% \end{solution}
%%%%%%%%%%%%%%%


% \vspace{0.50cm}
%%%%%%%%%%%%%%%
% \begin{ot}[]
% 
%   \noindent 参考资料:
%   \begin{itemize}
%     \item 
%   \end{itemize}
% \end{ot}

% \begin{solution}
% \end{solution}
%%%%%%%%%%%%%%%

%%%%%%%%%%%%%%%%%%%%
% 如果没有需要订正的题目,可以把这部分删掉

% \begincorrection
%%%%%%%%%%%%%%%%%%%%

%%%%%%%%%%%%%%%%%%%%
% 如果没有反馈,可以把这部分删掉
\beginfb

% 你可以写
% ~\footnote{优先推荐 \href{problemoverflow.top}{ProblemOverflow}}:
% \begin{itemize}
%   \item 对课程及教师的建议与意见
%   \item 教材中不理解的内容
%   \item 希望深入了解的内容
%   \item $\cdots$
% \end{itemize}
%%%%%%%%%%%%%%%%%%%%
% \bibliography{2-5-solving-recurrence}
% \bibliographystyle{plainnat}
%%%%%%%%%%%%%%%%%%%%
\end{document}