% 2-15-rb-tree.tex

%%%%%%%%%%%%%%%%%%%%
\documentclass[a4paper, justified]{tufte-handout}

\input{hw-preamble} % feel free to modify this file
%%%%%%%%%%%%%%%%%%%%
\title{第4-7讲: 代数编码}
\me{211240042 }{林凡琪 }{}{}
\date{\zhtoday} % or like 2019年9月13日
%%%%%%%%%%%%%%%%%%%%
\begin{document}
\maketitle
%%%%%%%%%%%%%%%%%%%%
\noplagiarism % always keep this line
%%%%%%%%%%%%%%%%%%%%
\begin{abstract}
  % \begin{center}{\fcolorbox{blue}{yellow!60}{\parbox{0.65\textwidth}{\large 
  %   \begin{itemize}
  %     \item 
  %   \end{itemize}}}}
  % \end{center}
\end{abstract}
%%%%%%%%%%%%%%%%%%%%
\beginrequired

%%%%%%%%%%%%%%%
\begin{problem}[TJ 8-6(b,d)]
\end{problem}

\begin{solution}
  (b)
  $$
    \begin{array}{|c|c|c|c|c|c|c|c|c|}
      \hline         & (011100) & (011011) & (111011) & (100011) & (000000) & (010101) & (110100) & (110011) \\
      \hline(011100) & 0        & 3        & 4        & 6        & 3        & 2        & 2        & 5        \\
      \hline(011011) & 3        & 0        & 1        & 3        & 4        & 3        & 5        & 2        \\
      \hline(111011) & 4        & 1        & 0        & 2        & 5        & 4        & 4        & 1        \\
      \hline(100011) & 6        & 3        & 2        & 0        & 3        & 4        & 4        & 1        \\
      \hline(000000) & 3        & 4        & 5        & 3        & 0        & 3        & 3        & 4        \\
      \hline(010101) & 2        & 3        & 4        & 4        & 3        & 0        & 2        & 3        \\
      \hline(110100) & 2        & 5        & 4        & 4        & 3        & 2        & 0        & 3        \\
      \hline(110011) & 5        & 2        & 1        & 1        & 4        & 3        & 3        & 0        \\
      \hline
    \end{array}
  $$\\
  $d_{min}=1$\\
  (d)
  $$
    \begin{array}{|c|c|c|c|c|c|c|c|c|}
      \hline         & (011100) & (011011) & (111011) & (100011) & (000000) & (010101) & (110100) & (110011) \\
      \hline(011100) & 0        & 3        & 4        & 6        & 3        & 2        & 2        & 5        \\
      \hline(011011) & 3        & 0        & 1        & 3        & 4        & 3        & 5        & 2        \\
      \hline(111011) & 4        & 1        & 0        & 2        & 5        & 4        & 4        & 1        \\
      \hline(100011) & 6        & 3        & 2        & 0        & 3        & 4        & 4        & 1        \\
      \hline(000000) & 3        & 4        & 5        & 3        & 0        & 3        & 3        & 4        \\
      \hline(010101) & 2        & 3        & 4        & 4        & 3        & 0        & 2        & 3        \\
      \hline(110100) & 2        & 5        & 4        & 4        & 3        & 2        & 0        & 3        \\
      \hline(110011) & 5        & 2        & 1        & 1        & 4        & 3        & 3        & 0        \\
      \hline
    \end{array}
  $$\\
  $d_{min}=2$
\end{solution}
%%%%%%%%%%%%%%%

%%%%%%%%%%%%%%%
\begin{problem}[TJ 8-7(c,d)]
\end{problem}

\begin{solution}
  (c)\\
  Null(H):
  (00000)(00100)(11010)(11110)(11001)(11101)(00011)(00111)\\
  (5,3)-block\\
  Generator:\\
  $$
    G=\left[\begin{array}{lll}
        0 & 1 & 1 \\
        0 & 1 & 1 \\
        1 & 0 & 0 \\
        0 & 1 & 0 \\
        0 & 0 & 1
      \end{array}\right]
  $$\\
  (d)\\
  Null(H):\\
  (0000000)(0001111)(0010110)(0011001)\\
  (0100101)(0101010)(0110011)(0111100)\\
  (1000011)(1001100)(1010101)(1011010)\\
  (1100110)(1101001)(1110000)(1111111)\\
  (7,4)-block\\
  Generator:
  $$
    G=\left[\begin{array}{llll}
        1 & 0 & 0 & 0 \\
        0 & 1 & 0 & 0 \\
        0 & 0 & 1 & 0 \\
        1 & 1 & 0 & 1 \\
        1 & 0 & 1 & 1 \\
        0 & 1 & 1 & 1 \\
        0 & 0 & 0 & 1
      \end{array}\right]
  $$
\end{solution}
%%%%%%%%%%%%%%%

%%%%%%%%%%%%%%%
\begin{problem}[TJ 8-9]
\end{problem}

\begin{solution}
  $H(01111)^T=(001)^T\Rightarrow (01111)->(01101)$\\
  $H(10101)^T=(110)^T\Rightarrow multipl errors$\\
  $H(01110)^T=(110)^T\Rightarrow multipl errors$\\
  $H(00011)^T=(110)^T\Rightarrow multipl errors$\\
\end{solution}
%%%%%%%%%%%%%%%

%%%%%%%%%%%%%%%
\begin{problem}[TJ 8-11(b,d)]
\end{problem}

\begin{solution}
  (b)
  这是标准奇偶校验矩阵。相应的标准生成矩阵:
  $$
    \left(\begin{array}{ll}
      1 & 0 \\
      0 & 1 \\
      0 & 1 \\
      1 & 1 \\
      0 & 1 \\
      1 & 1
    \end{array}\right)
  $$\\
  可以至少纠错1位、检测2位.\\
  (d)这是标准奇偶校验矩阵.相应的标准生成矩阵:
  $$
    \left(\begin{array}{lll}
      1 & 0 & 0 \\
      0 & 1 & 0 \\
      0 & 0 & 1 \\
      0 & 0 & 0 \\
      0 & 1 & 1 \\
      1 & 0 & 1 \\
      0 & 1 & 1
    \end{array}\right)
  $$\\
  可以检测1位,不可以纠错.
\end{solution}
%%%%%%%%%%%%%%%

%%%%%%%%%%%%%%%
\begin{problem}[TJ 8-13]
\end{problem}

\begin{solution}
  (a)$(001)^T$\\
  (b)$(101)^T$\\
  (c)$(111)^T$\\
  (d)$(011)^T$\\
\end{solution}
%%%%%%%%%%%%%%%

%%%%%%%%%%%%%%%
\begin{problem}[TJ 8-19]
\end{problem}

\begin{solution}
  (1)群$C$中权重都为奇数。\\
  因为$e\in C\land w(e)=0$,显然不成立。\\
  (2)群$C$中权重都为偶数。\\
  考虑群$C=\{e\}$,此时显然成立,故存在权重都为偶数的情况。\\
  (3)群$C$中权重有奇有偶。\\
  考虑$c\in C_{odd}$,构造函数$f:C_{even}\to C_{odd}$ by $x\to x+c$其中$x\in C_{even}$(显然$x+c\in C_{odd}$)。\\
  one to one:\\
  $\forall x_1, x_2\in C_{even}, x_1+ c = x_2 + c \to x_1 = x_2$\\
  onto:\\
  $\forall y\in C_{odd}, \exists x = y + c^{-1}\in C_{even}, st.x + c = y$\\
  So $|C_{even}|=|C_{odd}|$, 那么它们中的一半都有偶数权重\\

  \noindent 所以,每个码字的权重都是偶数,或者恰好一半的码字具有偶数权重。
\end{solution}
%%%%%%%%%%%%%%%

%%%%%%%%%%%%%%%
\begin{problem}[TJ 8-21]
\end{problem}

\begin{solution}
  (a)error-correcting linear code\\
  假设$H$矩阵为$m\times n$的,对$2^7=128$进行编码时,需要满足
  $$ \left\{
    \begin{aligned}
      n-m =  7        \\
      n \leq 2^ m - 1 \\
    \end{aligned}
    \right.
  $$
  $m=4,n=11$为符合条件的最小正整数解。则最小的generator matrix为$11\times 7$。\\
  同理,当对$2^8=256$进行编码时,需要满足
  $$ \left\{
    \begin{aligned}
      n-m =  8        \\
      n \leq 2^ m - 1 \\
    \end{aligned}
    \right.
  $$
  $m=4,n=12$为符合条件的最小正整数解。则最小的generator matrix为$12\times 8$。\\

  \noindent(b)only error detection\\
  假设$H$矩阵为$m\times n$的,对$2^7=128$进行编码时,需要满足
  $$ \left\{
    \begin{aligned}
      n-m =  7 \\
      n \geq 1 \\
    \end{aligned}
    \right.
  $$
  $m=1,n=8$为符合条件的最小正整数解。则最小的generator matrix为$8\times 7$。\\
  同理,当对$2^8=256$进行编码时,需要满足
  $$ \left\{
    \begin{aligned}
      n-m =  8 \\
      n \geq 1 \\
    \end{aligned}
    \right.
  $$
  $m=1,n=9$为符合条件的最小正整数解。则最小的generator matrix为$9\times 8$。\\
\end{solution}
%%%%%%%%%%%%%%%

%%%%%%%%%%%%%%%
\begin{problem}[TJ 8-22]
\end{problem}

\begin{solution}
  (1)three information position:\\
  $$H=\begin{pmatrix}1 & 1 & 1 & 1 \\
    \end{pmatrix}$$
  $$G=\begin{pmatrix}1 & 0 & 0 \\
               0 & 1 & 0 \\
               0 & 0 & 1 \\
               1 & 1 & 1 \\
    \end{pmatrix}$$

  \noindent (2)seven information position:\\
  $$H=\begin{pmatrix}1 & 1 & 1 & 1 & 1 & 1 & 1 & 1 \\
    \end{pmatrix}$$
  $$G=\begin{pmatrix}1 & 0 & 0 & 0 & 0 & 0 & 0 \\
               0 & 1 & 0 & 0 & 0 & 0 & 0 \\
               0 & 0 & 1 & 0 & 0 & 0 & 0 \\
               0 & 0 & 0 & 1 & 0 & 0 & 0 \\
               0 & 0 & 0 & 0 & 1 & 0 & 0 \\
               0 & 0 & 0 & 0 & 0 & 1 & 0 \\
               0 & 0 & 0 & 0 & 0 & 0 & 1 \\
               1 & 1 & 1 & 1 & 1 & 1 & 1
    \end{pmatrix}$$
\end{solution}
%%%%%%%%%%%%%%%

%%%%%%%%%%%%%%%
\begin{problem}[TJ 8-23]
\end{problem}

\begin{solution}
  (a)
  假设有$n-m$位information bits, $m$位check bits
  $$ \left\{
    \begin{aligned}
      n-m =  20       \\
      n \leq 2^ m - 1 \\
    \end{aligned}
    \right.
  $$
  解得$m\geq 5$\\
  (b)
  假设有$n-m$位information bits, $m$位check bits
  $$ \left\{
    \begin{aligned}
      n-m =  32       \\
      n \leq 2^ m - 1 \\
    \end{aligned}
    \right.
  $$
  解得$m\geq 6$
\end{solution}
%%%%%%%%%%%%%%%

%%%%%%%%%%%%%%%%%%%%
\beginoptional

%%%%%%%%%%%%%%%


%%%%%%%%%%%%%%%%%%%%
\beginot
%%%%%%%%%%%%%%%
\begin{ot}[各种花式距离]
  请查阅资料,介绍曼哈顿距离、欧几里得距离、契比雪夫距离分别是什么意思,他们的典型应用是什么。你还有哪些创意,来定义二进制位串之间的距离?
\end{ot}

% \begin{solution}
% \end{solution}
%%%%%%%%%%%%%%%

%%%%%%%%%%%%%%%
\begin{ot}[编码率]
  解释什么是编码率,分析hamming码的最大编码率,分析还有比hamming码编码率更好的方法吗?
\end{ot}


% \begin{solution}
% \end{solution}
%%%%%%%%%%%%%%%


% \vspace{0.50cm}
%%%%%%%%%%%%%%%
% \begin{ot}[]
% 
%   \noindent 参考资料:
%   \begin{itemize}
%     \item 
%   \end{itemize}
% \end{ot}

% \begin{solution}
% \end{solution}
%%%%%%%%%%%%%%%

%%%%%%%%%%%%%%%%%%%%
% 如果没有需要订正的题目,可以把这部分删掉

% \begincorrection
%%%%%%%%%%%%%%%%%%%%

%%%%%%%%%%%%%%%%%%%%
% 如果没有反馈,可以把这部分删掉
\beginfb

% 你可以写
% ~\footnote{优先推荐 \href{problemoverflow.top}{ProblemOverflow}}:
% \begin{itemize}
%   \item 对课程及教师的建议与意见
%   \item 教材中不理解的内容
%   \item 希望深入了解的内容
%   \item $\cdots$
% \end{itemize}
%%%%%%%%%%%%%%%%%%%%
% \bibliography{2-5-solving-recurrence}
% \bibliographystyle{plainnat}
%%%%%%%%%%%%%%%%%%%%
\end{document}