% 2-15-rb-tree.tex

%%%%%%%%%%%%%%%%%%%%
\documentclass[a4paper, justified]{tufte-handout}

\input{hw-preamble} % feel free to modify this file
%%%%%%%%%%%%%%%%%%%%
\title{第3-15讲: 矩阵计算}
\me{林凡琪}{211240042}{}{}
\date{\zhtoday} % or like 2019年9月13日
%%%%%%%%%%%%%%%%%%%%
\begin{document}
\maketitle
%%%%%%%%%%%%%%%%%%%%
\noplagiarism % always keep this line
%%%%%%%%%%%%%%%%%%%%
\begin{abstract}
  % \begin{center}{\fcolorbox{blue}x{yellow!60}{\parbox{0.65\textwidth}{\large 
  %   \begin{itemize}
  %     \item 
  %   \end{itemize}}}}X
  % \end{center}
\end{abstract}
%%%%%%%%%%%%%%%%%%%%
\beginrequired

%%%%%%%%%%%%%%%
\begin{problem}[TC 28.1-2]
\end{problem}

\begin{solution}
  $$
    \left(\begin{array}{ccc}
      4  & -5 & 6  \\
      8  & -6 & 7  \\
      12 & -7 & 12
    \end{array}\right)=\left(\begin{array}{ccc}
      1 & 0 & 0 \\
      2 & 1 & 0 \\
      3 & 2 & 1
    \end{array}\right)\left(\begin{array}{ccc}
      4 & -5 & 6  \\
      0 & 4  & -5 \\
      0 & 0  & 4
    \end{array}\right)
  $$
\end{solution}
%%%%%%%%%%%%%%%

%%%%%%%%%%%%%%%
\begin{problem}[TC 28.1-3]
\end{problem}

\begin{solution}
  已知
  $$
    A=\left(\begin{array}{ccc}
      1 & 5 & 4 \\
      2 & 0 & 3 \\
      5 & 8 & 2
    \end{array}\right)\\
    b = \left(\begin{array}{c}
      12 \\
      9  \\
      5
    \end{array}\right)$$
  \\LUP分解是:
  \\
  $$L=\left(\begin{array}{ccc}
      1   & 0        & 0 \\
      0.2 & 1        & 0 \\
      0.4 & -3.2/3.4 & 1
    \end{array}\right),
    U=\left(\begin{array}{ccc}
      5 & 8   & 2                     \\
      0 & 3.4 & 3.6                   \\
      0 & 0   & 2.2+\frac{11.52}{3.4}
    \end{array}\right),
    P=\left(\begin{array}{ccc}
      0 & 0 & 1 \\
      1 & 0 & 0 \\
      0 & 1 & 0
    \end{array}\right).$$\\
  进行代换, 解出$Ly=Pb$中的$y$:
  $$\left(\begin{array}{ccc}
      1   & 0        & 0 \\
      0.2 & 1        & 0 \\
      0.4 & -3.2/3.4 & 1
    \end{array}\right)
    \left(\begin{array}{c}
      y_1 \\
      y_2 \\
      y_3
    \end{array}\right)=
    \left(\begin{array}{c}
      5  \\
      12 \\
      9
    \end{array}\right),\\
  $$
  计算$y_1,y_2,y_3$, 我们得到
  $$
    y=\left(\begin{array}{c}
      5  \\
      11 \\
      7+35.2/3.4
    \end{array}\right)$$\\
  因为 $Ux=y$\\
  $$
    \left(\begin{array}{ccc}
      5 & 8   & 2                     \\
      0 & 3.4 & 3.6                   \\
      0 & 0   & 2.2+\frac{11.52}{3.4}
    \end{array}\right)
    \left(\begin{array}{c}
      x_1 \\
      x_2 \\
      x_3
    \end{array}\right)=
    \left(\begin{array}{c}
      5  \\
      11 \\
      7 + 35.2/34
    \end{array}\right)$$
  所以 $x=\left(\begin{array}{c}
      -3/19 \\
      -1/19 \\
      49/19
    \end{array}\right)$
\end{solution}
%%%%%%%%%%%%%%%

%%%%%%%%%%%%%%%
\begin{problem}[TC 28.1-6]
\end{problem}

\begin{solution}
  零矩阵始终具有LU分解,将L作为任意单位下三角矩阵,将U作为零矩阵,即上三角矩阵。
\end{solution}
%%%%%%%%%%%%%%%

%%%%%%%%%%%%%%%
\begin{problem}[TC 28.1-7]
\end{problem}

\begin{solution}
  对于LU-DECOMPOSITION,确实是有必要的.\\
  如果没有运行最外层for循环的第六行,$u_{nn}$将会保留初始值0,而不是等于$a_{nn}$\\
  但实际上任何非奇异矩阵的LU-DECOMPOSITION分解都必须具有行列式值为正的L和U.但是如果$u_{nn}=0$,那么$U$的行列式将会为0.\\
  K=N时发生的最外层for循环的迭代不会改变最终答案,因为$\pi$必须是一个on a single element的permutation, 因此它不能是非平凡的.(并且第十六行的for循环根本不会运行)
\end{solution}
%%%%%%%%%%%%%%%


%%%%%%%%%%%%%%%
\begin{problem}[TC 28.2-1]
\end{problem}

\begin{solution}
  It is obvious that multiplying and squaring matrices have essentially the same difficulty, since squaring a matrix is just multiplying it by itself.
\end{solution}
%%%%%%%%%%%%%%%

%%%%%%%%%%%%%%%
\begin{problem}[TC 28.2-2]
\end{problem}

\begin{solution}
  设A是一个$n \times n$的矩阵\\
  不失一般性地,我们假设$n=2^k$,并且规定$L(n/2)\leq cL(n)$其中$c<1/2$,并且$L(n)$是涨到$n \times n$矩阵地LUP分解所需要的时间.\\
  首先,将A分解为\\
  $A=\left(\begin{array}{ccc}
        A_1 \\
        A_2
      \end{array}
    \right)$
  再让$A_1=L_1U_1P_1$成为$A_1$的一个LUP分解.\\
  把$U_1$分解为$A_2P_1^{-1}$,其中$U_1=[\overline{U}_1|B]$并且$A_2P_1^{-1}=[C|D]$.\\
  设$F=D-C\overline{U}_1^{-1}B$. \\
  $$\rightarrow
    A=\left(\begin{array}{ccc}
        L_1                  & 0       \\
        C\overline{U}_1^{-1} & I_{n/2}
      \end{array}
    \right)
    \left(\begin{array}{ccc}
        \overline{U}_1 & B \\
        0              & F
      \end{array}
    \right)P_1$$\\
  现在让$F=L_2U_2P_2$为一个F的LUP分解,并且令$\overline{P}=\left(\begin{array}{ccc}
        I_{n/2} & 0 \\
        0       & F
      \end{array}\right)$.
  $$\rightarrow
    A=\left(\begin{array}{ccc}
        L_1                  & 0   \\
        C\overline{U}_1^{-1} & L_2
      \end{array}
    \right)
    \left(\begin{array}{ccc}
        \overline{U}_1 & BP_2^{-1} \\
        0              & U_2
      \end{array}
    \right)\overline{P}P_1$$\\
  这就是A的一个LUP分解.过程中我们完成了两次LUP分解,一个常数的矩阵乘法和一个常数的矩阵逆运算.因为矩阵逆运算和矩阵乘法等价,所以我们得出运行时间是$O(M(n))$

\end{solution}
%%%%%%%%%%%%%%%

%%%%%%%%%%%%%%%
\begin{problem}[TC 28.2-3]
\end{problem}

\begin{solution}
  从problem 28.2-2可知,我们能找到一个时间复杂度为$O(M(n))$的LU分解算法.所以我们运行那个算法并且将U对角线上的所有元素相乘,这将会是原矩阵的行列式.
\end{solution}
%%%%%%%%%%%%%%%

%%%%%%%%%%%%%%%
\begin{problem}[TC 28.3-1]
\end{problem}

\begin{solution}
  设$e_i$是一个除了第i位是1,其他位全是0的向量.\\
  $Ae_i$获取A的每一行并提取其中的第i行,并将这些值放进列向量.然后将其乘以左侧的$e^{T}_i$,得出该quantity的第i行.\\
  因为$e_i$是非零的,那么quantity一定是正的.因为我们对每个i都进行一样的操作,所以每个在对角线上的元素都是正数.
\end{solution}
%%%%%%%%%%%%%%%

%%%%%%%%%%%%%%%
\begin{problem}[TC 28.3-3]
\end{problem}

\begin{solution}
  反证法:\\
  假设存在一个$a_{ij},i\neq j$是最大的元素.设$e_i$是一个除了第i位是1,其他位全是0的向量.\\
  我们考虑$(e_i-e_j)^TA(e_i-e_j)$.\\
  $$
    \begin{aligned}
      \left(e_i-e_j\right)^{\mathrm{T}} A\left(e_i-e_j\right) & =a_{i i}-a_{i j}-a_{j i}+a_{j j}  \\
                                                              & =a_{i i}+a_{j j}-2 a_{i j} \leq 0
    \end{aligned}
  $$
  但A是正的,所以我们的假设错误.\\
  得证.
\end{solution}
%%%%%%%%%%%%%%%

%%%%%%%%%%%%%%%%%%%%%
%\beginoptional
%%%%%%%%%%%%%%%%
%%%%%%%%%%%%%%%%
%\begin{problem}[TC Problem 28-1]
%\end{problem}
%
%\begin{solution}
%\end{solution}
%%%%%%%%%%%%%%%%

%%%%%%%%%%%%%%%%%%%%%
\beginot
%%%%%%%%%%%%%%%%
\vspace{0.50cm}
%%%%%%%%%%%%%%
\begin{ot}[TC Problem 28-1]

\end{ot}

% \begin{solution}
% \end{solution}
%%%%%%%%%%%%%%%

%%%%%%%%%%%%%%%%%%%%
% 如果没有需要订正的题目,可以把这部分删掉

% \begincorrection
%%%%%%%%%%%%%%%%%%%%

%%%%%%%%%%%%%%%%%%%%
% 如果没有反馈,可以把这部分删掉
\beginfb

% 你可以写
% ~\footnote{优先推荐 \href{problemoverflow.top}{ProblemOverflow}}:
% \begin{itemize}
%   \item 对课程及教师的建议与意见
%   \item 教材中不理解的内容
%   \item 希望深入了解的内容
%   \item $\cdots$
% \end{itemize}
%%%%%%%%%%%%%%%%%%%%
% \bibliography{2-5-solving-recurrence}
% \bibliographystyle{plainnat}
%%%%%%%%%%%%%%%%%%%%
\end{document}