% 2-15-rb-tree.tex

%%%%%%%%%%%%%%%%%%%%
\documentclass[a4paper, justified]{tufte-handout}

\input{hw-preamble} % feel free to modify this file
%%%%%%%%%%%%%%%%%%%%
\title{第4-9讲: NP完全理论初步}
\me{ 林凡琪}{211240042 }{}{}
\date{\zhtoday} % or like 2019年9月13日
%%%%%%%%%%%%%%%%%%%%
\begin{document}
\maketitle
%%%%%%%%%%%%%%%%%%%%
\noplagiarism % always keep this line
%%%%%%%%%%%%%%%%%%%%
\begin{abstract}
  % \begin{center}{\fcolorbox{blue}{yellow!60}{\parbox{0.65\textwidth}{\large 
  %   \begin{itemize}
  %     \item 
  %   \end{itemize}}}}
  % \end{center}
\end{abstract}
%%%%%%%%%%%%%%%%%%%%
\beginrequired

%%%%%%%%%%%%%%%
\begin{problem}[TC  34.1-5]
\end{problem}

\begin{solution}
  (1)
  假设执行$k$次子例程。\\
  则第一次输入的规模为$O(n)$,输出的规模为$O(n^c)$,时间为$O(n^c)$。\\
  则第$k$次输入的规模为$O(n^{c^{k-1}})$,输出的规模为$O(n^{c^k})$,时间为$O(n^{c^k})$。\\
  其中$c^k$为常数,为多项式时间。\\

  \noindent 若执行$n$次子例程。\\
  则第$k$次输入的规模为$O(n^{c^{n-1}})$,输出的规模为$O(n^{c^n})$,时间为$O(n^{c^n})$。\\
  不为多项式时间。
\end{solution}
%%%%%%%%%%%%%%%

%%%%%%%%%%%%%%%
\begin{problem}[TC 34.2-3]
\end{problem}

\begin{solution}
  将图中顶点标号为$v_1,..,v_n$\\
  从$v_1$开始依次做如下操作:	\\
  设和$v_1$相连的边集为$E_1$.\\
  任取$e_1,e_2\in E_1$,检验图$G`=(V,(E-E_1)\lor(e_1,e_2))$是否有哈密顿回路。\\
  因为图$G$存在哈密顿回路,则总存在上述$G`$使得$G`$中存在哈密顿回路。\\
  找到存在哈密顿回路的$G`$后,令$G=G`$,继续对顶点$v_2,v_3,...,v_n$做上述操作。\\
  显然若哈密顿回路问题是$P$问题,则上述算法可在多项式时间内结束,并找到顶点集。\\
\end{solution}
%%%%%%%%%%%%%%%

%%%%%%%%%%%%%%%
\begin{problem}[TC 34.2-4]
\end{problem}

\begin{solution}
  假设$L_1, L_2\in NP$,$A_1$, $A_2$为 $L_1, L_2$ 的一个多项式验证算法。\\
  对于union: \\
  对于输入的(x,y),算法A: 若$A_1(x,y)$可接受,或$A_2(x,y)$可接受,则判定为接受,否则拒绝。\\
  显然,其可在多项式时间验证,属于$NP$.\\
  对于intersection: \\
  对于输入的(x,y),算法A: 若$A_1(x,y)$可接受,并且$A_2(x,y)$可接受,则判定为接受,否则拒绝。\\
  显然,其可在多项式时间验证,属于$NP$.\\
  对于concatenation: \\
  对于输入的(x,y),算法A: 若$A_1(x[1,...,i],y[1,...,j])$可接受,并且$A_2(x[i+1,...,n],y[j+1,...,m])$可接受,则判定为接受,否则拒绝。其中$n=|x|, m = |y|$。\\
  显然,其可在多项式时间验证,属于$NP$.\\
  Kleene star:\\
  在concatenation中,我们将$(x,y)$各分成$2$段。在该算法中,我们参照该做法,将$x$分为不超过
  在concatenation的验证中,我们将$x$各分成不超过$|x|$段,$y$分成不超过$|y|$段,若每段都可接受,则判定为接受,否则拒绝。\\
  讨论:\\
  若$L\in P$,则$\overline{L}\in P$。 若$L\in NPC$,则目前没有确定的答案,对$\overline{L}$给出明确的界定。
\end{solution}
%%%%%%%%%%%%%%%

%%%%%%%%%%%%%%%
\begin{problem}[TC 34.3-2]
\end{problem}

\begin{solution}
  若$L_1\leq L_2$,则存在多项式时间的转换函数$f(x)$,其中$x\in L_1, f(x) \in L_2$\\
  若$L_2\leq L_3$,则存在多项式时间的转换函数$g(x)$,其中$x\in L_2, g(x) \in L_3$\\
  从而有$g(f(x))$为多项式时间的转换函数,其中$x\in L_1, g(f(x)) \in L_3$\\
  故$L_1\leq L_3$
\end{solution}
%%%%%%%%%%%%%%%

%%%%%%%%%%%%%%%
\begin{problem}[TC 34.4-3]
\end{problem}

\begin{solution}
  对于有 $n$个变量的布尔公式, 列出其真值表的时间复杂度为$\Omega(2^n)$,故该方法不能多项式时间规约
\end{solution}
%%%%%%%%%%%%%%%

%%%%%%%%%%%%%%%
\begin{problem}[TC 34.2-11]
\end{problem}

\begin{solution}
  数学归纳法:\\
  若 n = 3,则 其为$K_3$,显然存在哈密顿回路\\
  假设对于所有点数为$k$ $(3 \leq k < n)$的连通图都成立\\
  则当$k=n$时\\
  令 T 表示 G 的任意生成树,取任意一点$x$,并且移除,可以得到 T 的若干个连通块 $G_1, G_2, · · · , G_l$\\
  对于任意 $|G_i|>1$,则选出$u, v\in G_i$,满足$e(u,x)\in G_i$,$e(u,v)\in G_i$。根据归纳假设$G_i$ 包含一条由$u$ 到 $v$ 的哈密顿通路。\\
  若$|G_i|=1$,则将该点视为该联通块内的哈密顿通路。\\
  显然,从$v$出发,依次遍历每个联通块,在每个联通块内按哈密顿通路遍历,即可找到哈密顿回路。\\
  综上得证.
\end{solution}
%%%%%%%%%%%%%%%

%%%%%%%%%%%%%%%
\begin{problem}[TC 34.4-7]
\end{problem}

\begin{solution}
  设有$n$个变量,$m$个子句,对于变量$x_1,x_2,...,x_n$,每个变量创建两个顶点$x_i$和$\neg x_i$。\\
  则若有子句$a\lor b$,则添加$ (\neg a, b),(\neg b, a)$两条边。\\
  tarjan缩点,若存在$x_i.\neg x_i$在同一强连通分量里则不可满足。否则可满足。\\
  复杂度$O(n+m)$
\end{solution}
%%%%%%%%%%%%%%%

%%%%%%%%%%%%%%%
\begin{problem}[TC 34.5-6]
\end{problem}

\begin{solution}
  (1)\\
  首先,我们证明哈密顿路径问题是NP问题。若给出证书$y$,我们只需检验$y$是否构成排列,并且$y$中相领点之间是否存在边即可。显然可多项式时间验证,故为NP问题。\\
  (2)\\
  证明可将哈密顿回路问题规约到哈密顿通路。\\
  对于图中的每一条边$e(u,v)\in G$\\
  构造$G_1$, $V(G_1)=V(G)\lor \{u_{new}, v_{new}\}$,$E(G_1) = E(G)\lor\{(u_{new}, u), (v_{new}, v)\}/\{(u,v)\}$\\
  下证$G$中存在哈密顿回路当且仅当存在符合构造的$G_1$,其存在哈密顿通路。\\
  若在原图$G$中存在哈密顿回路,则显然若$(u,v)$属于哈密顿圈,在删去$(u,v)$构造的$G_1$中,存在一条$u_{new}$出发,$v_{new}$结束的哈密顿路径。\\
  若在$G_1$中存在哈密顿通路,则一定是条$u_{new}$出发,$v_{new}$结束的哈密顿路径。原图$G$中显然存在哈密顿回路。\\
  调用需要$O(m)$次,显然调用为多项式时间规约。\\
  得证。
\end{solution}
%%%%%%%%%%%%%%%
%%%%%%%%%%%%%%%%%%%%
\beginoptional
%%%%%%%%%%%%%%%

%%%%%%%%%%%%%%%
\begin{problem}[TC 34.5-2]
\end{problem}

\begin{solution}
\end{solution}
%%%%%%%%%%%%%%%

%%%%%%%%%%%%%%%%%%%%
\beginot
%%%%%%%%%%%%%%%
\begin{ot}[NP]
  The name ``NP'' stands for ``nondeterministic polynomial time.'' The class NP was originally studied	in the context of nondeterminism, but this book uses the somewhat simpler yet equivalent notion of verification.
  Hopcroft and Ullman [180] give a good presentation of NP-completeness in terms of nondeterministic models of computation.

  阅读TC参考文献[180],介绍Hopcroft等人的NP问题定义,并说说两种定义方法是一致的吗?为什么?

  \href{http://ce.sharif.edu/courses/94-95/1/ce414-2/resources/root/Text\%20Books/Automata/John\%20E.\%20Hopcroft,\%20Rajeev\%20Motwani,\%20Jeffrey\%20D.\%20Ullman-Introduction\%20to\%20Automata\%20Theory,\%20Languages,\%20and\%20Computations-Prentice\%20Hall\%20(2006).pdf}{John E. Hopcroft and Jeffrey D. Ullman. Introduction to Automata Theory, Languages, and	Computation. Addison-Wesley, 1979.
  }
\end{ot}

% \begin{solution}
% \end{solution}
%%%%%%%%%%%%%%%

%%%%%%%%%%%%%%%
\begin{ot}[TSP is NP-hard]
  证明旅行商问题是NPC问题.
\end{ot}


% \begin{solution}
% \end{solution}
%%%%%%%%%%%%%%%


% \vspace{0.50cm}
%%%%%%%%%%%%%%%
% \begin{ot}[]
% 
%   \noindent 参考资料:
%   \begin{itemize}
%     \item 
%   \end{itemize}
% \end{ot}

% \begin{solution}
% \end{solution}
%%%%%%%%%%%%%%%

%%%%%%%%%%%%%%%%%%%%
% 如果没有需要订正的题目,可以把这部分删掉

% \begincorrection
%%%%%%%%%%%%%%%%%%%%

%%%%%%%%%%%%%%%%%%%%
% 如果没有反馈,可以把这部分删掉
\beginfb

% 你可以写
% ~\footnote{优先推荐 \href{problemoverflow.top}{ProblemOverflow}}:
% \begin{itemize}
%   \item 对课程及教师的建议与意见
%   \item 教材中不理解的内容
%   \item 希望深入了解的内容
%   \item $\cdots$
% \end{itemize}
%%%%%%%%%%%%%%%%%%%%
% \bibliography{2-5-solving-recurrence}
% \bibliographystyle{plainnat}
%%%%%%%%%%%%%%%%%%%%
\end{document}