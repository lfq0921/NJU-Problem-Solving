% 2-15-rb-tree.tex

%%%%%%%%%%%%%%%%%%%%
\documentclass[a4paper, justified]{tufte-handout}

\input{hw-preamble} % feel free to modify this file
%%%%%%%%%%%%%%%%%%%%
\title{第4-8讲: 形式化}
\me{ 211240042}{林凡琪 }{}{}
\date{\zhtoday} % or like 2019年9月13日
%%%%%%%%%%%%%%%%%%%%
\begin{document}
\maketitle
%%%%%%%%%%%%%%%%%%%%
\noplagiarism % always keep this line
%%%%%%%%%%%%%%%%%%%%
\begin{abstract}
  % \begin{center}{\fcolorbox{blue}{yellow!60}{\parbox{0.65\textwidth}{\large 
  %   \begin{itemize}
  %     \item 
  %   \end{itemize}}}}
  % \end{center}
\end{abstract}
%%%%%%%%%%%%%%%%%%%%
\beginrequired

%%%%%%%%%%%%%%%
\begin{problem}[JH  2.3.1.8]
\end{problem}

\begin{solution}
  $$
    \begin{gathered}
      {\left[\begin{array}{cccc}
            a_{11}  & a_{12}  & \cdots & a_{1 n} \\
            a_{21}  & a_{22}  & \cdots & a_{2 n} \\
            \cdots  & \cdots  & \cdots & \cdots  \\
            a_{n 1} & a_{n 2} & \cdots & a_{n n}
          \end{array}\right]} \\
      \Downarrow \\
      \bar{a}_{11} \# \bar{a}_{12} \# \cdots \# \bar{a}_{1 n} \# \# \\
      \bar{a}_{21} \# \bar{a}_{22} \# \cdots \# \bar{a}_{2 n} \# \# \\
      \cdots \\
      \bar{a}_{n 1} \# \bar{a}_{n 2} \# \cdots \# \bar{a}_{n n} \# \#
    \end{gathered}
  $$
  $\bar{a}_{i j} \in\{0,1\}^{+}$is the bianary representation of $a_{i j}$
\end{solution}
%%%%%%%%%%%%%%%



%%%%%%%%%%%%%%%
\begin{problem}[JH  2.3.3.8]
\end{problem}

\begin{solution}
  1.
  $$
    \begin{aligned}
      H C=\left\{w \in\{0,1, \#\}^* \quad \mid w\right. & \text{represents a graph that} \\&\text{contains a Hamiltonian cycle}\}
    \end{aligned}
  $$
  Given $w \in H C$, let $c$ be a certificate of $w$, i.e. $c$ is any path in $w$, where $|c|^{\prime}=n$
  A verifier checks the followings:
  $>\left(c_i, c_{i+1}\right) \in w . E$, for $1 \leq i<n$
  $>\left(c_n, c_1\right) \in w \cdot E$
  $>c_i \neq c_j$ for $1 \leq i, j \leq n, i \neq j$\\
  2.
  $$
    \begin{aligned}
      V C P=\left\{u \# w \in\{0,1, \#\}^{+} \quad\right. & \mid u \in\{0,1\}^{+} \text {and } w \text { represents }   \\
                                                          & \text { a graph that contains a vertext }                   \\
                                                          & \text { cover of size } \operatorname{Number}(\mathrm{u})\}
    \end{aligned}
  $$
  Given a graph $w$, and a certificate $c \subseteq w \cdot V$
  A verifier checks the following:
  $(i)|c|=N u m b e r(u)$
  $(ii)c$ covers all vertexes of $w$, i.e. $c \cup N(c)=w . V$
\end{solution}
%%%%%%%%%%%%%%%

%%%%%%%%%%%%%%%%%%%%
\beginoptional

%%%%%%%%%%%%%%%


%%%%%%%%%%%%%%%%%%%%
\beginot
%%%%%%%%%%%%%%%
\begin{ot}[Turing Machine]
  介绍一种确定性图灵机和一种非确定性图灵机模型.
\end{ot}

% \begin{solution}
% \end{solution}
%%%%%%%%%%%%%%%

%%%%%%%%%%%%%%%
\begin{ot}[SAT]
  介绍判定问题SAT和优化问题Max-SAT及其形式描述,简单讨论一下它们为什么会``很难''.
\end{ot}


% \begin{solution}
% \end{solution}
%%%%%%%%%%%%%%%


% \vspace{0.50cm}
%%%%%%%%%%%%%%%
% \begin{ot}[]
% 
%   \noindent 参考资料:
%   \begin{itemize}
%     \item 
%   \end{itemize}
% \end{ot}

% \begin{solution}
% \end{solution}
%%%%%%%%%%%%%%%

%%%%%%%%%%%%%%%%%%%%
% 如果没有需要订正的题目,可以把这部分删掉

% \begincorrection
%%%%%%%%%%%%%%%%%%%%

%%%%%%%%%%%%%%%%%%%%
% 如果没有反馈,可以把这部分删掉
\beginfb

% 你可以写
% ~\footnote{优先推荐 \href{problemoverflow.top}{ProblemOverflow}}:
% \begin{itemize}
%   \item 对课程及教师的建议与意见
%   \item 教材中不理解的内容
%   \item 希望深入了解的内容
%   \item $\cdots$
% \end{itemize}
%%%%%%%%%%%%%%%%%%%%
% \bibliography{2-5-solving-recurrence}
% \bibliographystyle{plainnat}
%%%%%%%%%%%%%%%%%%%%
\end{document}