% 2-15-rb-tree.tex

%%%%%%%%%%%%%%%%%%%%
\documentclass[a4paper, justified]{tufte-handout}

\input{hw-preamble} % feel free to modify this file
%%%%%%%%%%%%%%%%%%%%
\title{第4-1讲: 群论初步}
\me{ 林凡琪}{211240042 }{}{}
\date{\zhtoday} % or like 2019年9月13日
%%%%%%%%%%%%%%%%%%%%
\begin{document}
\maketitle
%%%%%%%%%%%%%%%%%%%%
\noplagiarism % always keep this line
%%%%%%%%%%%%%%%%%%%%
\begin{abstract}
	% \begin{center}{\fcolorbox{blue}{yellow!60}{\parbox{0.65\textwidth}{\large 
	%   \begin{itemize}
	%     \item 
	%   \end{itemize}}}}
	% \end{center}
\end{abstract}
%%%%%%%%%%%%%%%%%%%%
\beginrequired

%%%%%%%%%%%%%%%
\begin{problem}[TJ 3-3]
Wrte out Cayley tables for groups formed by the symmetries of a rectangle and for $(\mathbb{Z}_4, +)$. How many elements are in each group? Are the groups the same? Why or why not?
\end{problem}

\begin{solution}
	Te symmetries of a rectangle with centroid at the origin and sides parallel to the coordinate axes are generated by reflections $\sigma_x$ in the x-axis and $\sigma_y$ in the y-axis.\\
	Their square is identity $e$ and their product (in either order) is the rotation $\rho $ of $\pi / 2 $ about the origin.\\
	$\rho \circ \sigma_y=(\sigma_x \circ \sigma_y) \circ \sigma_y=\sigma_x\circ (\sigma_y \circ \sigma_y) = \sigma_x \circ e = \sigma_x$\\
	Cayley tables:
	\newpage
	\begin{table}[]
		\begin{tabular}{|l||l|l|l|l|}
			\hline$\circ$    & $e$        & $\sigma_x$ & $\sigma_y$ & $\rho$     \\
			\hline \hline$e$ & $e$        & $\sigma_x$ & $\sigma_y$ & $\rho$     \\
			\hline$\sigma_x$ & $\sigma_x$ & $e$        & $\rho$     & $\sigma_y$ \\
			\hline$\sigma_y$ & $\sigma_y$ & $\rho$     & $e$        & $\sigma_x$ \\
			\hline$\rho$     & $\rho$     & $\sigma_y$ & $\sigma_x$ & $e$        \\
			\hline
		\end{tabular}
		\begin{tabular}{|c||c|c|c|c|}
			\hline$+$       & 0 & 1 & 2 & 3 \\
			\hline \hline 0 & 0 & 1 & 2 & 3 \\
			\hline 1        & 1 & 2 & 3 & 0 \\
			\hline 2        & 2 & 3 & 0 & 1 \\
			\hline 3        & 3 & 0 & 1 & 2 \\
			\hline
		\end{tabular}
	\end{table}
	These groups are not the same. While each symmetry has square the identity $e$; the square of 1 and 3 is 2; which is not the identity 0.
\end{solution}
%%%%%%%%%%%%%%%

%%%%%%%%%%%%%%%
\begin{problem}[TJ-3-7]
Let S = $\mathbb{R}$ \ \{-1\} and define a binary operation on S by $a * b = a+b+ab.$ Prove that$(S, *)$ is an abelian group.
\end{problem}

\begin{solution}
	To proof $(S,*)$ is a group, we must show that $(S,*)$ have the proposition of group.\\
	Closure:if $a,b\in S$, then $a*b \in S$.\\
	We prove the contrapositive: if $a*b \notin S$, either $a\notin S$ or $b\notin S$.\\
	If $a*b \notin S$, then $a*b = a+b+ab=-1$\\
	Adding 1 to both sides:\\
	$1+a+b+ab=(1+a)(1+b)=0$\\
	So $a=-1\notin S$ or $b=-1\notin S$\\
	Associativity:\\
	$a,b,c\in S, (a*b)*c=(a*b)+c+(a*b)c=(a+b+ab)+c+(a+b+ab)c=a+(b+c+bc)+a(b+c+bc)=a+(b*c)+a(b*c)=a*(b*c)$\\
	Identity element:0;\\
	$a\in S, a*0=a+0+a\times 0 = a$\\
	Inverse:\\
	For $a\in S$, the inverse element is $\frac{-a}{a+1}$\\
	$a *\left(\frac{-a}{a+1}\right)=a+\frac{-a}{a+1}+a\left(\frac{-a}{a+1}\right)=\frac{a(a+1)-a-a^2}{a+1}=0$\\
	So $(S,*)$ is a group.\\
	Commutativity:\\
	$\forall a,b \in S, a * b=a+b+ab$\\
	exchange the position of a, b.\\
	for addition and multipliation on $mathbb{R}$ is commutative, so $\rightarrow b*a=b+a+ba = a+b+ab = a*b$\\
	So $(S,*)$ is an abelian group.
\end{solution}
%%%%%%%%%%%%%%%

%%%%%%%%%%%%%%%
\begin{problem}[TJ 3-39]
Let $\mathbb{T} = \{z\in \mathbb{C}^*:|z|=1\}$. Prove that $\mathbb{T}$ is a subgroup of $\mathbb{C}^*$.
\end{problem}

\begin{solution}
	According to proposition 3.30:\\
	1.Identity:\\
	The identity of $\mathbb{C}^*$ is 1. And for $z\in \mathbb{T}$, we can suppose that $z = \cos x + i\sin x \rightarrow 1 \times z = 1 \times (\cos x + i\sin x)= \cos x + i\sin x = z$\\
	So the identity of $\mathbb{T}$ is also 1.\\
	2.We can suppose that $z_1, z_2\in \mathbb{T}, z_1 = \cos x + i\sin x, z_2 = \cos y + i\sin y$\\
	$z_1z_2 = \cos x\cos y -\sin x\sin y + i(\cos x \sin y +\sin x \cos y)=\cos(x+y)+i\sin(x+y) = z_3 \land |z_3|=1\rightarrow z_3 \in \mathbb{T}$\\
	3.We can suppose that $z\in \mathbb{T}, z = \cos x + i\sin x$\\
	$z^{-1}=\frac{\cos x-\sin xi}{\cos ^2x+\sin^2x}$\\
	$zz^{-1} = (\cos x + i\sin x)(\frac{\cos x-i\sin x}{\cos ^2x+\sin^2x}) = \frac{\cos ^2x+\sin^2x}{\cos ^2x+\sin^2x} = 1$\\
	$|z^{-1}| = \sqrt[2]{(\frac{\cos x}{\cos ^2x+\sin^2x})^2 + (\frac{-\sin xi}{\cos ^2x+\sin^2x})^2} = 1$\\
	So $z^{-1} \in \mathbb{T}$.
\end{solution}
%%%%%%%%%%%%%%%

%%%%%%%%%%%%%%%
\begin{problem}[TJ 3-42]
\end{problem}

\begin{solution}
	1.Identity:\\
	The identity of $G$ is
	$
		\begin{pmatrix}
			0 & 0 \\
			0 & 0 \\
		\end{pmatrix}
	$
	\\
	For $
		\begin{pmatrix}
			a & b \\
			c & d \\
		\end{pmatrix} \in H(a+d=0)$\\
	$\begin{pmatrix}
			a & b \\
			c & d \\
		\end{pmatrix}
		+
		\begin{pmatrix}
			0 & 0 \\
			0 & 0 \\
		\end{pmatrix}
		=
		\begin{pmatrix}
			a+0 & b+0 \\
			c+0 & d+0 \\
		\end{pmatrix}
		=
		\begin{pmatrix}
			a & b \\
			c & d \\
		\end{pmatrix} $\\
	So $\begin{pmatrix}
			0 & 0 \\
			0 & 0 \\
		\end{pmatrix}$ is the identity of $H$.\\
	2.We can suppose that $\begin{pmatrix}
			a & b \\
			c & d \\
		\end{pmatrix} \in H and
		\begin{pmatrix}
			x & y \\
			z & w \\
		\end{pmatrix} \in H, (a+d=0\land x+w=0)\\$
	$\begin{pmatrix}
		a & b \\
		c & d \\
	\end{pmatrix}
	+
	\begin{pmatrix}
		x & y \\
		z & w \\
	\end{pmatrix}
	=
	\begin{pmatrix}
		a+x & b+y \\
		c+z & d+w \\
	\end{pmatrix}
	$\\
		We can know that $(a+x)+(d+w)=(a+d)+(x+w)=0+0=0$\\
		So $\begin{pmatrix}
		a+x & b+y \\
		c+z & d+w \\
	\end{pmatrix} \in H
	$\\
		3.Let $\begin{pmatrix}
		a & b \\
		c & d \\
	\end{pmatrix}(a+d=0)$ denoted by A;\\
	$A^{-1}=\frac{1}{a d-b c}\left(\begin{array}{cc}d & -b \\ -c & a\end{array}\right) = \begin{pmatrix}
		\frac{d}{a d-b c}  & \frac{-b}{a d-b c} \\
		\frac{-c}{a d-b c} & \frac{a}{a d-b c}  \\
	\end{pmatrix}$\\
		And $\frac{d}{a d-b c}+\frac{a}{a d-b c} = \frac{a+d}{a d-b c} =\frac{0}{a d-b c}=0$\\
		So $A^{-1} \in H$
\end{solution}
%%%%%%%%%%%%%%%

%%%%%%%%%%%%%%%
\begin{problem}[TJ 3-49]
Let $a$ and $b$ be elements of a group $G$. If $a^4b = ba$ and $a^3 = e$, prove that $ab = ba$.
\end{problem}

\begin{solution}
	According to the usual laws of exponents: $a^4=a^3a$\\
	$\rightarrow a^4b=a^3ab = (a^3)ab=eab=ab$\\
	Since $a^4b=ba$, $\rightarrow ab=ba$
\end{solution}
%%%%%%%%%%%%%%%

%%%%%%%%%%%%%%%
\begin{problem}[TJ 3-51]
If $xy = x^{-1}y^{-1}$ for all $x$ and $y$ in $G$, prove that $G$ must be abelian.
\end{problem}

\begin{solution}
	According to proposition 3.19, $(ab)^{-1}=b^{-1}a^{-1}$\\
	Since $b^{-1}, a^{-1} \in G$ and $(ab)^{-1}=b^{-1}a^{-1}=a^{-1}b^{-1}$, $\forall a,b\in G, ab=ba$\\
	So G is abelian.
\end{solution}
%%%%%%%%%%%%%%%

%%%%%%%%%%%%%%%
\begin{problem}[TJ 4-1]
\end{problem}

\begin{solution}
	(a)False.\\
	Disprove:\\
	According to the corollary 4.14.\\
	The generators of $\mathbb{Z}_n$ are the integers r such that $1\leq r < n$ and gcd(r,n)=1.\\
	One of the generators is 49, and it is not prime.\\
	(b)False.\\
	Disprove:\\
	The multipliation table for U(8):\\
	$$
		\begin{array}{c|cccc}
			\cdot    & 1 & 3 & 5 & 7 \\
			\hline 1 & 1 & 3 & 5 & 7 \\
			3        & 3 & 1 & 7 & 5 \\
			5        & 5 & 7 & 1 & 3 \\
			7        & 7 & 5 & 3 & 1
		\end{array}
	$$\\
	$|1|=1$; $|3|=|5|=|7|=2$\\
	1, 3, 5, 7 are all not generator of U(8).
	(c)False.\\
	We can assume that $g$ is a generator of $\mathbb{Q}$, but it can not generat $g/2$.\\
	(d)False.\\
	Counterexample:$S_3$\\
	The subgroup of $S_3$ are all cyclic, but $S_3$ is not.\\
	(e) True.\\
	Since an infinite group has infinite number of subgroups, we can know a group with a finite number of subgroup is finite.\\

\end{solution}
%%%%%%%%%%%%%%%

%%%%%%%%%%%%%%%
\begin{problem}[TJ 4-24]
Let $p$ and $q$ be distinct primes. How many generators does $\mathbb{Z}_{pq}$ have?
\end{problem}

\begin{solution}
	We should find out how many $r$ satisfying corollary 4.14.\\
	%$pq-(p-1)-(q-1)-1=pq-p-q+1$\\
	因为$p,q$都是distinct primes, 所以$\varphi(pq) = \varphi(p)\varphi(q)$, 并且$\varphi(p) =p-1,\varphi(q) =q-1$\\
	$\varphi (p) \varphi(q)=(p-1)(q-1)=pq-p-q+1$\\
	(此处算法致谢$https://blog.csdn.net/AgCl_LHY/article/details/107624346$)
\end{solution}
%%%%%%%%%%%%%%%




%%%%%%%%%%%%%%%
\begin{problem}[TJ 4-12]
Find a cyclic group with exactly one generator. Can you find cyclic groups with exactly two generators? Four generators? How about n generators?
\end{problem}

\begin{solution}
	1 generator:$\mathbb{Z}_2$:1\\
	2 generators:$\mathbb{Z}_3$:1,2\\
	4 generators:$\mathbb{Z}_5$:1,2,3,4\\
	n generators:\\
	if(n>2 and n=1(mod 2)), then it is impossible.For that, if $a$ is a generator, then $a^{-1}$ must also be a generator.\\
	If (n = 0(mod 2)):$\mathbb{Z}_m,m=\varphi(n)$
\end{solution}
%%%%%%%%%%%%%%%

%%%%%%%%%%%%%%%
\begin{problem}[TJ 4-32]
Let $G$ be a finite cyclic group of order $n$ generated by $x$. Show that if $y = x^k$ where $gcd(k, n) = 1$, then $y$ must be a generator of $G$.
\end{problem}

\begin{solution}
	According to TH 4.13.\\
	The order of $y$ is $n/gcd(k, n)=n$\\
	So $y$ is a generator of $G$.
\end{solution}
%%%%%%%%%%%%%%%

%%%%%%%%%%%%%%%%%%%%
\beginoptional

%%%%%%%%%%%%%%%
\begin{problem}[$Z_p$]
证明:设$p$为素数,则$Z_p=\{1,2,...,p-1\}$关于$p$\textbf{乘法}构成的$p-1$阶循环群。(此处的$1,2,...,p-1$是模$p$等价类的代表元)
\end{problem}

\begin{solution}
\end{solution}
%%%%%%%%%%%%%%%

%%%%%%%%%%%%%%%
\begin{problem}[SageMath学习]
安装 \href{https://www.sagemath.org/}{SageMath},并学习 TJ 第三章 3.6节、3.7节; 第四章 4.6节、4.7节 关于 SageMath 的内容
\end{problem}

\begin{solution}
\end{solution}
%%%%%%%%%%%%%%%

%%%%%%%%%%%%%%%%%%%%
\beginot
%%%%%%%%%%%%%%%
\fig{ }{figs/mobile-group-demo.png}

在二维平面上的``移动''(例如向东北30度移动9公里)。
你能够以这些``移动''为元素构建一个群吗?
\begin{ot}[``移动''群-1]
	\begin{itemize}
		\item 它的几何元素和运算分别是什么?
		\item 它为什么符合群的定义?
		\item 它是阿贝尔群吗?为什么?
	\end{itemize}
\end{ot}

% \begin{solution}
% \end{solution}
%%%%%%%%%%%%%%%

%%%%%%%%%%%%%%%
\begin{ot}[``移动''群-2]
	\begin{itemize}
		\item 你能找出它的一些子群吗?并说明为什么找到的是子群
		\item 它是循环群吗?如果是,生成元是什么?生成元唯一吗?如果不是,如何改造出一个循环群?
		\item 你能找出这个(改造后的)循环群的一些子群么?它们是循环群么?
	\end{itemize}
\end{ot}


% \begin{solution}
% \end{solution}
%%%%%%%%%%%%%%%


% \vspace{0.50cm}
%%%%%%%%%%%%%%%
% \begin{ot}[]
% 
%   \noindent 参考资料:
%   \begin{itemize}
%     \item 
%   \end{itemize}
% \end{ot}

% \begin{solution}
% \end{solution}
%%%%%%%%%%%%%%%

%%%%%%%%%%%%%%%%%%%%
% 如果没有需要订正的题目,可以把这部分删掉

% \begincorrection
%%%%%%%%%%%%%%%%%%%%

%%%%%%%%%%%%%%%%%%%%
% 如果没有反馈,可以把这部分删掉
\beginfb

% 你可以写
% ~\footnote{优先推荐 \href{problemoverflow.top}{ProblemOverflow}}:
% \begin{itemize}
%   \item 对课程及教师的建议与意见
%   \item 教材中不理解的内容
%   \item 希望深入了解的内容
%   \item $\cdots$
% \end{itemize}
%%%%%%%%%%%%%%%%%%%%
% \bibliography{2-5-solving-recurrence}
% \bibliographystyle{plainnat}
%%%%%%%%%%%%%%%%%%%%
\end{document}