% 2-15-rb-tree.tex

%%%%%%%%%%%%%%%%%%%%
\documentclass[a4paper, justified]{tufte-handout}

\input{hw-preamble} % feel free to modify this file
%%%%%%%%%%%%%%%%%%%%
\title{第4-7讲: 代数编码}
\me{朱宇博}{191220186}{}{}
\date{\zhtoday} % or like 2019年9月13日
%%%%%%%%%%%%%%%%%%%%
\begin{document}
\maketitle
%%%%%%%%%%%%%%%%%%%%
\noplagiarism % always keep this line
%%%%%%%%%%%%%%%%%%%%
\begin{abstract}
  % \begin{center}{\fcolorbox{blue}{yellow!60}{\parbox{0.65\textwidth}{\large 
  %   \begin{itemize}
  %     \item 
  %   \end{itemize}}}}
  % \end{center}
\end{abstract}
%%%%%%%%%%%%%%%%%%%%
\beginrequired

%%%%%%%%%%%%%%%
\begin{problem}[TJ 8-6(b,d)]
best situation?
\end{problem}

\begin{solution}
(b)\\
最小距离为$d_{min}(100011, 110011)=1$\\
best situation: 发出编码(000000),他与其他编码的最小距离最大,为3。故可检测2位错或纠正1位错。\\
(d)\\
最小距离为$d_{min}(0111100, 0110110)=2$\\
best situation: 发出编码(1110000) 或 (1111111) 或 (0001111) 或(0000000),他与其他编码的最小距离最大,为3。故可检测2位错或纠正1位错。\\
\end{solution}
%%%%%%%%%%%%%%%

%%%%%%%%%%%%%%%
\begin{problem}[TJ 8-7(c,d)]
\end{problem}

\begin{solution}
(c)\\
null space: (00100),(00000),(11001),(11101),(11110),(11010),(00111),(00011)\\
type:(5,3)-block\\
generator matrices:\\
 $$\begin{pmatrix}0&1&1\\
			   0&1&1\\
			   1&0&0\\
			   0&1&0\\
			   0&0&1
     \end{pmatrix}$$
      $$\begin{pmatrix}1&0&1\\
			   1&0&1\\
			   0&1&0\\
			   0&0&1\\
			   1&0&0
     \end{pmatrix}$$
因此不唯一\\
(d)\\
null space:
\begin{figure}[htbp]
    \centering
    \includegraphics[width = 0.30\linewidth]{figs/a}
  \end{figure}  
\noindent type: (7,4)-block

generator matrices:\\
 $$\begin{pmatrix}1&0&0&0\\
			   0&1&0&0\\
			   0&0&1&0\\
			   0&0&0&1\\
			   0&1&1&1\\
			   1&0&1&1\\
			   1&1&0&1
     \end{pmatrix}$$
     $$\begin{pmatrix}1&0&0&0\\
			   0&1&0&0\\
			   0&0&0&1\\
			    0&0&1&0\\
			   0&1&1&1\\
			   1&0&1&1\\
			   1&1&0&1
     \end{pmatrix}$$因此不唯一\\
\end{solution}
%%%%%%%%%%%%%%%

%%%%%%%%%%%%%%%
\begin{problem}[TJ 8-9]
\end{problem}

\begin{solution}
$H(01111)^T=(001)^T$, so $(01111)\to (01101)$\\
$H(10101)^T=(110)^T$, so Multiple errors\\
$H(01110)^T=(110)^T$, so Multiple errors\\
$H(00011)^T=(110)^T$, so Multiple errors\\
\end{solution}
%%%%%%%%%%%%%%%

%%%%%%%%%%%%%%%
\begin{problem}[TJ 8-11(b,d)]
\end{problem}

\begin{solution}
(b)\\
This is canonical parity-check matrix.\\
corresponding standard generator matrices:\\
 $$\begin{pmatrix}
   			   1&0\\
			   0&1\\
			   0&1\\
			   1&1\\
			   0&1\\
			   1&1\\
     \end{pmatrix}$$
\noindent 可至少纠错一位、检测2位。\\
(d)\\
This is canonical parity-check matrix.\\
corresponding standard generator matrices:\\
 $$\begin{pmatrix}
 1&0&0\\
			   0&1&0\\
			   0&0&1\\0&0&0\\
			   0&1&1\\
			   1&0&1\\
			   0&1&1\\
     \end{pmatrix}$$   
\noindent 可检测一位错,不可纠错。\\
\end{solution}
%%%%%%%%%%%%%%%

%%%%%%%%%%%%%%%
\begin{problem}[TJ 8-13]
\end{problem}

\begin{solution}
(a)$(001)^T$\\
(b)$(101)^T$\\
(c)$(111)^T$\\
(d)$(011)^T$\\
\end{solution}
%%%%%%%%%%%%%%%

%%%%%%%%%%%%%%%
\begin{problem}[TJ 8-19]
\end{problem}

\begin{solution}
(1)群$C$中权重都为奇数。\\
因为$e\in C\land w(e)=0$,显然不成立。\\
(2)群$C$中权重都为偶数。\\
考虑群$C=\{e\}$,此时显然成立,故存在权重都为偶数的情况。\\
(3)群$C$中权重有奇有偶。\\
考虑$c\in C_{odd}$,构造函数$f:C_{even}\to C_{odd}$ by $x\to x+c$其中$x\in C_{even}$(显然$x+c\in C_{odd}$)。\\
one to one:\\
$\forall x_1, x_2\in C_{even}, x_1+ c = x_2 + c \to x_1 = x_2$(right and left cancellation laws in groups)\\
onto:\\
$\forall y\in C_{odd}, \exists x = y + c^{-1}\in C_{even}, st.x + c = y$\\
So $|C_{even}|=|C_{odd}|$, and then exactly half of them have even weight.\\

\noindent Therefore, either every codeword has even weight or exactly half of the codewords have even weight.
\end{solution}
%%%%%%%%%%%%%%%

%%%%%%%%%%%%%%%
\begin{problem}[TJ 8-21]
\end{problem}

\begin{solution}
(a)error-correcting linear code\\
假设$H$矩阵为$m\times n$的,对$2^7=128$进行编码时,需要满足
$$ \left\{
\begin{aligned}
n-m =  7 \\
n \leq 2^ m - 1 \\
\end{aligned}
\right.
$$
$m=4,n=11$为符合条件的最小正整数解。则最小的generator matrix为$11\times 7$。\\
同理,当对$2^8=256$进行编码时,需要满足
$$ \left\{
\begin{aligned}
n-m =  8 \\
n \leq 2^ m - 1 \\
\end{aligned}
\right.
$$
$m=4,n=12$为符合条件的最小正整数解。则最小的generator matrix为$12\times 8$。\\

\noindent(b)only error detection\\
假设$H$矩阵为$m\times n$的,对$2^7=128$进行编码时,需要满足
$$ \left\{
\begin{aligned}
n-m =  7 \\
n \geq 1 \\
\end{aligned}
\right.
$$
$m=1,n=8$为符合条件的最小正整数解。则最小的generator matrix为$8\times 7$。\\
同理,当对$2^8=256$进行编码时,需要满足
$$ \left\{
\begin{aligned}
n-m =  8 \\
n \geq 1 \\
\end{aligned}
\right.
$$
$m=1,n=9$为符合条件的最小正整数解。则最小的generator matrix为$9\times 8$。\\
\end{solution}
%%%%%%%%%%%%%%%

%%%%%%%%%%%%%%%
\begin{problem}[TJ 8-22]
\end{problem}

\begin{solution}
(1)three information position:\\
 $$H=\begin{pmatrix}1&1&1&1\\
 \end{pmatrix}$$
 $$G=\begin{pmatrix}1&0&0\\
			   0&1&0\\
			   0&0&1\\
			   1&1&1\\
 \end{pmatrix}$$
 
 \noindent (2)seven information position:\\
  $$H=\begin{pmatrix}1&1&1&1&1&1&1&1\\
 \end{pmatrix}$$
 $$G=\begin{pmatrix}1&0&0&0&0&0&0\\
			   0&1&0&0&0&0&0\\
			   0&0&1&0&0&0&0\\
			   0&0&0&1&0&0&0\\
			   0&0&0&0&1&0&0\\
			   0&0&0&0&0&1&0\\
			   0&0&0&0&0&0&1\\
			   1&1&1&1&1&1&1
			   \end{pmatrix}$$		 
\end{solution}
%%%%%%%%%%%%%%%

%%%%%%%%%%%%%%%
\begin{problem}[TJ 8-23]
\end{problem}

\begin{solution}
(a)\\
假设有$n-m$位information bits, $m$位check bits
$$ \left\{
\begin{aligned}
n-m =  20 \\
n \leq 2^ m - 1 \\
\end{aligned}
\right.
$$
解得$m\geq 5$\\
(b)\\
假设有$n-m$位information bits, $m$位check bits
$$ \left\{
\begin{aligned}
n-m =  32 \\
n \leq 2^ m - 1 \\
\end{aligned}
\right.
$$
解得$m\geq 6$
\end{solution}
%%%%%%%%%%%%%%%

%%%%%%%%%%%%%%%%%%%%
\beginoptional

%%%%%%%%%%%%%%%


%%%%%%%%%%%%%%%%%%%%
\beginot
%%%%%%%%%%%%%%%
\begin{ot}[各种花式距离]	
	请查阅资料,介绍曼哈顿距离、欧几里得距离、契比雪夫距离分别是什么意思,他们的典型应用是什么。你还有哪些创意,来定义二进制位串之间的距离?
\end{ot}

% \begin{solution}
% \end{solution}
%%%%%%%%%%%%%%%

%%%%%%%%%%%%%%%
\begin{ot}[编码率]	
	解释什么是编码率,分析hamming码的最大编码率,分析还有比hamming码编码率更好的方法吗?
\end{ot}


% \begin{solution}
% \end{solution}
%%%%%%%%%%%%%%%


% \vspace{0.50cm}
%%%%%%%%%%%%%%%
% \begin{ot}[]
% 
%   \noindent 参考资料:
%   \begin{itemize}
%     \item 
%   \end{itemize}
% \end{ot}

% \begin{solution}
% \end{solution}
%%%%%%%%%%%%%%%

%%%%%%%%%%%%%%%%%%%%
% 如果没有需要订正的题目,可以把这部分删掉

% \begincorrection
%%%%%%%%%%%%%%%%%%%%

%%%%%%%%%%%%%%%%%%%%
% 如果没有反馈,可以把这部分删掉
\beginfb

% 你可以写
% ~\footnote{优先推荐 \href{problemoverflow.top}{ProblemOverflow}}:
% \begin{itemize}
%   \item 对课程及教师的建议与意见
%   \item 教材中不理解的内容
%   \item 希望深入了解的内容
%   \item $\cdots$
% \end{itemize}
%%%%%%%%%%%%%%%%%%%%
% \bibliography{2-5-solving-recurrence}
% \bibliographystyle{plainnat}
%%%%%%%%%%%%%%%%%%%%
\end{document}