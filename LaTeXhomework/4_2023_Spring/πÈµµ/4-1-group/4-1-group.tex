% 2-15-rb-tree.tex

%%%%%%%%%%%%%%%%%%%%
\documentclass[a4paper, justified]{tufte-handout}

\input{hw-preamble} % feel free to modify this file
%%%%%%%%%%%%%%%%%%%%
\title{第4-1讲: 群论初步}
\me{朱宇博}{191220186 }{}{}
\date{\zhtoday} % or like 2019年9月13日
%%%%%%%%%%%%%%%%%%%%
\begin{document}
\maketitle
%%%%%%%%%%%%%%%%%%%%
\noplagiarism % always keep this line
%%%%%%%%%%%%%%%%%%%%
\begin{abstract}
  % \begin{center}{\fcolorbox{blue}{yellow!60}{\parbox{0.65\textwidth}{\large 
  %   \begin{itemize}
  %     \item 
  %   \end{itemize}}}}
  % \end{center}
\end{abstract}
%%%%%%%%%%%%%%%%%%%%
\beginrequired

%%%%%%%%%%%%%%%
\begin{problem}[TJ 3-3]
\end{problem}

\begin{solution}
 \begin{figure}[htbp]
    \centering
    \includegraphics[width = 0.30\linewidth]{figs/a}
  \end{figure}
   \begin{figure}[htbp]
    \centering
    \includegraphics[width = 0.90\linewidth]{figs/b}
  \end{figure}

\noindent $\mathbb{Z}_4$  has $1$ nontrivial proper subset $\{0,2\}$. The number of other group's nontrivial proper is more than one. So they are not same group.  
\end{solution}
%%%%%%%%%%%%%%%

%%%%%%%%%%%%%%%
\begin{problem}[TJ-3-7]
\end{problem}

\begin{solution}
(1)associative law:\\
\[
(a* b) * c = (ab+a+b)*c=a+b+c+ab+ac+bc+ab+abc=a*(bc+b+c)=a*(b*c)
\]
(2)identity element:\\
\[
\forall x\in R\backslash \{-1\},x*0=x
\]
(3)inverse element:\\
\[
\forall x\in R\backslash \{-1\},\exists y = -\frac{a}{a+1}\in R\backslash \{-1\},st. a*b=0
\]
(4)abelian:\\
\[
a*b=a+b+ab=b*a 
\]
\end{solution}
%%%%%%%%%%%%%%%

%%%%%%%%%%%%%%%
\begin{problem}[TJ 3-39]
\end{problem}

\begin{solution}
\[
\forall a=x+yi, b=m+ni\in \mathbb{T}, ab^{-1}=(x+yi)(m-ni)=xm+yn+(my-nx)i\in \mathbb{T}
\]
So $\mathbb{T}$ is a subgroup of $\mathbb{C^*}$
\end{solution}
%%%%%%%%%%%%%%%

%%%%%%%%%%%%%%%
\begin{problem}[TJ 3-42]
\end{problem}

\begin{solution}
$\forall g,h\in H$, Let
\[
g=\begin{pmatrix}
    a_1 & b_1\\\\
    c_1 & d_1 \\\\
 \end{pmatrix}
\qquad h = \begin{pmatrix}
    a_2 & b_2 \\\\
    c_2 & d_2 \\\\
\end{pmatrix}
\]
We have that
\[
h^{-1}=  \begin{pmatrix}
    -a_2 & -b_2 \\\\
    -c_2 & -d_2 \\\\
\end{pmatrix}
\qquad g\circ h^{-1} = \begin{pmatrix}
    a_1-a_2 & b_1-b_2 \\\\
    c_1-c_2 & d_1-d_2 \\\\
\end{pmatrix}
\]
So $g\circ h^{-1}\in H$, $H$ is a subgroup of $G$.
\end{solution}
%%%%%%%%%%%%%%%

%%%%%%%%%%%%%%%
\begin{problem}[TJ 3-49]
\end{problem}

\begin{solution}
\[
a^4b=a^3ab=eab=ab=ba
\]
\end{solution}
%%%%%%%%%%%%%%%

%%%%%%%%%%%%%%%
\begin{problem}[TJ 3-51]
\end{problem}

\begin{solution}
\[
\forall x\in G, xe=x=x^{-1}e^{-1}=x^{-1}
\]
\[
\to \forall x,y\in G, xy\in G, xy=(xy)^{-1}=(y^{-1}x^{-1})=yx\\
\]
So the group $G$ is abelian.
\end{solution}
%%%%%%%%%%%%%%%

%%%%%%%%%%%%%%%
\begin{problem}[TJ 4-1]
\end{problem}

\begin{solution}
(a)\\
False. $49$ is a generator of $\mathbb{Z}_{60}$, but it is not prime.\\
(b)\\
False. $1,3,5,7$ are all not generator of $U(8)$.\\
(c)\\
False. Assume that g is a generator of $\mathbb{Q}$, but $g$ can not generate $\frac{g}{2}$.\\
(d)\\
False. The symmetry group of an equilateral triangle $S3$ is not cyclic, but the subgroup of $S3$ are all cyclic.\\
(e)\\
True. \\
Suppose, to the contrary, an infinite group $G$ has finite number of subgroup.\\
(1) If $G$ has an infinite order  generator $g$, then $<g>, <g^2>, <g^3>, <g^4>,...,<g^k>$ are all the subgroup of $G$, so it has infinite number of subgroup. It is contradict with the assumption.\\
(2) If the order of generators are all finite, then we let S = $\{<x>|x\in G\}$. We have that $S$ is a finite set. The group $G$ is the union of the finite set $S$, so $G$ is finite, it is contradict with the assumption.\\
Therefore, a group with a finite number of subgroups is finite.
\end{solution}
%%%%%%%%%%%%%%%

%%%%%%%%%%%%%%%
\begin{problem}[TJ 4-24]
\end{problem}

\begin{solution}
$\phi(pq)=\phi(p)\phi(q)=(p-1)(q-1)=pq-p-q+1$(p and q are different primes)
\end{solution}
%%%%%%%%%%%%%%%




%%%%%%%%%%%%%%%
\begin{problem}[TJ 4-12]
\end{problem}

\begin{solution}
one generator: $\mathbb{Z}_2$\\
Two generators: $\mathbb{Z}_4$\\
Four generators: $\mathbb{Z}_8$\\
$n$ generators: $\exists x, st. \phi(x)=n$. $\mathbb{Z}_{x}$ has $n$ generators.
\end{solution}
%%%%%%%%%%%%%%%

%%%%%%%%%%%%%%%
\begin{problem}[TJ 4-32]
\end{problem}

\begin{solution}
Due to Theorem4.13 in TJ, $y=x^{k}$, the order of $y$ is $\frac{n}{gcd(k,n)}=\frac{n}{1}=n$. Therefore, $y$ is a generator of $G$.
\end{solution}
%%%%%%%%%%%%%%%

%%%%%%%%%%%%%%%%%%%%
\beginoptional

%%%%%%%%%%%%%%%
\begin{problem}[$Z_p$]
证明:设$p$为素数,则$Z_p=\{1,2,...,p-1\}$关于$p$\textbf{乘法}构成的$p-1$阶循环群。(此处的$1,2,...,p-1$是模$p$等价类的代表元)
\end{problem}

\begin{solution}
\end{solution}
%%%%%%%%%%%%%%%

%%%%%%%%%%%%%%%
\begin{problem}[SageMath学习]
安装 \href{https://www.sagemath.org/}{SageMath},并学习 TJ 第三章 3.6节、3.7节; 第四章 4.6节、4.7节 关于 SageMath 的内容
\end{problem}

\begin{solution}
\end{solution}
%%%%%%%%%%%%%%%

%%%%%%%%%%%%%%%%%%%%
\beginot
%%%%%%%%%%%%%%%
\fig{ }{figs/mobile-group-demo.png}

在二维平面上的``移动''(例如向东北30度移动9公里)。
	你能够以这些``移动''为元素构建一个群吗?
\begin{ot}[``移动''群-1]	
	\begin{itemize}
	\item 它的几何元素和运算分别是什么?
	\item 它为什么符合群的定义?
	\item 它是阿贝尔群吗?为什么?
	\end{itemize}
\end{ot}

% \begin{solution}
% \end{solution}
%%%%%%%%%%%%%%%

%%%%%%%%%%%%%%%
\begin{ot}[``移动''群-2]	
	\begin{itemize}
	\item 你能找出它的一些子群吗?并说明为什么找到的是子群
	\item 它是循环群吗?如果是,生成元是什么?生成元唯一吗?如果不是,如何改造出一个循环群?
	\item 你能找出这个(改造后的)循环群的一些子群么?它们是循环群么?
	\end{itemize}
\end{ot}


% \begin{solution}
% \end{solution}
%%%%%%%%%%%%%%%


% \vspace{0.50cm}
%%%%%%%%%%%%%%%
% \begin{ot}[]
% 
%   \noindent 参考资料:
%   \begin{itemize}
%     \item 
%   \end{itemize}
% \end{ot}

% \begin{solution}
% \end{solution}
%%%%%%%%%%%%%%%

%%%%%%%%%%%%%%%%%%%%
% 如果没有需要订正的题目,可以把这部分删掉

% \begincorrection
%%%%%%%%%%%%%%%%%%%%

%%%%%%%%%%%%%%%%%%%%
% 如果没有反馈,可以把这部分删掉
\beginfb

% 你可以写
% ~\footnote{优先推荐 \href{problemoverflow.top}{ProblemOverflow}}:
% \begin{itemize}
%   \item 对课程及教师的建议与意见
%   \item 教材中不理解的内容
%   \item 希望深入了解的内容
%   \item $\cdots$
% \end{itemize}
%%%%%%%%%%%%%%%%%%%%
% \bibliography{2-5-solving-recurrence}
% \bibliographystyle{plainnat}
%%%%%%%%%%%%%%%%%%%%
\end{document}