% 2-15-rb-tree.tex

%%%%%%%%%%%%%%%%%%%%
\documentclass[a4paper, justified]{tufte-handout}

\input{hw-preamble} % feel free to modify this file
%%%%%%%%%%%%%%%%%%%%
\title{第4-12讲: 启发式算法}
\me{朱宇博}{191220186}{}{}
\date{\zhtoday} % or like 2019年9月13日
%%%%%%%%%%%%%%%%%%%%
\begin{document}
\maketitle
%%%%%%%%%%%%%%%%%%%%
\noplagiarism % always keep this line
%%%%%%%%%%%%%%%%%%%%
\begin{abstract}
  % \begin{center}{\fcolorbox{blue}{yellow!60}{\parbox{0.65\textwidth}{\large 
  %   \begin{itemize}
  %     \item 
  %   \end{itemize}}}}
  % \end{center}
\end{abstract}
%%%%%%%%%%%%%%%%%%%%
\beginrequired

%%%%%%%%%%%%%%%
\begin{problem}[探索题]
选择一道典型"难题", 给出一种启发式算法(不限于模拟退火与遗传算法;可百度,可Bing,可Google)并作简要分析
\end{problem}

\begin{solution}
可选择用模拟退火算法求解TSP问题。\\
该问题的解可表示为$(a_1, a_2, ..., a_n)$,其中序列$a$为一个长度为$n$的排列,代表遍历城市的顺序。目标为使得路程之和最小化\\
在开始时设置初始温度$T$,终止条件,并随机一个排列做为初始解\\
之后的每一轮,在以下两种策略中,随机选择一种:
1,  随机选择不同的下标$i,j$,交换$a_i$, $a_j$\\
2,随机选择不同的下标$i,j,k(i < j <k)$,将$a_i$到$a_j$之间的路径,插到$a_k$后面。\\
在此之后若新解优于原解,则接受;否则,则以$e^{\frac{\delta}{T}}$的概率接受。此后降低温度。\\
若降低温度后达到结束温度,则结束并输出答案,否则,继续迭代。
\end{solution}
%%%%%%%%%%%%%%%



%%%%%%%%%%%%%%%
%%%%%%%%%%%%%%%%%%%%
\beginoptional
%%%%%%%%%%%%%%%


%%%%%%%%%%%%%%%%%%%%
\beginot
%%%%%%%%%%%%%%%
\begin{ot}[超级玛丽是 NP-hard 的.]
	请证明超级玛丽是 NP-hard的。	
	\begin{itemize}
	\item 参考资料:\href{http://cslabcms.nju.edu.cn/problem_solving/images/e/e0/Classic_Nintendo_Games_are_Computationally_Hard_\%28arXiv12_1203.1895\%29.pdf}{Classic Nintendo Games are Computationally Hard}
	\end{itemize}
\end{ot}

% \begin{solution}
% \end{solution}
%%%%%%%%%%%%%%%


\begin{ot}[经典随机算法介绍.]
	介绍几个经典的(没有在课堂上讲解过的)随机算法。
	\begin{itemize}
	\item 参考资料:\href{https://immorlica.com/randAlg/Karp91.pdf}{An introduction to randomized algorithms}
	\end{itemize}
\end{ot}

% \begin{solution}
% \end{solution}
%%%%%%%%%%%%%%%

% \vspace{0.50cm}
%%%%%%%%%%%%%%%
% \begin{ot}[]
% 
%   \noindent 参考资料:
%   \begin{itemize}
%     \item 
%   \end{itemize}
% \end{ot}

% \begin{solution}
% \end{solution}
%%%%%%%%%%%%%%%

%%%%%%%%%%%%%%%%%%%%
% 如果没有需要订正的题目,可以把这部分删掉

% \begincorrection
%%%%%%%%%%%%%%%%%%%%

%%%%%%%%%%%%%%%%%%%%
% 如果没有反馈,可以把这部分删掉
\beginfb

% 你可以写
% ~\footnote{优先推荐 \href{problemoverflow.top}{ProblemOverflow}}:
% \begin{itemize}
%   \item 对课程及教师的建议与意见
%   \item 教材中不理解的内容
%   \item 希望深入了解的内容
%   \item $\cdots$
% \end{itemize}
%%%%%%%%%%%%%%%%%%%%
% \bibliography{2-5-solving-recurrence}
% \bibliographystyle{plainnat}
%%%%%%%%%%%%%%%%%%%%
\end{document}