\documentclass{ctexart}

\title{如何使用 \LaTeX 描述算法}
\author{李晗}
\date{November 2021}

\usepackage{amsmath, amsthm, amssymb, mathrsfs, graphicx}
\usepackage{tcolorbox, hyperref, enumerate}
\usepackage{tabu}
\usepackage{xcolor}
\usepackage{listings}
\usepackage{algorithm}
\usepackage{algpseudocode}

\begin{document}

\maketitle

\section{文本环境}

如果需要让 \LaTeX 在单行文本中 \textbf{原样输出} ,可以使用 \verb|\verb| 环境。
具体使用方式见源代码。

使用 \verb|\verbatim| 或 \verb|\lstlisting| 等环境都可以让环境内的多行文本 \textbf{原样输出},包括空格、换行等。

\noindent verbatim:
\begin{verbatim}
#include <stdio.h> 
int main() {
    printf("Hello world!");
}
\end{verbatim}

\noindent lstlisting:
\begin{lstlisting}
#include <stdio.h> 
int main() {
    printf("Hello world!");
}
\end{lstlisting}

也许你会觉得 \verb|\lstlisting| 环境看上去比较丑,但是 1-6 作业中的代码就是用它完成的。经过配置之后看上去能舒服一些。

\lstdefinestyle{style1}{
    basicstyle=\ttfamily,
    breaklines=true,
    numbers=left,
    keywordstyle=\color{purple}\bfseries,
    identifierstyle=\color{brown!80!black},
    commentstyle=\color{gray},
    showstringspaces=false,
    frame=trBL,
    frameround=fftt,
    backgroundcolor=\color[RGB]{245,245,244},
}
\begin{lstlisting}[language=C++,style=style1]
#include <stdio.h> 
int main() {
    printf("Hello world!");
}
\end{lstlisting}

\section{algorithm环境}

使用 \verb|algorithm| 环境能够方便的排版非常直观的伪代码代码。

\begin{algorithm}[H]
  \caption{The Name of Algorithm}
  \label{alg:sum}
  \begin{algorithmic}[1]
    \Procedure{Fib}{$n$} \Comment{通常伪代码的函数名是这样特殊的大写}
      \If {$n \le 1$}
        \State \Return $1$
      \ElsIf {$n = 2$}   \Comment{仅为说明如何使用 else if}
        \State \Return $2$
      \Else
        \State \Return $n \times \Call{Fib}{n}$ \Comment{使用 Call 符合伪代码函数名}
      \EndIf
    \EndProcedure
    \Procedure{Another}{}
      \For {$ i \gets 1, n $}
        \State $ sum \gets sum + i $
      \EndFor
      \While {$ sum > 0 $}  \label{The While Loop}
        \State $ sum-- $
      \EndWhile
      \Repeat
        \State 编不下去了
        \State while 循环在第\ref{The While Loop}行
      \Until {$cond$}
    \EndProcedure
  \end{algorithmic}
\end{algorithm}
\end{document}
