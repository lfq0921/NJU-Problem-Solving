% 1-6-programming-language.tex

%%%%%%%%%%%%%%%%%%%%
\documentclass[a4paper, justified]{tufte-handout}

\input{hw-preamble} % feel free to modify this file
\usepackage{listings}
\lstset{
    keywordstyle=\color{purple}\bfseries,
    identifierstyle=\color{brown!80!black},
    commentstyle=\color{gray},
    showstringspaces=false,
    frame=trBL,
    frameround=fftt,
    backgroundcolor=\color[RGB]{245,245,244},
    breaklines=true,
    numbers=left,
}
%%%%%%%%%%%%%%%%%%%%
\title{第6讲: 程序设计语言的语法与语义}
\me{林凡琪}{211240042}{}{}
\date{\zhtoday} % or like 2019年9月13日
%%%%%%%%%%%%%%%%%%%%
\begin{document}
\maketitle
%%%%%%%%%%%%%%%%%%%%
\noplagiarism % always keep this line
%%%%%%%%%%%%%%%%%%%%
\begin{abstract}
  \mfig{width = 0.85\textwidth}{figs/syntax-semantics}
  \begin{center}{\fcolorbox{blue}{yellow!60}{\parbox{0.70\textwidth}{\large
          \begin{itemize}
            \item 逻辑系统有语法、语义之分; 程序设计语言亦如是
            \item 欢迎进入\href{https://en.wikipedia.org/wiki/Programming\_language\_theory}{程序设计语言理论}的广阔世界
          \end{itemize}}}}
  \end{center}
\end{abstract}
%%%%%%%%%%%%%%%%%%%%
\beginrequired

%%%%%%%%%%%%%%%
\begin{problem}[IMP]
考虑程序设计语言 {\bf IMP} (Imperative), 它包含如下元素:
\begin{itemize}
  \item 整数集合 $\mathbb{Z}$ (可用 $m,n$ 表示其中的元素)
  \item 真值集合 $\mathbb{T} = \set{T, F}$
  \item 变量集合 $\mathbb{V}$ (可用 $v$ 表示其中的元素)
  \item 算法表达式 {\bf Aexp}, 支持 ``$+, -, \times$'' 三则运算
  \item 布尔表达式 {\bf Bexp}, 支持 ``$=, \le$'' 比较操作与基本的逻辑操作
  \item 语句 {\bf St}, 包括赋值语句 ($:=$)、顺序语句 ($;$)、选择语句 (if-then-else)与循环语句 (while-do)
\end{itemize}

\noindent 请为 {\bf IMP} 设计语法,并使用 BNF 描述。\\
\noindent 例如:
\item  <字母>::=<大写字母><小写字母>\\
\item <大写字母>::=A|B|C|D|E|F|G|H|I|J|K|L|M|N|O|P|Q|R|S|T|U|V|W|X|Y|Z\\
\item <小写字母>::=a|b|c|d|e|f|g|h|i|j|k|l|m|n|o|p|q|r|s|t|u|v|w|x|y|z\\
\end{problem}
\begin{solution}

  \noindent <整数集合$\mathbb{Z}$>::=("1"|"2"|"3"|"4"|"5"|"6"|"7"|"8"|"9")[$\left\{"1"|"2"|"3"|"4"|"5"|"6"|"7"|"8"|"9"|"0"\right\}$] \\

  \noindent <真值集合$\mathbb{T}$>::=T|F\\

  \noindent<变量集合 $\mathbb{V}$>::= <真值集合$\mathbb{T}$>|<整数集合$\mathbb{Z}$>\\

  \noindent<\bf Aexp>::=<整数集合$\mathbb{Z}$> | <变量集合 $\mathbb{V}$> | <Aexp> "+" <Aexp>|<Aexp> "-" <Aexp>|<Aexp> "$\times$" <Aexp>\\

  \noindent <\bf Bexp>::=<Aexp>"$=$"<Aexp>|<Aexp>"$\le$"<Aexp>|<Bexp> "and" <Bexp>|<Bexp> "or" <Bexp>|"not"<Bexp>\\

  \noindent <语句\bf St>::=<赋值语句>|<顺序语句>|<选择语句>|<循环语句>\\

  \noindent <赋值语句>::=<变量集合 $\mathbb{V}$>":=" <整数集合$\mathbb{Z}$>|<变量集合 $\mathbb{V}$>":="<真值集合$\mathbb{T}$>\\

  \noindent <顺序语句>::=<语句\bf St>";"\\

  \noindent <选择语句>::=if <真值集合$\mathbb{T}$> then <语句\bf St> else <语句\bf St>\\

  \noindent <循环语句>::=while <真值集合$\mathbb{T}$> do <语句\bf St>\\

\end{solution}


\begin{problem}[Prolog 逻辑推理]
学习 Prolog 相关知识
\footnote{推荐阅读 \href{https://zhzluke96.github.io/prolog-tut-cn/tut/chapter1.html}{教程} 3.1 至 3.9 章的内容。}
,并且完成下列程序
\footnote{你可以使用 \href{https://swish.swi-prolog.org}{在线运行环境} 完成程序。}
。

这个程序定义了一个 parent 关系和两个谓词 isAncestor 和 isCollaterialRelative。
其中,两个谓词的定义并不完整,因此也不能通过下方附带的测试样例。
请修改 isAncestor 和 isCollaterialRelative 的定义,使其符合注释要求的功能。

\lstinputlisting[language=Prolog]{prolog1.pl}
\end{problem}

\begin{solution}
  \footnote{作业压缩包中含有上述程序的源代码 prolog1-sol.pl,你可以取消下面一行的注释并将修改后的程序放入 prolog1-sol.pl 中。}

  \lstinputlisting[language=Prolog]{prolog1-sol.pl}
\end{solution}
%%%%%%%%%%%%%%%

%%%%%%%%%%%%%%%
\begin{problem}[Prolog 逻辑推理]
下面这个 Prolog 程序试图解决一个经典的逻辑推理题。
请修改谓词 xxxGuess 的定义,使其能够对 X, Pos 变量返回正确的结果。

\begin{quote}
  赵钱孙李周吴郑王八位将军外出打猎,有一位将军的箭射中了一只鹿。
  在拔出箭头前,众人饶有兴致的猜起来是谁射中了鹿。
  \begin{enumerate}
    \item 赵说:“或者是王将军射中的,或者是吴将军射中的。”
    \item 钱说:“如果这只箭正好射中了鹿的头部,那么肯定是我射中的。”
    \item 孙说:“我确定是郑将军射中的。”
    \item 李说:“即使箭正好在鹿的头上,也不可能是钱将军射中的。”
    \item 周说:“赵将军猜错了。”
    \item 吴说:“不是我射中的,也不是王将军射中的。”
    \item 郑说:“不是孙将军射中的。”
    \item 王说:“赵将军没有猜错。”
  \end{enumerate}
  最后,大家把鹿身上的箭拔出来查看,八位将军中正好有三个人猜对了。
\end{quote}

\lstinputlisting[language=Prolog]{prolog2.pl}
\end{problem}

\begin{solution}
  程序计算出的结果为:\\
  X将军射中了鹿头/鹿身,且将军A、B、C猜对了。
  \footnote{作业压缩包中含有上述程序的源代码 prolog2-sol.pl,你可以取消下面一行的注释并将修改后的程序放入 prolog2-sol.pl 中。}

  修改后的程序:\\
  \lstinputlisting[language=Prolog]{prolog2-sol.pl}
\end{solution}


\begin{problem}[Prolog 斐波那契]
这个程序定义了两个谓词fib和find\_fib。
其中,两个谓词的定义并不完整,因此也不能通过下方附带的测试样例。
请修改fib和 find\_fib 的定义,使其符合注释要求的功能。


\lstinputlisting[language=Prolog]{prolog3.pl}
\end{problem}

\begin{solution}
  \footnote{作业压缩包中含有上述程序的源代码 prolog3-sol.pl,你可以取消下面一行的注释并将修改后的程序放入 prolog3-sol.pl 中。}

  \lstinputlisting[language=Prolog]{prolog3-sol.pl}
\end{solution}
%%%%%%%%%%%%%%%




%%%%%%%%%%%%%%%

%%%%%%%%%%%%%%%%%%%%
\beginot

%%%%%%%%%%%%%%%
\begin{ot}[正则表达式]
  请介绍正则表达式 (Regualr Expression) 的语法、语义与用例等。

  \noindent 基本要求:
  \begin{itemize}
    \item 循序渐进
    \item 使用有趣而实用的例子
  \end{itemize}

  \noindent 参考资料:
  \begin{itemize}
    \item \href{https://en.wikipedia.org/wiki/Regular\_expression}{Regular expression @ wiki}
    \item \href{https://regex101.com/}{regex101}
  \end{itemize}
\end{ot}

% \begin{solution}
% \end{solution}
%%%%%%%%%%%%%%%
\vspace{0.50cm}
%%%%%%%%%%%%%%%
\begin{ot}[程序设计语言的语义]
  阅读并介绍经典论文
  \href{https://github.com/hengxin/problem-solving-class-paperswelove/tree/master/1st-semester}{``CACM1969 (Hoare) An Axiomatic Basis for Computer Progamming'' @ problem-solving-class-paperswelove}:

  \mfig{width = 0.75\textwidth}{figs/hoare}
  \begin{itemize}
    \item 作者简介 \href{https://en.wikipedia.org/wiki/Tony\_Hoare}{Tony Hoare @ wiki}
    \item 概览文章的结构与贡献
    \item 重点介绍第三节 ``Program Execution'' 的内容
    \item 介绍 Table III 中的证明示例
  \end{itemize}
\end{ot}

% \begin{solution}
% \end{solution}
%%%%%%%%%%%%%%%

%%%%%%%%%%%%%%%%%%%%
\begincorrection
%%%%%%%%%%%%%%%%%%%%

%%%%%%%%%%%%%%%%%%%%
\beginfb

你可以写
~\footnote{优先推荐 \href{problemoverflow.top}{ProblemOverflow}}:
\begin{itemize}
  \item 对课程及教师的建议与意见
  \item 教材中不理解的内容
  \item 希望深入了解的内容
  \item $\cdots$
\end{itemize}
%%%%%%%%%%%%%%%%%%%%
\end{document}