% 1-9-relation.tex

%%%%%%%%%%%%%%%%%%%%
\documentclass[a4paper, justified]{tufte-handout}

\input{hw-preamble} % feel free to modify this file
%%%%%%%%%%%%%%%%%%%%
\title{第9讲: 关系及其性质}
\me{林凡琪}{211240042}{}{}
\date{\zhtoday} % or like 2019年9月13日
%%%%%%%%%%%%%%%%%%%%
\begin{document}
\maketitle
%%%%%%%%%%%%%%%%%%%%
\noplagiarism % always keep this line
%%%%%%%%%%%%%%%%%%%%
\begin{abstract}
  \begin{center}{\fcolorbox{blue}{yellow!60}{\parbox{0.26\textwidth}{\large
          \begin{itemize}
            \item ``关系''至关重要
          \end{itemize}}}}
  \end{center}
\end{abstract}
%%%%%%%%%%%%%%%%%%%%
\beginrequired

%%%%%%%%%%%%%%%
\begin{problem}[UD Problem 10.9]
\end{problem}

\begin{solution}
  This relation is an equivalence relation.\\
  reflexive:$\forall (x_1, x_2), x_1 - x_1 = 0;x_2 - x_2 = 0;$and 0 is even;therefore $(x_1, x_2)\sim(x_1, x_2)$\\
  symmetry:$\forall (x_1, x_2),(y_1, y_2)$ if $x_1 - y_1 = 2m; x_2 - y_2 = 2n$ then $(x_1, x_2)\sim(y_1, y_2)$;therefore $ y_1 - x_1 = -2m; y_2 - x_2 = -2n$, therefore $(y_1, y_2)\sim(x_1, x_2)$.\\
  transitive:if$(x_1, x_2)\sim(y_1, y_2)$ and $(z_1, z_2)\sim(y_1, y_2)$, then $x_1 - y_1 = 2m; x_2 - y_2 = 2n$ and $z_1 - y_1 = 2p; z_2 - y_2 = 2q$; therefore $x_1 - z_1 = (x_1 - y_1) - (z_1 - y_1) = 2(m - p), x_2 - z_2 = (x_2 - y_2) - (z_2 - y_2) = 2(n - q)$; therefore $(x_1, x_2)\sim(z_1, z_2)$
\end{solution}
%%%%%%%%%%%%%%%

%%%%%%%%%%%%%%%
\begin{problem}[UD Problem 10.10]
\end{problem}

\begin{solution}
  如果$E_x = E_y$ 则 $\forall x \in E_x$,满足$x \in E_y$ , 且$\forall x \in E_y$,满足$x \in E_x$.\\
  不妨设$z \in E_x$且$z \in E_y$,则$z\sim x$且$z\sim y$\\
  根据传递性,$x\sim y$.\\
  反推亦成立.
\end{solution}
%%%%%%%%%%%%%%%

%%%%%%%%%%%%%%%
\begin{problem}[UD Problem 10.13]
\end{problem}

\begin{proof}
  (a) 是一个等价关系.\\
  自反性:p(0) = p(0) = a0;\\
  对称性:因为p(0) = q(0) 所以 p$\sim $q;又因为q(0) = p(0),所以q$\sim$p\\
  传递性:因为p(0) = q(0),所以a0 = b0;\\
  因为q(0) = r(0), 所以b0 = c0;\\
  所以a0 = c0,所以q(0) = r(0)\\
  E = {任何常数项为0的多项式}\\
  (b)是一个等价关系;\\
  $E_r =\{ a_2 * x^2 + a_1 * x + a_0 | a_2 \neq 0\}$\\
  (c)不是一个等价关系.不满足对称性.
\end{proof}
%%%%%%%%%%%%%%%

%%%%%%%%%%%%%%%
\begin{problem}[UD Problem 11.4]
\end{problem}

\begin{solution}
  (a)是partition.是法向量为(1,1,1)的无数个平行平面的集合.\\
  (b)是partition.是无数个球心在原点半径为r的球壳的集合.
\end{solution}
%%%%%%%%%%%%%%%

%%%%%%%%%%%%%%%
\begin{problem}[UD Problem 11.8]
\end{problem}

\begin{proof}
  (a)对于条件一:每一个级数的多项式都存在,所以A必不是空集.\\
  对于条件二:$\cup_A = X$所有级数的多项式的集合就是所有多项式的集合\\
  对于条件三:不妨设A集合中的多项式级数为a,B集合中的多项式级数为b,若级数相同,则a = b,$\forall x\in A, x \in B$即$A = B$;\\
  若级数不同,$a \neq b$,则$\forall x\in A, x \notin B$即$A \cap B = \emptyset$\\
  (b)条件一:c一定存在,所以$A_c$必不是空集\\
  条件二:$\cup A_c = {A_c |c \in R}$又因为c是实数,所以$\cup A_c$是所有多项式的集合.\\
  条件三:不妨设$A_c$中P(0) = c, $A_d$中P(0) = d,若c = d 则$\forall P(P(0) = c = d) \in A_c$并且$\forall P(P(0) = c = d) \in A_d$,所以$A_c = A_d$\\
  (c)不是partition.违反了条件三.\\
  p = qr,若q = m * n;则$p \in A_q$且$p \in A_m$且$p \in A_n$.即$A_m \neq A_n$且$A_m \cap A_n \neq \emptyset$.\\
  (d)不是partition.违反了条件三\\
  反例如下:令$p = x^2 -3x + 2$;$p(1) = p(2) = 0$;则$p \in A_1$且$p \in A_2$.即$A_1 \neq A_2$且$A_1 \cap A_2 \neq \emptyset$.
\end{proof}
%%%%%%%%%%%%%%%

%%%%%%%%%%%%%%%
\begin{problem}[UD Problem 11.10]
\end{problem}

\begin{proof}
  (a)是partition.\\
  条件一显然符合.\\
  条件二:$\cup A_\alpha = X$,则$\cup (A_alpha \cap B) = \cup A_\alpha \cap B = X \cap B = B$\\
  因为$\{A_\alpha\}$是X的partition,所以$\{A_\alpha\}$必满足条件三,又因为$\{A_\alpha \cap B\} \subseteq \{A_\alpha\}$,所以${A_\alpha \cap B}$一定也满足条件三.\\
  (b)当$A_\alpha = \emptyset$则${X \backslash A_alpha}$是X的partition.\\
  否则$X\backslash A_a \cap X\backslash A_b = X\backslash (A_a \cap A_b)$即不满足条件三.
\end{proof}
%%%%%%%%%%%%%%%

%%%%%%%%%%%%%%%
\begin{problem}[UD Problem 12.11 (a, b)]
\end{problem}

\begin{solution}
  (a)假设$supS > sup(S\cup T)$,因为supS是上确界,所以任意y<supS,都会$\exists x \in S$ 使得$x < y$所以若$supS > sup(S\cup T)$,则$exists x \in (S \cup T), x >  sup(S\cup T)$,此时$ sup(S\cup T)$不符合定义.所以假设不成立.\\
  同理可证$supT <= sup(S\cup T)$\\
  (b)若$sup(S \cup T) \neq supS$则$\exists x\in T,x >supS,但x<=supT$,所以$sup(S \cup T) = sup T$\\
  同理可证若 $sup(S \cup T) \neq supT$,则$sup(S \cup T) = sup S$\\
  得证.
\end{solution}
%%%%%%%%%%%%%%%

%%%%%%%%%%%%%%%
\begin{problem}[UD Problem 12.12]
\end{problem}

\begin{solution}
  (a)$\exists M, (M \in \mathbb{R} \land \forall y, (y \in S \rightarrow y\leqq M)) \rightarrow \exists N, (N = M + x \land \forall y, (y \in x + S \rightarrow y <= N))$\\
  (b)$\forall y,(y \in S, \rightarrow y\leqq sup S)\rightarrow \forall y, (y \in x + sup S)$\\
  (c)$\forall y,(y \in x + S \rightarrow y <= x + sup S)\\
    \forall M,(M \in \mathbb{R} \land\forall y, (y \in S \rightarrow y\leqq M) \rightarrow M >= supS)\\
    \rightarrow \forall N, (N \in \mathbb{R} \land \forall y, (y \in x + sup S \rightarrow y <= N) \rightarrow N>= x + sup S)$\\
  (1)$\land$(2) $\rightarrow x + sup S = sup(x + S).$
\end{solution}
%%%%%%%%%%%%%%%

%%%%%%%%%%%%%%%
\begin{problem}[UD Problem 13.14]
\end{problem}

\begin{solution}
  1. $\forall x \in S,x \subseteq x$\\
  2. $\forall x, y \in S,,y \subseteq x,\rightarrow x = y$\\
  3. $\forall x,y,z\in S,x \subseteq y, y \subseteq z, \rightarrow x \subseteq z$\\
  4.若$A = \{a|a\in P(A)\},B = \{b,c|b, c\in P(A)\},\rightarrow A\nsubseteq B,B\nsubseteq A$\\
  前三条满足偏序,第四条不满足全序.\\
  得证.
\end{solution}
%%%%%%%%%%%%%%%

%%%%%%%%%%%%%%%%%%%%
\beginoptional

%%%%%%%%%%%%%%%
\begin{problem}[关系的复合]
\mfigcap{width = 1.00\textwidth}{figs/wulin-relation}{``舅老爷''是什么关系复合而成的?}
定义二元关系 $R$ 与 $S$ 的复合为:
\[
  S \circ R = \set{(x, z) \mid \exists y \big((x, y) \in R \land (y, z) \in S \big)}.
\]

\noindent 请证明复合操作满足结合律:
\[
  T \circ (S \circ R) = (T \circ S) \circ R.
\]
\end{problem}

\begin{proof}
\end{proof}
%%%%%%%%%%%%%%%

%%%%%%%%%%%%%%%%%%%%
\beginot

%%%%%%%%%%%%%%%
\begin{ot}[二元关系]
  介绍花样繁多的``二元关系'', 如 (不限于):
  \begin{itemize}
    \item Preorder
    \item Strict weak order
    \item Strict partial order
    \item $\cdots$
  \end{itemize}

  \noindent 基本要求:
  \begin{itemize}
    \item 举例说明每种二元关系的应用
  \end{itemize}

  \noindent 参考资料:
  \begin{itemize}
    \item \href{https://en.wikipedia.org/wiki/Binary\_relation}{Binary relation @ wiki}
  \end{itemize}
\end{ot}
%%%%%%%%%%%%%%%
\vspace{0.50cm}
%%%%%%%%%%%%%%%
\begin{ot}[实数]
  介绍实数的完备性 (Completeness), 如 (不限于):
  \begin{itemize}
    \item 概念
    \item 等价形式
    \item 实数的构造方式
  \end{itemize}

  \noindent 参考资料:
  \begin{itemize}
    \item \href{https://en.wikipedia.org/wiki/Completeness\_of\_the\_real\_numbers}{Completeness of the real numbers @ wiki}
  \end{itemize}
\end{ot}
%%%%%%%%%%%%%%%

%%%%%%%%%%%%%%%%%%%%
% 如果没有需要订正的题目,可以把这部分删掉
\begincorrection

%%%%%%%%%%%%%%%%%%%%

%%%%%%%%%%%%%%%%%%%%
% 如果没有反馈,可以把这部分删掉
\beginfb

% 你可以写
% ~\footnote{优先推荐 \href{problemoverflow.top}{ProblemOverflow}}:
% \begin{itemize}
%   \item 对课程及教师的建议与意见
%   \item 教材中不理解的内容
%   \item 希望深入了解的内容
%   \item $\cdots$
% \end{itemize}
%%%%%%%%%%%%%%%%%%%%
\end{document}