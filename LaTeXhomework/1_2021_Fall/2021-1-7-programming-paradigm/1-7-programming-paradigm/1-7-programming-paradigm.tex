% 1-6-programming-language.tex

%%%%%%%%%%%%%%%%%%%%
\documentclass[a4paper, justified]{tufte-handout}

\input{hw-preamble} % feel free to modify this file
\usepackage{listings}
\lstset{
    keywordstyle=\color{purple}\bfseries,
    identifierstyle=\color{brown!80!black},
    commentstyle=\color{gray},
    showstringspaces=false,
    frame=trBL,
    frameround=fftt,
    backgroundcolor=\color[RGB]{245,245,244},
    breaklines=true,
    numbers=left,
}
%%%%%%%%%%%%%%%%%%%%
\title{第7讲: 程序设计范型}
\me{林凡琪}{211240042}{}{}
\date{\zhtoday} % or like 2019年9月13日
%%%%%%%%%%%%%%%%%%%%
\begin{document}
\maketitle
%%%%%%%%%%%%%%%%%%%%
\noplagiarism % always keep this line
%%%%%%%%%%%%%%%%%%%%
\begin{abstract}
  \mfig{width = 0.80\textwidth}{figs/haskell-side-effects}
  \begin{center}{\fcolorbox{blue}{yellow!60}{\parbox{0.45\textwidth}{\large
          \begin{itemize}
            \item 函数式程序设计, 你值得拥有
          \end{itemize}}}}
  \end{center}
\end{abstract}
%%%%%%%%%%%%%%%%%%%%
\beginrequired

%%%%%%%%%%%%%%%
\begin{problem}[Haskell 与 函数式]
学习 Haskell 语言和函数式程序设计
\footnote{推荐学习 \href{https://learnyouahaskell.mno2.org/zh-cn}{在线教程} 的第2章和第5章。}
,完成下列题目
\footnote{推荐使用 \href{https://replit.com/languages/haskell}{在线平台} 运行程序。(免注册,但请记得保存!)}
。
\end{problem}
\begin{solution}
  \footnote{作业压缩包中含有上述程序的源代码 haskell1-sol.hs,你可以将修改后的程序放入该文件中并与作业一起上传。}

  \lstinputlisting[language=Haskell]{haskell1-sol.hs}
\end{solution}
%%%%%%%%%%%%%%%

\beginoptional
%%%%%%%%%%%%%%%
\begin{problem}[24 点]
下面的程序试图使用 Haskell 解决 24 点问题。
得益于 Haskell 和 函数式 的强大表达能力,多数函数均只有一行。
推荐在开始前学习 \href{https://learnyouahaskell.mno2.org/zh-cn}{教程} 第4章、第6章的内容。
\end{problem}
\begin{solution}
  \footnote{作业压缩包中含有上述程序的源代码 haskell2-sol.hs,你可以将修改后的程序放入该文件中并与作业一起上传。}

  \lstinputlisting[language=Haskell]{haskell2-sol.hs}
\end{solution}
%%%%%%%%%%%%%%%

%%%%%%%%%%%%%%%%%%%%
\beginot

%%%%%%%%%%%%%%%
\begin{ot}[Lambda Calculus]
  请介绍 lambda 演算的历史和主要概念。

  \noindent 参考资料:
  \begin{itemize}
    \item \href{https://en.wikipedia.org/wiki/Lambda\_calculus}{Lambda Calculus @ wiki}
    \item \href{https://plato.stanford.edu/entries/lambda-calculus/}{Stanford Encyclopedia of Philosophy}
  \end{itemize}
\end{ot}

\begin{ot}[函数式编程]
  以Haskell为例,请介绍函数式编程语言如何体现Lambda Calculus的主要概念。

  \noindent 参考资料:
  \begin{itemize}
    \item \href{https://en.wikipedia.org/wiki/Functional\_programming}{Functional Programming @ wiki}
  \end{itemize}
\end{ot}
%%%%%%%%%%%%%%%

%%%%%%%%%%%%%%%%%%%%
\begincorrection
%%%%%%%%%%%%%%%%%%%%

%%%%%%%%%%%%%%%%%%%%
\beginfb

你可以写
~\footnote{优先推荐 \href{problemoverflow.top}{ProblemOverflow}}:
\begin{itemize}
  \item 对课程及教师的建议与意见
  \item 教材中不理解的内容
  \item 希望深入了解的内容
  \item $\cdots$
\end{itemize}
%%%%%%%%%%%%%%%%%%%%
\end{document}