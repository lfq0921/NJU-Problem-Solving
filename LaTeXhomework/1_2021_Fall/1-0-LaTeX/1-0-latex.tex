\documentclass[12pt, a4paper, oneside]{ctexart}

\usepackage{amsmath, amsthm, amssymb, mathrsfs, graphicx}
\usepackage{tcolorbox, hyperref, enumerate}

\input{hw-preamble}

\title{\vspace{-2em}\textbf{第0讲:\LaTeX}}
\me{李四}{210000000}{}{}
\date{\today}

\begin{document}

\maketitle
\section*{Problem 1}
\begin{problem}
证明:
$$
  Pr(\bigcup_{i=1}^n A_i) \leq \sum_{i=1}^n Pr(A_i)
$$
\end{problem}

\begin{proof}
  概率空间的定义:非负性和可加性。

  首先,证明一个性质$$P(A\cup B) \leq P(A) + P(B)$$(事件的并概率上界)

  设两个相交的事件,即A和B。

  可知$A \cup B = A\cup (A^c \cap B), B=(A\cap B)\cup (A^c \cap B)$

  由可加性可知:

  $$P(A\ cup B) = P(A)+P(A^c \cap B)$$
  $$P(B) = P(A\cap B) + P(A^c\cap B)$$

  $$\Rightarrow P(A \cup B) = P(A)+P(B) - P(A \cap B)$$

  由非负性可知, $$P(A \cup B) \leq P(A)+P(B)$$

  至此证成事件的并概率上界性质;

  可以将此性质用于$A_1$和$A_2\cup A_3 \cup ... \cup A_n$

  $$Pr(A_1) \cup Pr(A_2\cup A_3 \cup ... \cup A_n) \leq Pr(A_1) + Pr(A_2\cup A_3 \cup ... \cup A_n)$$

  再用此方法计算$Pr(A_2)$和$Pr(A_3\cup A_4 \cup ... \cup A_n)$;

  得到$$Pr(A_2) \cup Pr(A_3\cup A_4 \cup ... \cup A_n) \leq Pr(A_2) + Pr(A_3\cup A_4 \cup ... \cup A_n)$$

  以此类推,可得

  $$\Rightarrow Pr(A_1 \cup A_2 \cup .. \cup A_n) \leq Pr(A_1) + Pr(A_2) + ... + Pr(A_n)$$
  即
  $$Pr(\bigcup_{i=1}^n A_i) \leq \sum_{i=1}^n Pr(A_i)
  $$
\end{proof}


\begin{problem}
[Principle of Inclusion and Exclusion (PIE)] Prove that
$\mathbf{Pr}\left( \bigcup_{i=1}^n A_i\right) = \sum_{\emptyset \neq S \subseteq [n]} (-1)^{|S|-1} \mathbf{Pr}\left( \bigcap_{i \in S} A_i \right) $
, where$ [n]=\{1,2,\ldots,n\} $.
\end{problem}

\begin{proof}

  将用数学归纳法证明.

  Consider a single set $A_1$. Then the principle of inclusion-exclusion states that $|A_1| = |A_1| + | A_1 | - | A_1 \cap A_1 | = | A_1 |$, which is trivially true. Now consider a collection of exactly two sets $A$ and $B$. Then $|A \cup B| = |A| + |B| - |A \cap B|$. Assume that the principle of inclusion-exclusion holds for unions of $n$ terms. By grouping terms, and simplifying some of them, the principle can be deduced for unions of $n+1$ terms⁴.

  Therefore, $\mathbf{Pr}\left( \bigcup_{i=1}^n A_i\right) = \sum_{\emptyset \neq S \subseteq [n]} (-1)^{|S|-1} \mathbf{Pr}\left( \bigcap_{i \in S} A_i \right) $, where $[n]=\{1,2,\ldots,n\}$.


\end{proof}

\begin{note}
  无论 \textbackslash itemlize 还是\textbackslash enumerate 中的 \textbackslash item 都支持一个可选参数以临时更换列表标志(即无序列表前的点或有序列表前的 ``1.'')。
  此外,使用 \textbackslash enumerate 的可选参数也可以改变有序列表的图标。当然,这些列表是可以嵌套的。
\end{note}

\begin{problem}
请用 \LaTeX 输出下图中的公式。

\includegraphics[width=1\textwidth]{figs/formula}
\end{problem}

\begin{solution}

\end{solution}

\begin{note}
  当你不认识某些数学符号的时候,你可以使用 \href{http://detexify.kirelabs.org/classify.html}{Detexify} 或 \href{https://mathpix.com/}{mathpix} 等工具进行识别。
  你也可以使用 \href{https://latex.codecogs.com/legacy/eqneditor/editor.php}{Online LaTeX Equation Editor} 或者你编辑器中的符号表进行输入。
\end{note}

\begin{problem}
请用 \LaTeX 输出下图中的表格。

\includegraphics[width=.3\textwidth]{figs/table}
\end{problem}

\begin{solution}

\end{solution}

\begin{note}
  如果你觉得 \LaTeX 的表格填起来太麻烦,你也可以使用 \href{https://www.tablesgenerator.com/}{TablesGenerator} 帮你生成。
\end{note}

\begin{problem}
(此部分为选做)本学期还有可能需要你编写一些伪代码,\LaTeX 中当然有相应的宏包—— \href{http://tug.ctan.org/macros/latex/contrib/algorithmicx/algorithmicx.pdf}{algorithmicx} 。

请用 \LaTeX 输出下图中的算法。

\includegraphics[width=\textwidth]{figs/algorithm}
\end{problem}

\begin{solution}

\end{solution}

\end{document}