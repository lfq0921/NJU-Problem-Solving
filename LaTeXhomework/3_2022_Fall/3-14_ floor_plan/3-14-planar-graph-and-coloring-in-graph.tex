% 2-15-rb-tree.tex

%%%%%%%%%%%%%%%%%%%%
\documentclass[a4paper, justified]{tufte-handout}

\input{hw-preamble} % feel free to modify this file
%%%%%%%%%%%%%%%%%%%%
\title{第3-14讲: 平面图与图的染色}
\me{林凡琪}{211240042}{}{}
\date{\zhtoday} % or like 2019年9月13日
%%%%%%%%%%%%%%%%%%%%
\begin{document}
\maketitle
%%%%%%%%%%%%%%%%%%%%
\noplagiarism % always keep this line
%%%%%%%%%%%%%%%%%%%%
\begin{abstract}
  % \begin{center}{\fcolorbox{blue}{yellow!60}{\parbox{0.65\textwidth}{\large 
  %   \begin{itemize}
  %     \item 
  %   \end{itemize}}}}
  % \end{center}
\end{abstract}
%%%%%%%%%%%%%%%%%%%%
\beginrequired

%%%%%%%%%%%%%%%
\begin{problem}[CZ 9.3]
\end{problem}

\begin{solution}
  We can know that:
  \begin{equation}
    p=
    \left\{
    \begin{array}{ll}
      1,                          & f(x_1)\leq f(x) \\
      e^{-\frac{f(x_1)-f(x)}{T}}, & f(x_1)> f(x)
    \end{array}
    \right.
  \end{equation}
  \\
  (a)
  $m = \frac{(3+4+4+4+5+6+6)}{2} = 16, n = 6$.\\
  For $m> 3n - 6$, so G is nonplanar.\\
  \\
  (b)
  $m = \frac{(4+4+4+5+5+5+6+6+6+7+7+7)}{2} = 33, n = 12$.\\
  For $m> 3n - 6$, so G is nonplanar.
\end{solution}
%%%%%%%%%%%%%%%

%%%%%%%%%%%%%%%
\begin{problem}[CZ 9.5]
\end{problem}

\begin{solution}
  (a)
  \begin{figure}[htbp]
    \centering
    \includegraphics[width = 0.30\linewidth]{figs/a}
  \end{figure}
  \\
  $K_5$\\
  (b)
  \begin{figure}[htbp]
    \centering
    \includegraphics[width = 0.30\linewidth]{figs/b}
  \end{figure}
  \\
  $K_6$\\
  (c)Since $\frac{r\times n}{2} \leq 3n-6$, we can know that there is no r-regular planar graph for $r\geq 6$.
\end{solution}
%%%%%%%%%%%%%%%

%%%%%%%%%%%%%%%
\begin{problem}[CZ 9.7]

\end{problem}

\begin{solution}
  (a)$C_4$\\
  (b)There is no such graph.The graph of order 4 doesn't include $K_{3,3},K_5$ and subgraphs for their segmentation.\\
  (c)
  \begin{figure}[htbp]
    \centering
    \includegraphics[width = 0.30\linewidth]{figs/c}
  \end{figure}\\
  (d)Since $n=5,m=10$, we can know it is $K_5$. So it can not be a planar graph.\\
  (e)$C_3$\\
  (f)\begin{figure}[htbp]
    \centering
    \includegraphics[width = 0.30\linewidth]{figs/d}
  \end{figure}\\
\end{solution}
%%%%%%%%%%%%%%%

%%%%%%%%%%%%%%%
\begin{problem}[CZ 9.8]
\end{problem}

\begin{solution}
  $G=K_4\times K_2$ can not be a planar graph.\\
  The subgraph of $K_4 \times K_2$ contains a breakdown of $K_5$, as shown in the figure.\\
  \begin{figure}[htbp]
    \centering
    \includegraphics[width = 0.30\linewidth]{figs/e}
  \end{figure}\\
\end{solution}
%%%%%%%%%%%%%%%


%%%%%%%%%%%%%%%
\begin{problem}[CZ 10.2]
\end{problem}

\begin{solution}
  (a)
  If the maximum degree of the Petersen diagram is 3 and contains odd circles, its color number is 3.\\
  (b)
  n-cube is a bipartite with a color number of 2\\
  (c)
  When $n=1$, it is the number of color 2.\\
  $n > 1$ and $n$ is even, the number of colors is 3.\\
  $n > 1$ and $n$ is odd, the color number is 4.
\end{solution}
%%%%%%%%%%%%%%%

%%%%%%%%%%%%%%%
\begin{problem}[CZ 10.3]
\end{problem}

\begin{solution}
  If there is only 1 point, the number of colors is 1. Otherwise, the number of colors is 2.\\
\end{solution}
%%%%%%%%%%%%%%%

%%%%%%%%%%%%%%%
\begin{problem}[CZ 10.4]
\end{problem}

\begin{solution}
  (a)
  No, $K_4$ is a counterexample. \\
  (b)
  No, $C_4$ is a counterexample. \\
  (c)
  No, $K_2$ is a counterexample. \\
  (d)
  No, $K_{3,3}$ is a counterexample.
\end{solution}
%%%%%%%%%%%%%%%

%%%%%%%%%%%%%%%
\begin{problem}[CZ 10.5]
\end{problem}

\begin{solution}
  Suppose it can be divided into three independent sets of $ V_1, V_2, V_3 $. \\
  Generally assumes $ | v_1 | \geq | v_2 | \geq | V_3 | $. \\
  If $ | v_1 | \geq 3 $, the number of edges is $ \leq | k_6 |-| k_ {v_1} | = 12 $ \\
  Otherwise, $ | v_1 | = | v_2 | = | v_3 | = 2 $, then number of edges $ \geq | K_6 | -3 = 12 $
\end{solution}
%%%%%%%%%%%%%%%
%%%%%%%%%%%%%%%%%%%%
\beginot
%%%%%%%%%%%%%%%

%%%%%%%%%%%%%%%
\begin{ot}[请证明Brooks定理]
  \textbf{(Brooks’ Theorem)}For every connected graph G that is not an odd cycle or a
  complete graph, $\chi(G) \le \Delta(G)$
\end{ot}

% \begin{solution}
% \end{solution}
%%%%%%%%%%%%%%%

\begin{ot}[Martin Gardner的愚人节礼物]
  《科学美国人》即《Scientific American》,是美国出版的一种著名科学杂志,在国际上极富声誉。该刊1975年4月号上登载了著
  名数学专栏作家,马丁·加德纳(Martin Gardner)的一篇文章。文章附了一张有着110个区域的地图:
  \fig{width=0.5\linewidth}{figs/Martin-Gardner-fool‘s-day-gift.png}

  加德纳在该图下赫然写道:“四色定理被推翻了!”正文中他还语气肯定地说:该地图不能用少于5种颜色使相邻区域着不同颜色。

  请问:四色定理真的被推翻了么?
\end{ot}

% \begin{solution}
% \end{solution}
%%%%%%%%%%%%%%%




% \vspace{0.50cm}
%%%%%%%%%%%%%%%
% \begin{ot}[]
% 
%   \noindent 参考资料:
%   \begin{itemize}
%     \item 
%   \end{itemize}
% \end{ot}

% \begin{solution}
% \end{solution}
%%%%%%%%%%%%%%%

%%%%%%%%%%%%%%%%%%%%
% 如果没有需要订正的题目,可以把这部分删掉

% \begincorrection
%%%%%%%%%%%%%%%%%%%%

%%%%%%%%%%%%%%%%%%%%
% 如果没有反馈,可以把这部分删掉
\beginfb

% 你可以写
% ~\footnote{优先推荐 \href{problemoverflow.top}{ProblemOverflow}}:
% \begin{itemize}
%   \item 对课程及教师的建议与意见
%   \item 教材中不理解的内容
%   \item 希望深入了解的内容
%   \item $\cdots$
% \end{itemize}
%%%%%%%%%%%%%%%%%%%%
% \bibliography{2-5-solving-recurrence}
% \bibliographystyle{plainnat}
%%%%%%%%%%%%%%%%%%%%
\end{document}