% 2-15-rb-tree.tex

%%%%%%%%%%%%%%%%%%%%
\documentclass[a4paper, justified]{tufte-handout}

\input{hw-preamble} % feel free to modify this file
%%%%%%%%%%%%%%%%%%%%
\title{第3-9讲: 多源最短路}
\me{林凡琪}{211240042}{}{}
\date{\zhtoday} % or like 2019年9月13日
%%%%%%%%%%%%%%%%%%%%
\begin{document}
\maketitle
%%%%%%%%%%%%%%%%%%%%
\noplagiarism % always keep this line
%%%%%%%%%%%%%%%%%%%%
\begin{abstract}
  % \begin{center}{\fcolorbox{blue}{yellow!60}{\parbox{0.65\textwidth}{\large 
  %   \begin{itemize}
  %     \item 
  %   \end{itemize}}}}
  % \end{center}
\end{abstract}
%%%%%%%%%%%%%%%%%%%%
\beginrequired

%%%%%%%%%%%%%%%
\begin{problem}[TC 25.1-5]
说明如何将单源最短路径问题表示为矩阵和向量的乘积,并解释该乘积的计算过程如何对应Bellman-Ford算法?(请参阅24.1节)
\end{problem}

\begin{solution}
  向量就是所有节点对矩阵L的一行(行号为单源起点),W依然是权重矩阵。矩阵乘法中,向量与W的第i列相乘对应算法第3,4行中对 (?, i) 这类的边进行松弛。
\end{solution}
%%%%%%%%%%%%%%%

%%%%%%%%%%%%%%%
\begin{problem}[TC 25.1-6]
假定我们还希望在本节所讨论的算法里计算出最短路径上的节点,说明如何在$O(n^3)$的时间内从已经计算出的最短路径权值矩阵L计算出前驱矩阵$\Pi$
\end{problem}

\begin{solution}
  \noindent verbatim:
  \begin{verbatim}
  $FING-\Pi(L, w)$
	for i <- 1 to n
		for j <- 1 to n
			for k <- 1 to n
				do if L(i,k)+w(k,j) = L(i,j)
					do Π(i,j) = k
\end{verbatim}
\end{solution}
%%%%%%%%%%%%%%%

%%%%%%%%%%%%%%%
\begin{problem}[TC 25.1-9]
修改$FASTER-ALL-PAIRS-SHORTEST-PATHS$,使其可以判断一个图是否包含一个权重为负值的环路。
\end{problem}

\begin{solution}
  如果$L^{n-1}$矩阵上的元素有负值,就说明有负权回路.\\
  如果是有三个节点,ABC,由A到B为-1,B到C为-1,C到A为-1,最后的$L^{(n-1)}$矩阵的对角线不会是负值。\\
  因为算法考虑的是最多n-1条边的情况,而负环的最大可能性是n条边,所以如果不多循环一次的话是无法检测出的;另外也可以多循环一次,查看两个矩阵是否有变化,这样是$O(n^2)$,而仅仅是多循环一次检查对角线的话只用$O(n)$
\end{solution}
%%%%%%%%%%%%%%%

%%%%%%%%%%%%%%%
\begin{problem}[TC 25.1-10]
给出一个有效算法来在图中找到最短长度的权重为负值的环路的长度(边的条数)。
\end{problem}

\begin{solution}
  在每次计算$L^m$时检查对角线有没有负值出现,第一次出现负值的m即为答案。
\end{solution}
%%%%%%%%%%%%%%%

%%%%%%%%%%%%%%%
\begin{problem}[TC 25.2-2]
说明该如何使用25.1节的技术来计算传递闭包。
\end{problem}

\begin{solution}
  将EXTEND-SHORTEST-PATHS中第7行的min换成OR,+换成AND
\end{solution}
%%%%%%%%%%%%%%%

%%%%%%%%%%%%%%%
\begin{problem}[TC 25.2-4]
\end{problem}

\begin{solution}
  当然是正确的,类似于背包问题。这个动态规划只需要保存上一个状态。也就是要计算当前这个状态只需要借助上一个状态,并不需要真的储存所有状态。
\end{solution}
%%%%%%%%%%%%%%%


%%%%%%%%%%%%%%%
\begin{problem}[TC 25.2-6]
我们怎样才能使用$Floyd-Warshall$算法的输出来检测权重为负值的环路?
\end{problem}

\begin{solution}
  权重矩阵D对角线上出现了负值。
  或者,在正常的Floyd-Warshall算法完成后再多跑一个循环,如果有一个值还能更新则说明有负权回路。
\end{solution}
%%%%%%%%%%%%%%%

%%%%%%%%%%%%%%%
\begin{problem}[TC 25.2-8]
给出一个$O(VE)$时间复杂度的算法来计算有向图$G=(V,E)$的传递闭包。
\end{problem}

\begin{solution}
  对每个节点都跑一次DFS. 一共V个点,E条边,所以是O(VE)的时间.
  \begin{algorithm}[H]
    \begin{algorithmic}[1]
      \Procedure{chuandibibao}{$G(V,E)$}
      \For {$i = vertice\in G$}
      \State $DFS(i)$;
      \EndFor
      \EndProcedure
    \end{algorithmic}
  \end{algorithm}
\end{solution}
%%%%%%%%%%%%%%%

%%%%%%%%%%%%%%%
\begin{problem}[TC 25.3-2]
What is the purpose of adding the new vertex s to V , yielding V' ?
\end{problem}

\begin{solution}
  无论我们从哪里开始BellmanFord's的算法,其中一些顶点成本将是无限的。Johnson的算法通过在图中添加一个新的顶点s来避免这个问题,零权重边从s到其他每个顶点,但没有边回到s。此添加不会更改任何其他顶点对之间的最短路径,因为没有到达 s 的路径。
\end{solution}
%%%%%%%%%%%%%%%

%%%%%%%%%%%%%%%
\begin{problem}[TC 25.3-3]
Suppose that $w(u,v) >= 0$ for all edges $(u,v)$ belonging to E. What is the relationship between the weight functions $w$ and $w'$?
\end{problem}

\begin{solution}
  在这种情况下,对于属于图的所有顶点 $v$,$h(v)$ 将只是 0,因为从 $s$ 到任何顶点的最短路径是从 s 到该顶点的直接路径,并且所有这些边的权重为 0。因此,$w' = w$。
\end{solution}
%%%%%%%%%%%%%%%



%%%%%%%%%%%%%%%%%%%%
\beginoptional

%%%%%%%%%%%%%%%
\begin{problem}[TC Problem 25-2]
\end{problem}

\begin{solution}
\end{solution}
%%%%%%%%%%%%%%%


%%%%%%%%%%%%%%%%%%%%
\beginot
%%%%%%%%%%%%%%%
%%%%%%%%%%%%%%%
\begin{ot}[Constructing Shortest Path with Floyd-Warshall]
  我们如何在Floyd-Warshall算法基础上,构造最短路径?


\end{ot}

% \begin{solution}
% \end{solution}
%%%%%%%%%%%%%%%

\begin{ot}[Parallel all-pairs shortest path algorithm]
  \noindent 参考资料:
  \begin{itemize}
    \item \href{https://en.wikipedia.org/wiki/Parallel_all-pairs_shortest_path_algorithm}{https://en.wikipedia.org/wiki/Parallel\_all-pairs\_shortest\_path\_algorithm}
  \end{itemize}

\end{ot}

% \begin{solution}
% \end{solution}
%%%%%%%%%%%%%%%




% \vspace{0.50cm}
%%%%%%%%%%%%%%%
% \begin{ot}[]
% 
%   \noindent 参考资料:
%   \begin{itemize}
%     \item 
%   \end{itemize}
% \end{ot}

% \begin{solution}
% \end{solution}
%%%%%%%%%%%%%%%

%%%%%%%%%%%%%%%%%%%%
% 如果没有需要订正的题目,可以把这部分删掉

% \begincorrection
%%%%%%%%%%%%%%%%%%%%

%%%%%%%%%%%%%%%%%%%%
% 如果没有反馈,可以把这部分删掉
\beginfb

% 你可以写
% ~\footnote{优先推荐 \href{problemoverflow.top}{ProblemOverflow}}:
% \begin{itemize}
%   \item 对课程及教师的建议与意见
%   \item 教材中不理解的内容
%   \item 希望深入了解的内容
%   \item $\cdots$
% \end{itemize}
%%%%%%%%%%%%%%%%%%%%
% \bibliography{2-5-solving-recurrence}
% \bibliographystyle{plainnat}
%%%%%%%%%%%%%%%%%%%%
\end{document}