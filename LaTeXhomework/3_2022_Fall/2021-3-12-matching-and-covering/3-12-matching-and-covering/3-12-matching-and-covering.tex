% 2-15-rb-tree.tex

%%%%%%%%%%%%%%%%%%%%
\documentclass[a4paper, justified]{tufte-handout}

\input{hw-preamble} % feel free to modify this file
%%%%%%%%%%%%%%%%%%%%
\title{第3-12讲: 图中的匹配与覆盖}
\me{林凡琪}{211240042}{}{}
\date{\zhtoday} % or like 2019年9月13日
%%%%%%%%%%%%%%%%%%%%
\begin{document}
\maketitle
%%%%%%%%%%%%%%%%%%%%
\noplagiarism % always keep this line
%%%%%%%%%%%%%%%%%%%%
\begin{abstract}
  % \begin{center}{\fcolorbox{blue}{yellow!60}{\parbox{0.65\textwidth}{\large 
  %   \begin{itemize}
  %     \item 
  %   \end{itemize}}}}
  % \end{center}
\end{abstract}
%%%%%%%%%%%%%%%%%%%%
\beginrequired

%%%%%%%%%%%%%%%
\begin{problem}[CZ 8.3]
\end{problem}

\begin{solution}
  对G1:\\
  由$\alpha_1(G_1)+\beta_1(G_1) = n$可知,因为边覆盖数为$\alpha_1(G_1)=\lceil n/2 \rceil=5$所以$\beta_1(G_1)=5$\\
  匹配方案之一:\\
  $\{a,w\},\{b,z\},\{c,v\},\{d,x\},\{e,y\}$\\
  所以对于G1,U可以匹配到W上\\
  对G2:\\
  对于G2,U不能匹配到W上,因为存在U的含四个顶点的子集$X=\{v,x,y\}$,而$|N(X)| = 2 < 3 = |X|$,所以G2是不友好的,U不能匹配到W上.
\end{solution}
%%%%%%%%%%%%%%%

%%%%%%%%%%%%%%%
\begin{problem}[CZ 8.5]
证明:任一树之多包含一个完美匹配.
\end{problem}

\begin{solution}
  利用叶子节点在晚辈匹配中的唯一性.对于每个叶子节点,它只能和与它唯一相邻的点匹配,如果有一个结点连接了两个及以上的叶子节点,那么其中至少会有一个叶子节点不能匹配,则此时不存在完美匹配\\
  所以只有每个结点最多只与一个叶子节点项链是,才有可能存在完美匹配.于是在这种情况下 ,可以去掉叶子节点和它相邻的结点回得到森林,对森林中的每棵树上不断重复上面过程,知道所有的结点都被匹配或者有点不能被匹配.在这个过程中,如果存在完美匹配,匹配的方法都是唯一确定的.
\end{solution}
%%%%%%%%%%%%%%%

%%%%%%%%%%%%%%%
\begin{problem}[CZ 8.14]
证明:不含孤立点的图G有完美匹配当且仅当$\alpha_1(G)=\beta_1(G)$.
\end{problem}

\begin{solution}
  因为对于任意不包含 孤立点的图G,有$\alpha_1(G)+\beta_1(G)=n$又因为$\alpha_1(G)=\beta_1(G)$所以$\alpha_1(G)=\beta_1(G) = n/2$\\
  即有$n/2$条边互不邻接,这$n/2$条边覆盖了n个点,所以选取这$n/2$条边作为集合,能够得到一个完美匹配.
\end{solution}
%%%%%%%%%%%%%%%

%%%%%%%%%%%%%%%
\begin{problem}[CZ 8.16]
证明:设$G$中最小点覆盖集为$S$。\\
设$T$为$G-S$。由点覆盖集性质可得$S$在$G$中所连点全在$T$中,则$|T|\leq \delta \times |S|$\\
则有$n-\beta(G)\leq \delta \times \beta(G)$\\
得 $\beta(G)\geq \frac{n}{\delta + 1}$
\end{problem}

\begin{solution}
  覆盖所有边的顶点个数.\\
  最大度越大, 覆盖所有边的顶点个数越少
\end{solution}
%%%%%%%%%%%%%%%

%%%%%%%%%%%%%%%
\begin{problem}[CZ 8.18]
列举一个不含1因子的5正则图
\end{problem}

\begin{solution}

\end{solution}
%%%%%%%%%%%%%%%

%%%%%%%%%%%%%%%
\begin{problem}[CZ 8.21]
Use Tutte’s characterization of graphs with 1-factors ( Theorem 8.10) to show that $K_{3,5}$ does not have a 1-factor.
\end{problem}

\begin{solution}
  设其两个部分$S$和$T$中,$|S|= 3$, $|T|=5$。\\
  则有$K_0(G-S)=5<|S|$\\
  由定理8.10可知,$G$不含有1因子。
\end{solution}
%%%%%%%%%%%%%%%

%%%%%%%%%%%%%%%
\begin{problem}[CZ 8.24]
证明:任一不含割边的3正则图包含2因子
\end{problem}

\begin{solution}
  没有割边的3正则图一定含有1因子。在该图中去掉这个完美匹配后的图为$2$正则子图。\\
  由定理可知,原没有割边的3正则图含有2因子。
\end{solution}
%%%%%%%%%%%%%%%

%%%%%%%%%%%%%%%%%%%%
\beginot
%%%%%%%%%%%%%%%
%%%%%%%%%%%%%%%
\begin{ot}[点独立与点覆盖]
  请证明定理8.8。在证明中,请你给出以下思考:关于点覆盖/独立的所有相关定理,是否在边覆盖/独立讨论范畴内,均有相应的定理?你能“杜撰”出几条吗?

\end{ot}

% \begin{solution}
% \end{solution}
%%%%%%%%%%%%%%%

\begin{ot}[Kőnig's theorem]
  \noindent 参考资料:
  \begin{itemize}
    \item \href{https://en.wikipedia.org/wiki/K\%C5\%91nig\%27s_theorem_(graph_theory)}{https://en.wikipedia.org/wiki/K\%C5\%91nig\%27s\_theorem\_(graph\_theory)}
  \end{itemize}

\end{ot}

% \begin{solution}
% \end{solution}
%%%%%%%%%%%%%%%




% \vspace{0.50cm}
%%%%%%%%%%%%%%%
% \begin{ot}[]
% 
%   \noindent 参考资料:
%   \begin{itemize}
%     \item 
%   \end{itemize}
% \end{ot}

% \begin{solution}
% \end{solution}
%%%%%%%%%%%%%%%

%%%%%%%%%%%%%%%%%%%%
% 如果没有需要订正的题目,可以把这部分删掉

% \begincorrection
%%%%%%%%%%%%%%%%%%%%

%%%%%%%%%%%%%%%%%%%%
% 如果没有反馈,可以把这部分删掉
\beginfb

% 你可以写
% ~\footnote{优先推荐 \href{problemoverflow.top}{ProblemOverflow}}:
% \begin{itemize}
%   \item 对课程及教师的建议与意见
%   \item 教材中不理解的内容
%   \item 希望深入了解的内容
%   \item $\cdots$
% \end{itemize}
%%%%%%%%%%%%%%%%%%%%
% \bibliography{2-5-solving-recurrence}
% \bibliographystyle{plainnat}
%%%%%%%%%%%%%%%%%%%%
\end{document}