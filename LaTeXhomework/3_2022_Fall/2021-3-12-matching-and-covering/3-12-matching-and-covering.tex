% 2-15-rb-tree.tex

%%%%%%%%%%%%%%%%%%%%
\documentclass[a4paper, justified]{tufte-handout}

\input{hw-preamble} % feel free to modify this file
%%%%%%%%%%%%%%%%%%%%
\title{第3-12讲: 图中的匹配与覆盖}
\me{朱宇博}{191220186}{}{}
\date{\zhtoday} % or like 2019年9月13日
%%%%%%%%%%%%%%%%%%%%
\begin{document}
\maketitle
%%%%%%%%%%%%%%%%%%%%
\noplagiarism % always keep this line
%%%%%%%%%%%%%%%%%%%%
\begin{abstract}
  % \begin{center}{\fcolorbox{blue}{yellow!60}{\parbox{0.65\textwidth}{\large 
  %   \begin{itemize}
  %     \item 
  %   \end{itemize}}}}
  % \end{center}
\end{abstract}
%%%%%%%%%%%%%%%%%%%%
\beginrequired

%%%%%%%%%%%%%%%
\begin{problem}[CZ 8.3]
 Figure 8.5 shows two bipartite graphs G1 and G2, each with partite sets U = {v, w, x, y, z } and W = {a, b, c, d, e}. In each case, can U be matched to W?
 \begin{figure}[htbp]
    \centering
    \includegraphics[width = 0.50\linewidth]{figs/a}
  \end{figure}
\end{problem}

\begin{solution}
G1可以。(a,v), (b,w), (c,y), (d,z),(e,x)\\
G2不行,$b,d,e$三点的领接点数量为$2$,小于$3$
\end{solution}
%%%%%%%%%%%%%%%

%%%%%%%%%%%%%%%
\begin{problem}[CZ 8.5]
Prove that every tree has at most one perfect matching.
\end{problem}

\begin{solution}
假设树$G$含有完美匹配。\\
考虑其叶子结点,叶子结点的度数均为$1$,故其只存在一种匹配方式,即和自己的父亲节点匹配。将这些确定的匹配方式所含点删去,得到图$G_1$。\\
重复该操作,直至最后只有两个结点,将其匹配。此时已经完成所有点对的匹配。\\
由此可见。对于一棵树,若其存在完美匹配,则匹配方案唯一。\\
故树最多有一种完美匹配
\end{solution}
%%%%%%%%%%%%%%%

%%%%%%%%%%%%%%%
\begin{problem}[CZ 8.14]
 Prove that a graph G without isolated vertices has a perfect matching if and only if   $\alpha_1(G) = \beta_1(G)$.
\end{problem}

\begin{solution}
若$n$j阶图$G$有完美匹配,则有最大匹配数=$\beta_1{G}=\frac{n}{2}$。\\
由定理可知$\beta_1{G}+\alpha_1{G}=n$,故 $\alpha_1(G) = \beta_1(G)$。\\
若$\alpha_1(G) = \beta_1(G)$,则$\alpha_1(G) = \beta_1(G)=\frac{n}{2}$\\
因此图的最大匹配数为$\frac{n}{2}$,存在完美匹配。\\
综上,得证
\end{solution}
%%%%%%%%%%%%%%%

%%%%%%%%%%%%%%%
\begin{problem}[CZ 8.16]
 Prove that if G is a graph of order n, maximum degree $\delta$ and having no isolated vertices, then $\beta(G)\geq \frac{n}{\delta + 1}$
\end{problem}

\begin{solution}
设$G$中最小点覆盖集为$S$。\\
设$T$为$G-S$。由点覆盖集性质可得$S$在$G$中所连点全在$T$中,则$|T|\leq \delta \times |S|$\\
则有$n-\beta(G)\leq \delta \times \beta(G)$\\
得 $\beta(G)\geq \frac{n}{\delta + 1}$
\end{solution}
%%%%%%%%%%%%%%%

%%%%%%%%%%%%%%%
\begin{problem}[CZ 8.18]
Give an example of a 5-regular graph that contains no 1-factor.
\end{problem}

\begin{solution}
\newpage
\end{solution}
%%%%%%%%%%%%%%%

%%%%%%%%%%%%%%%
\begin{problem}[CZ 8.21]
Use Tutte’s characterization of graphs with 1-factors ( Theorem 8.10) to show that $K_{3,5}$ does not have a 1-factor.
\end{problem}

\begin{solution}
设其两个部分$S$和$T$中,$|S|= 3$, $|T|=5$。\\
则有$K_0(G-S)=5<|S|$\\
由定理8.10可知,$G$不含有1因子。
\end{solution}
%%%%%%%%%%%%%%%

%%%%%%%%%%%%%%%
\begin{problem}[CZ 8.24]
Show that every 3-regular bridgeless graph contains a 2-factor.
\end{problem}

\begin{solution}
没有割边的3正则图一定含有1因子。在该图中去掉这个完美匹配后的图为$2$正则子图。\\
由定理可知,原没有割边的3正则图含有2因子。
\end{solution}
%%%%%%%%%%%%%%%

%%%%%%%%%%%%%%%%%%%%
\beginot
%%%%%%%%%%%%%%%
%%%%%%%%%%%%%%%
\begin{ot}[点独立与点覆盖]
	请证明定理8.8。在证明中,请你给出以下思考:关于点覆盖/独立的所有相关定理,是否在边覆盖/独立讨论范畴内,均有相应的定理?你能“杜撰”出几条吗?
	
\end{ot}

% \begin{solution}
% \end{solution}
%%%%%%%%%%%%%%%

\begin{ot}[Kőnig's theorem]	
请你给出一种合理的循环赛排名方法
\noindent 参考资料:
  \begin{itemize}
     \item \href{https://en.wikipedia.org/wiki/K\%C5\%91nig\%27s_theorem_(graph_theory)}{https://en.wikipedia.org/wiki/K\%C5\%91nig\%27s\_theorem\_(graph\_theory)}
  \end{itemize}
		
\end{ot}

% \begin{solution}
% \end{solution}
%%%%%%%%%%%%%%%




% \vspace{0.50cm}
%%%%%%%%%%%%%%%
% \begin{ot}[]
% 
%   \noindent 参考资料:
%   \begin{itemize}
%     \item 
%   \end{itemize}
% \end{ot}

% \begin{solution}
% \end{solution}
%%%%%%%%%%%%%%%

%%%%%%%%%%%%%%%%%%%%
% 如果没有需要订正的题目,可以把这部分删掉

% \begincorrection
%%%%%%%%%%%%%%%%%%%%

%%%%%%%%%%%%%%%%%%%%
% 如果没有反馈,可以把这部分删掉
\beginfb

% 你可以写
% ~\footnote{优先推荐 \href{problemoverflow.top}{ProblemOverflow}}:
% \begin{itemize}
%   \item 对课程及教师的建议与意见
%   \item 教材中不理解的内容
%   \item 希望深入了解的内容
%   \item $\cdots$
% \end{itemize}
%%%%%%%%%%%%%%%%%%%%
% \bibliography{2-5-solving-recurrence}
% \bibliographystyle{plainnat}
%%%%%%%%%%%%%%%%%%%%
\end{document}