% 2-15-rb-tree.tex

%%%%%%%%%%%%%%%%%%%%
\documentclass[a4paper, justified]{tufte-handout}

\input{hw-preamble} % feel free to modify this file
%%%%%%%%%%%%%%%%%%%%
\title{第3-3讲: 均摊分析}
\me{林凡琪}{211240042}{}{}
\date{\zhtoday} % or like 2019年9月13日
%%%%%%%%%%%%%%%%%%%%
\begin{document}
\maketitle
%%%%%%%%%%%%%%%%%%%%
\noplagiarism % always keep this line
%%%%%%%%%%%%%%%%%%%%
\begin{abstract}
    % \begin{center}{\fcolorbox{blue}{yellow!60}{\parbox{0.65\textwidth}{\large 
    %   \begin{itemize}
    %     \item 
    %   \end{itemize}}}}
    % \end{center}
\end{abstract}
%%%%%%%%%%%%%%%%%%%%
\beginrequired

%%%%%%%%%%%%%%%
\begin{problem}[TC 17.1-3]
Suppose we perform a sequence of $n$ operations on a data structure in which the $i$ th operation costs $i$ if $i$ is an exact power of $2$, and $1$ otherwise. Use aggregate analysis to determine the amortized cost per operation.
\end{problem}

\begin{solution}
    \[
        \begin{aligned}
            S & \leq\sum\limits_{i = 0}^{log_2n}\frac{n}{2^i} + n - \lfloor log_2n\rfloor \\
              & \leq 3n                                                                   \\
              & = O(n)
        \end{aligned}
    \]
    单个操作复杂度$\frac{O(n)}{n}=O(1)$
\end{solution}
%%%%%%%%%%%%%%%

%%%%%%%%%%%%%%%
\begin{problem}[TC 17.2-2]
Redo Exercise 17.1-3 using an accounting method of analysis.
\end{problem}

\begin{solution}
    设平摊代价为:\\
    对于$2$的幂次方(不含$1$):0\\
    对于其他数:支付$1$, 留下余额$4$\\
    对于数字$2$,数字$1$的余额足以将其支付。\\
    对于数字$2^n(n>1)$,$(2^{n-1},2^n)$之间的整数以将其支付。(当$n\geq2$时,$4*(2^n-2^{n-1}-1)\geq 2^n$)\\
    故总代价$S\leq 5n$\\
    单个操作复杂度$\frac{O(n)}{n}=O(1)$
\end{solution}
%%%%%%%%%%%%%%%

%%%%%%%%%%%%%%%
\begin{problem}[TC 17.4-1]
Suppose that we wish to implement a dynamic, open-address hash table. Why might we consider the table to be full when its load factor reaches some value  ̨ that is strictly less than $1$? Describe briefly how to make insertion into a dynamic, open-address hash table run in such a way that the expected value of the amortized cost per insertion is $O(1)$. Why is the expected value of the actual cost per insertion not necessarily $O(1)$ for all insertions?
\end{problem}

\begin{solution}
    当$\alpha$接近$1$时,平均次数$\frac{1}{1-\alpha}$将会非常大,从而运行效率下降。\\
    在动态开放寻址中,取扩张的$\alpha$为$\frac{1}{2}$,收缩的$\alpha$为$\frac{1}{4}$即可。\\
    在大小为$n$,有$\frac{n}{2}-1$个元素的表中进行插入时,单次插入可能达到$O(n)$。在算法的实际分析中,只需要代价均摊$O(1)$即可。
\end{solution}
%%%%%%%%%%%%%%%

%%%%%%%%%%%%%%%%%%%%
\beginoptional

%%%%%%%%%%%%%%%

%%%%%%%%%%%%%%%

%%%%%%%%%%%%%%%%%%%%
\beginot
%%%%%%%%%%%%%%%
\begin{ot}[Binomial heap]
    请介绍 Binomial heap 的结构以及主要操作,分析其时间复杂度。

    \noindent 参考资料
    \begin{itemize}
        \item \href{https://en.wikipedia.org/wiki/Binomial_heap}{https://en.wikipedia.org/wiki/Binomial\_heap}
    \end{itemize}
\end{ot}

% \begin{solution}
% \end{solution}
%%%%%%%%%%%%%%%

%%%%%%%%%%%%%%%
\begin{ot}[Day–Stout–Warren algorithm]
    请介绍 Day–Stout–Warren algorithm,分析其时间复杂度。

    \noindent 参考资料
    \begin{itemize}
        \item \href{https://en.wikipedia.org/wiki/Day%E2%80%93Stout%E2%80%93Warren_algorithm}{https://en.wikipedia.org/wiki/Day-Warren-algorithm}
    \end{itemize}
\end{ot}

% \begin{solution}
% \end{solution}
%%%%%%%%%%%%%%%


% \vspace{0.50cm}
%%%%%%%%%%%%%%%
% \begin{ot}[]
% 
%   \noindent 参考资料:
%   \begin{itemize}
%     \item 
%   \end{itemize}
% \end{ot}

% \begin{solution}
% \end{solution}
%%%%%%%%%%%%%%%

%%%%%%%%%%%%%%%%%%%%
% 如果没有需要订正的题目,可以把这部分删掉

% \begincorrection
%%%%%%%%%%%%%%%%%%%%

%%%%%%%%%%%%%%%%%%%%
% 如果没有反馈,可以把这部分删掉
\beginfb

% 你可以写
% ~\footnote{优先推荐 \href{problemoverflow.top}{ProblemOverflow}}:
% \begin{itemize}
%   \item 对课程及教师的建议与意见
%   \item 教材中不理解的内容
%   \item 希望深入了解的内容
%   \item $\cdots$
% \end{itemize}
%%%%%%%%%%%%%%%%%%%%
% \bibliography{2-5-solving-recurrence}
% \bibliographystyle{plainnat}
%%%%%%%%%%%%%%%%%%%%
\end{document}