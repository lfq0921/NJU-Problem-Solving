% 2-15-rb-tree.tex

%%%%%%%%%%%%%%%%%%%%
\documentclass[a4paper, justified]{tufte-handout}

\input{hw-preamble} % feel free to modify this file
%%%%%%%%%%%%%%%%%%%%
\title{第3-4讲: 图的基本概念}
\me{林凡琪}{211240042}{}{}
\date{\zhtoday} % or like 2019年9月13日
%%%%%%%%%%%%%%%%%%%%
\begin{document}
\maketitle
%%%%%%%%%%%%%%%%%%%%
\noplagiarism % always keep this line
%%%%%%%%%%%%%%%%%%%%
\begin{abstract}
	% \begin{center}{\fcolorbox{blue}{yellow!60}{\parbox{0.65\textwidth}{\large 
	%   \begin{itemize}
	%     \item 
	%   \end{itemize}}}}
	% \end{center}
\end{abstract}
%%%%%%%%%%%%%%%%%%%%
\beginrequired

%%%%%%%%%%%%%%%
\begin{problem}[CZ 1.2]
仿照例1.1, 自己用9名编辑和8个委员会构造一个例子,并画出相应的图。
\end{problem}

\begin{solution}
	例子:一家大型的出版公司在科学、技术和计算领域内有9名编辑(分别用1,2,...,9)来标记,他们分成8个委员会$c_1 = \{3,4\}, c_2 = \{1,3,5\}, c_3 = \{2,4,6,7\}, c_4 =\{1,3,8,9\},c_5=\{6,7,9\},c_6=\{3,4,8\},c_7=\{1,5\},c_8=\{1,9\}$,这八个委员会在同一天的不同四个时间段内开会.
	\includegraphics[width = 0.8\textwidth]{1.2.jpg}
\end{solution}
%%%%%%%%%%%%%%%

%%%%%%%%%%%%%%%
\begin{problem}[CZ 1.3]
设$S = \{2,3,4,7,11,13\}$,画出一个图$G$,其顶点集是S,而且对于$i,j\in S$,当$i+j\in S$或者$|i-j|\in S$,则$ij\in E(G)$.
\end{problem}

\begin{solution}
	\includegraphics[width = 0.8\textwidth]{1.3.jpg}
\end{solution}
%%%%%%%%%%%%%%%

%%%%%%%%%%%%%%%
\begin{problem}[CZ 1.11]
\end{problem}

\begin{solution}
	\includegraphics[width = 0.8\textwidth]{1.11.png}
\end{solution}
%%%%%%%%%%%%%%%

%%%%%%%%%%%%%%%
\begin{problem}[CZ 1.12]
\end{problem}

\begin{solution}
	(a)W:x,u,r,v,u,v,y\\
	(b)W:v,u,r,v,w\\
	(c)不存在,r到z最少经过3条边\\
	(d)不存在\\
	(e)W:x,u,v,t\\
	(f)W:r,s,t,v,w,z,y,x,u,v,r\\
	(g)W:r,s,t,v,w,z,y,v,r\\
	(h)W:r,v,y,z
\end{solution}
%%%%%%%%%%%%%%%

%%%%%%%%%%%%%%%
\begin{problem}[CZ 1.24]
\end{problem}

\begin{solution}
	\includegraphics[width = 0.8\textwidth]{1.24.jpg}
	对于第二张图不为二分图,因为其存在奇圈${x_2,r_2,w_2,z_2,y_2}$
\end{solution}
%%%%%%%%%%%%%%%

%%%%%%%%%%%%%%%
\begin{problem}[CZ 2.1]
\end{problem}

\begin{solution}
	(a)不存在。其总度和位13,不为偶数,与图论第一定律相悖。\\
	(b)不存在。点数位7的图,度数最大为6.

	(c)不存在。点数为4的图中度数为3,说明其中有3个点与另外所有点都有连边,则不可能出现度数为1的点
\end{solution}
%%%%%%%%%%%%%%%

%%%%%%%%%%%%%%%
\begin{problem}[CZ 2.19]
\end{problem}

\begin{solution}
	\includegraphics[width = 0.8\textwidth]{2.19.jpg}
\end{solution}
%%%%%%%%%%%%%%%


%%%%%%%%%%%%%%%
\begin{problem}[CZ 2.31]
\end{problem}

\begin{solution}
	假设存在G,度序列为$d_1,..,d_n$,则对其补图,原图G 中度为$d_i$的点,在补图中度为$n-d_i-1$,因此补图存在$n-d_1-1,...,n-d_n-1$的度序列,可图。反之亦成立。
\end{solution}
%%%%%%%%%%%%%%%

%%%%%%%%%%%%%%%
\begin{problem}[CZ 3.1]
\end{problem}

\begin{solution}
	\includegraphics[width = 0.8\textwidth]{3.1.jpg}
\end{solution}
%%%%%%%%%%%%%%%

%%%%%%%%%%%%%%%
\begin{problem}[CZ 3.2]
\end{problem}

\begin{solution}
	\includegraphics[width = 0.8\textwidth]{3.2.jpg}
\end{solution}
%%%%%%%%%%%%%%%

%%%%%%%%%%%%%%%%%%%%
\beginoptional

%%%%%%%%%%%%%%%
%\begin{problem}[TBD]
%\end{problem}
%
%\begin{solution}
%\end{solution}
%%%%%%%%%%%%%%%

%%%%%%%%%%%%%%%%%%%%
\beginot
%%%%%%%%%%%%%%%
\begin{ot}[图的应用-1]
	\begin{description}
		\item[\textbf{Tower of Hanoi}] 请尝利用Graph对汉诺塔问题进行建模,并指出在建模得到的图中,原先求解汉诺塔的问题,转换为图论中什么问题。
		\item[\textbf{Pagerank算法}]
			Pagerank如何对网络结构进行建模,以及大概的算法思想。

			参考资料:\href{https://en.m.wikipedia.org/wiki/PageRank}{https://en.m.wikipedia.org/wiki/PageRank}
	\end{description}
\end{ot}

% \begin{solution}
% \end{solution}
%%%%%%%%%%%%%%%

%%%%%%%%%%%%%%%
\begin{ot}[程序中的图]
	\begin{itemize}
		\item 简要介绍程序分析中常用各种图的基本概念。例如,调用图(Call Graph)、控制流图(Control-flow Graph)、程序依赖图(Program Dependence Graph)等。

		\item 参考资料:
		      \begin{itemize}
			      \item \href{https://en.m.wikipedia.org/wiki/Control-flow_graph}{https://en.m.wikipedia.org/wiki/Control-flow\_graph}
			      \item \href{https://en.m.wikipedia.org/wiki/Call_graph}{https://en.m.wikipedia.org/wiki/Call\_graph}
			      \item \href{https://dl.acm.org/doi/10.1145/24039.24041}{https://dl.acm.org/doi/10.1145/24039.24041}
		      \end{itemize}
	\end{itemize}

\end{ot}

% \begin{solution}
% \end{solution}
%%%%%%%%%%%%%%%


% \vspace{0.50cm}
%%%%%%%%%%%%%%%
% \begin{ot}[]
% 
%   \noindent 参考资料:
%   \begin{itemize}
%     \item 
%   \end{itemize}
% \end{ot}

% \begin{solution}
% \end{solution}
%%%%%%%%%%%%%%%

%%%%%%%%%%%%%%%%%%%%
% 如果没有需要订正的题目,可以把这部分删掉

% \begincorrection
%%%%%%%%%%%%%%%%%%%%

%%%%%%%%%%%%%%%%%%%%
% 如果没有反馈,可以把这部分删掉
\beginfb

% 你可以写
% ~\footnote{优先推荐 \href{problemoverflow.top}{ProblemOverflow}}:
% \begin{itemize}
%   \item 对课程及教师的建议与意见
%   \item 教材中不理解的内容
%   \item 希望深入了解的内容
%   \item $\cdots$
% \end{itemize}
%%%%%%%%%%%%%%%%%%%%
% \bibliography{2-5-solving-recurrence}
% \bibliographystyle{plainnat}
%%%%%%%%%%%%%%%%%%%%
\end{document}