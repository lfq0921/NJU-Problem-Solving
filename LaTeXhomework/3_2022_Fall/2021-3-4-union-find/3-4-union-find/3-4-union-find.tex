% 2-15-rb-tree.tex

%%%%%%%%%%%%%%%%%%%%
\documentclass[a4paper, justified]{tufte-handout}

\input{hw-preamble} % feel free to modify this file
%%%%%%%%%%%%%%%%%%%%
\title{第3-4讲: 用于动态等价关系的数据结构}
\me{林凡琪}{211240042}{}{}
\date{\zhtoday} % or like 2019年9月13日
%%%%%%%%%%%%%%%%%%%%
\begin{document}
\maketitle
%%%%%%%%%%%%%%%%%%%%
\noplagiarism % always keep this line
%%%%%%%%%%%%%%%%%%%%
\begin{abstract}
  % \begin{center}{\fcolorbox{blue}{yellow!60}{\parbox{0.65\textwidth}{\large 
  %   \begin{itemize}
  %     \item 
  %   \end{itemize}}}}
  % \end{center}
\end{abstract}
%%%%%%%%%%%%%%%%%%%%
\beginrequired

%%%%%%%%%%%%%%%
\begin{problem}[TC 21.1-2]
证明:$CONNECTED-COMPONENTS$处理完所有的边后,两个顶点在相同的连通分量中当且仅当它们在同一个集合中。
\end{problem}

\begin{solution}
  必要性:由$SAME-COMPONENT$可知,若两个元素在同一个set里,那么一定在相同的连通分量中。\\
  充分性:如果两个顶点$u$和$v$在同相同的连通分量中,则$\exists P{(u, v_1), (v_1, v_2),...(v_n, v)}$其中$v_n \in G.V (n\in N)$,又因为在$CONNECTED-COMPONENTS$中边的两端都被放进了同一个集合,所以u和v在同一个集合中。
\end{solution}
%%%%%%%%%%%%%%%

%%%%%%%%%%%%%%%
\begin{problem}[TC 21.1-3]
在$CONNECTED-COMPONENTS$作用于一个有$k$个连通分量的无向图$G=(V,E)$的过程中,$FIND-SET$需要调用多少次?$UNION$需要调用多少次?
\end{problem}

\begin{solution}
  $FIND-SET$被调用$|V|-k$次
  $UNION$被调用$k$次
\end{solution}
%%%%%%%%%%%%%%%

%%%%%%%%%%%%%%%
\begin{problem}[TC 21.2-1]
使用链表表示和加权合并启发式策略,写出$MAKE-SET$、$FIND-SET$、和$UNION$操作的伪代码。并指定你在集合对象和表对象中所使用的属性。
\end{problem}

\begin{solution}
  \begin{algorithm}[H]
    %caption{Disjoint_Set}
    %\label{alg:sum}
    \begin{algorithmic}[1]
      \Procedure{MAKE-SET}{$X$}
      \State Create a $S$
      \State $S.head = x$
      \State $S.tail = x$
      \State $x.next = NIL$
      \State $x.p = x$
      \State $S.size = 1$
      \EndProcedure
      \Procedure{UNION}{$u, v$}
      \State $S1 = u.set$
      \State $S2 = v.set$
      \If {S1.size >= S2.size}
      \State $z = S2.head$
      \While {$ z \neq NIL$}
      \State $z.p = S1.head$
      \State z=z.next
      \EndWhile
      \State $S1.tail = S2.tail$
      \If {$S1.size == S2.size$}
      \State $S1.size++$
      \EndIf
      \Else
      \State $UNION\{v, u\}$
      \EndIf
      \EndProcedure
      \Function{FIND-SET}{$x$}
      \State \Return $x.set.head$
      \EndFunction
    \end{algorithmic}
  \end{algorithm}



\end{solution}
%%%%%%%%%%%%%%%

%%%%%%%%%%%%%%%
\begin{problem}[TC 21.2-3]
对定理21.1的整体证明进行改造,得到使用链表表示和加权合并启发式策略下的MAKE-SET和FIND-SET的摊还时间上界为$O(\lg n)$
\end{problem}

\begin{solution}
  在定理 21.1 的证明中,我们得出结论,$n$ text{UNION} 操作运行的时间最多为 $O(n lg n)$。这意味着每次最多花费$O(lg n)$的摊销时间。此外,由于在执行 text{MAKE-SET} 和 text{FIND-SET} 操作时只有恒定的实际工作量,并且这些操作都未用于抵消 text{UNION} 操作的成本,因此它们都具有 $O(1)$
\end{solution}
%%%%%%%%%%%%%%%

%%%%%%%%%%%%%%%
\begin{problem}[TC 21.2-6]
假设对UNION过程做一个简单的改动,在采用链表表示中拿掉让集合对象的tail指针总指向每个表的最后一个对象的要求。无论是使用还是不使用加权合并启发式策略,这个修改不应该改变UNION过程的渐近运行时间(提示:而不是把一个表链接到另一个表后面,将它们拼接在一起)
\end{problem}

\begin{solution}
  在UNION过程中,每次和tail有关的操作都只占用$O(1)$,其他的操作和tail无关,所以去掉tail相关操作并不影响UNION过程的渐近运行时间。
\end{solution}
%%%%%%%%%%%%%%%

%%%%%%%%%%%%%%%
\begin{problem}[TC 21.3-1]
用按秩合并与路径压缩启发式策略的不相交集合森林重做练习21.2-2
\end{problem}

\begin{solution}
  $FIND-SET(x_2) = x_1$
  $FNID-SET(x_9) = x_1$
  \includegraphics[width = 0.8\textwidth]{tree.jpg}
\end{solution}
%%%%%%%%%%%%%%%

%%%%%%%%%%%%%%%
\begin{problem}[TC 21.3-2]
写出使用路径压缩的FIND-SET过程的非递归版本。
\end{problem}

\begin{solution}
  \begin{algorithm}[H]
    \begin{algorithmic}[1]
      \Function{FIND-SET}{$x$}
      \State $tmp = x$
      \While{$tmp.p \neq tmp$}
      \State $tmp = tmp.p$
      \EndWhile
      \While{$x \neq tmp$}
      \State $x.p = tmp$
      \State $x = x.p$
      \EndWhile
      \State \Return tmp
      \EndFunction
    \end{algorithmic}
  \end{algorithm}
\end{solution}
%%%%%%%%%%%%%%%

%%%%%%%%%%%%%%%
\begin{problem}[TC 21.3-3]
给出一个包含$m$个MAKE-SET、UNION和FIND-SET操作的序列(其中有n个是MAKE-SET操作),当仅使用按秩合并时,需要$\Omega(m\lg n)$的时间。
\end{problem}

\begin{solution}
  \begin{algorithm}[H]
    \begin{algorithmic}[1]
      \Procedure{TC 21.3-3}{}
      \For{$i \gets 1 to n$}
      \State $MAKE-SET(x[i])$
      \EndFor
      \For{$i \gets 1 to k$}
      \For {$j \gets 1 to n' - 2^{i=1} by 2^i$}
      \State $UNION(x_i, x_{i+2^{j-1}})$
      \EndFor
      \EndFor
      \For {$i\gets 1 to m$}
      \State $FIND-SET(x_1)$
      \EndFor
      \EndProcedure
    \end{algorithmic}
  \end{algorithm}
\end{solution}
%%%%%%%%%%%%%%%
%%%%%%%%%%%%%%%%%%%%
\beginoptional

%%%%%%%%%%%%%%%
\begin{problem}[TC Problem 21-1 (Off-line minimum)]
\end{problem}

\begin{solution}
\end{solution}
%%%%%%%%%%%%%%%

%%%%%%%%%%%%%%%%%%%%
\beginot
%%%%%%%%%%%%%%%
%%%%%%%%%%%%%%%
\begin{ot}[Off-line LCA (TC Problem 21.3)]


\end{ot}

% \begin{solution}
% \end{solution}
%%%%%%%%%%%%%%%

\begin{ot}[Partition refinement]
  参考资料:\href{https://en.wikipedia.org/wiki/Partition_refinement}{https://en.wikipedia.org/wiki/Partition\_refinement}
\end{ot}

% \begin{solution}
% \end{solution}
%%%%%%%%%%%%%%%




% \vspace{0.50cm}
%%%%%%%%%%%%%%%
% \begin{ot}[]
% 
%   \noindent 参考资料:
%   \begin{itemize}
%     \item 
%   \end{itemize}
% \end{ot}

% \begin{solution}
% \end{solution}
%%%%%%%%%%%%%%%

%%%%%%%%%%%%%%%%%%%%
% 如果没有需要订正的题目,可以把这部分删掉

% \begincorrection
%%%%%%%%%%%%%%%%%%%%

%%%%%%%%%%%%%%%%%%%%
% 如果没有反馈,可以把这部分删掉
\beginfb

% 你可以写
% ~\footnote{优先推荐 \href{problemoverflow.top}{ProblemOverflow}}:
% \begin{itemize}
%   \item 对课程及教师的建议与意见
%   \item 教材中不理解的内容
%   \item 希望深入了解的内容
%   \item $\cdots$
% \end{itemize}
%%%%%%%%%%%%%%%%%%%%
% \bibliography{2-5-solving-recurrence}
% \bibliographystyle{plainnat}
%%%%%%%%%%%%%%%%%%%%
\end{document}