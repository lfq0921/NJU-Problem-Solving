% 2-15-rb-tree.tex

%%%%%%%%%%%%%%%%%%%%
\documentclass[a4paper, justified]{tufte-handout}

\input{hw-preamble} % feel free to modify this file
%%%%%%%%%%%%%%%%%%%%
\title{第3-11讲: 旅行问题}
\me{林凡琪}{211240042}{}{}
\date{\zhtoday} % or like 2019年9月13日
%%%%%%%%%%%%%%%%%%%%
\begin{document}
\maketitle
%%%%%%%%%%%%%%%%%%%%
\noplagiarism % always keep this line
%%%%%%%%%%%%%%%%%%%%
\begin{abstract}
  % \begin{center}{\fcolorbox{blue}{yellow!60}{\parbox{0.65\textwidth}{\large 
  %   \begin{itemize}
  %     \item 
  %   \end{itemize}}}}
  % \end{center}
\end{abstract}
%%%%%%%%%%%%%%%%%%%%
\beginrequired

%%%%%%%%%%%%%%%
\begin{problem}[CZ 6.4]
\end{problem}

\begin{solution}
  (a)\\
  \begin{figure}[htbp]
    \centering
    \includegraphics[width = 0.30\linewidth]{figs/b.jpg}
  \end{figure}

  (b)\\
  \begin{figure}[htbp]
    \centering
    \includegraphics[width = 0.30\linewidth]{figs/c}
  \end{figure}

  (c)\\
  \newpage
  \begin{figure}[htbp]
    \centering
    \includegraphics[width = 0.30\linewidth]{figs/e}
  \end{figure}

  (d)\\
  \begin{figure}[htbp]
    \centering
    \includegraphics[width = 0.30\linewidth]{figs/d}
  \end{figure}

  (e)\\
  \begin{figure}[htbp]
    \centering
    \includegraphics[width = 0.30\linewidth]{figs/f}
  \end{figure}

\end{solution}
%%%%%%%%%%%%%%%

%%%%%%%%%%%%%%%
\begin{problem}[CZ 6.6]
\end{problem}

\begin{solution}
  If $n$ is connected to $k$ regular graph $G$ is not a Euler chart, it is known that $k$ is odd, and it can be known that $n$ can only be even. \\
  Then in $\overline {G}$, the degree of each point is $n-k-1$, which is an even number. \\
  Therefore, if $\overline {G}$ is connected, it is a regular graph.
\end{solution}
%%%%%%%%%%%%%%%

%%%%%%%%%%%%%%%
\begin{problem}[CZ 6.10]
\end{problem}

\begin{solution}
  $(i)$ When G is Hamiltonian: \\
  $\forall x, y \in V(G) \land (x,y) \notin E(G)$\\
  $\rightarrow deg(x)+ deg(y) \geq 6+ 6 = 12 \geq 10$\\
  \\
  $(ii)$ When G-v is Hamiltonian: \\
  $\forall x, y \in V(G-v) \land (x,y) \notin E(G-v)$ \\
  $\rightarrow deg(x)+ deg(y) \geq 5 + 5 = 10 \geq 9$\\
  \\
  $(iii)$ When G-v-u is Hamiltonian: \\
  $\forall x, y \in V(G-v-u)\land (x,y) \notin E(G-v-u)$\\
  $\rightarrow deg(x)+ deg(y) \geq 4 + 4 \geq 8$
\end{solution}
%%%%%%%%%%%%%%%

%%%%%%%%%%%%%%%
\begin{problem}[CZ 6.12]
\end{problem}

\begin{solution}
  (a)
  The nodes in $G+H$, if originally in $G$, is $14$ in the new figure; If it was originally in $H$, it is $16$ in the new figure; So it is $Eulerian$\\
  (b)
  For any two nonadjacent nodes $u,v$ in $G+H$, there is $deg(u)+deg(v)\geq 14+14\geq23$. Hence it is $ Hamiltonian$
\end{solution}
%%%%%%%%%%%%%%%

%%%%%%%%%%%%%%%
\begin{problem}[CZ 6.20]
\end{problem}

\begin{solution}
  (a)
  Proof by contradiction.\\
  Suppose there is a secant point $S$ in $G$, and from the inscription, we can see that there is a Hamiltonian path starting with $S$, let the path be $S, x_1, x_2,.., T$. Then after deleting $S$, there is still a path $x_1, x_2,..,T$, so that the remaining points are connected. Therefore, $S$ is not a cut point, which contradicts the assumption.\\
  (b)
  A counterexample\\
  \begin{figure}[htbp]
    \centering
    \includegraphics[width = 0.30\linewidth]{figs/a.jpg}
  \end{figure}
\end{solution}
%%%%%%%%%%%%%%%



%%%%%%%%%%%%%%%%%%%%
\beginot
%%%%%%%%%%%%%%%
%%%%%%%%%%%%%%%
\begin{ot}[竞赛图]
  底图是完全图的有向图称为竞赛图。请证明:竞赛图一定含有有向哈密尔顿通路


\end{ot}

% \begin{solution}
% \end{solution}
%%%%%%%%%%%%%%%

\begin{ot}[循环赛排名]
  请你给出一种合理的循环赛排名方法


\end{ot}

% \begin{solution}
% \end{solution}
%%%%%%%%%%%%%%%




% \vspace{0.50cm}
%%%%%%%%%%%%%%%
% \begin{ot}[]
% 
%   \noindent 参考资料:
%   \begin{itemize}
%     \item 
%   \end{itemize}
% \end{ot}

% \begin{solution}
% \end{solution}
%%%%%%%%%%%%%%%

%%%%%%%%%%%%%%%%%%%%
% 如果没有需要订正的题目,可以把这部分删掉

% \begincorrection
%%%%%%%%%%%%%%%%%%%%

%%%%%%%%%%%%%%%%%%%%
% 如果没有反馈,可以把这部分删掉
\beginfb

% 你可以写
% ~\footnote{优先推荐 \href{problemoverflow.top}{ProblemOverflow}}:
% \begin{itemize}
%   \item 对课程及教师的建议与意见
%   \item 教材中不理解的内容
%   \item 希望深入了解的内容
%   \item $\cdots$
% \end{itemize}
%%%%%%%%%%%%%%%%%%%%
% \bibliography{2-5-solving-recurrence}
% \bibliographystyle{plainnat}
%%%%%%%%%%%%%%%%%%%%
\end{document}