% 2-3-counting.tex

%%%%%%%%%%%%%%%%%%%%
\documentclass[a4paper, justified]{tufte-handout}

\input{hw-preamble} % feel free to modify this file
%%%%%%%%%%%%%%%%%%%%
\title{第3讲: 组合与计数}
\me{ 林凡琪}{211240042}{}{}
\date{\zhtoday} % or like 2019年9月13日
%%%%%%%%%%%%%%%%%%%%
\begin{document}
\maketitle
%%%%%%%%%%%%%%%%%%%%
\noplagiarism % always keep this line
%%%%%%%%%%%%%%%%%%%%
\begin{abstract}
  \mfig{width = 1.00\textwidth}{figs/einstein-count}
  % \mfigcap{width = 0.85\textwidth}{figs/George-Boole}{George Boole}
  % \begin{center}{\fcolorbox{blue}{yellow!60}{\parbox{0.65\textwidth}{\large 
  %   \begin{itemize}
  %     \item 
  %   \end{itemize}}}}
  % \end{center}
\end{abstract}
%%%%%%%%%%%%%%%%%%%%
\beginrequired

%%%%%%%%%%%%%%%
\begin{problem}[CS 1.2-1]
In how many ways can we pass out k distinct pieces of fruit to n children (with no restriction on how many pieces of fruit a child may get)?
\end{problem}

\begin{solution}
  According to the method learned in the high school, we can get that:\\
  $C(k + 1, n - 1) = \frac{(k + 1)!}{(n - 1)! * (k + 1 - (n - 1))!}$
\end{solution}
%%%%%%%%%%%%%%%

%%%%%%%%%%%%%%%
\begin{problem}[CS 1.2-5]
Assuming k ≤ n, in how many ways can we pass out k distinct pieces of fruit to n children if each child may get at most one piece? What if k>n? Assume for both questions that we pass out all the fruit.
\end{problem}

\begin{solution}
  (a)$k <= n$\\
  There are at most k children who can get one piece of fruit. So we should choose k children from n children. And for that the fruit pieces are distinct, so in different cases even the chosen children are the same, the kind of fruit pieces are still different.\\
  $C(n, k) * A(k, k) = \frac{n!}{k!(n - k)!} * k! = \frac{n!}{(n - k)!} = n * (n - 1) * ... * (n - k + 1)$\\
  So there are $\frac{n!}{(n - k)!}$ways we can pass out k distinct pieces of fruit to n children if each child may get at most one piece.\\
  (b)$k > n$\\
  There is no way to give out all the fruit, when each child may get at most one piece. So the answer is 0.
\end{solution}
%%%%%%%%%%%%%%%

%%%%%%%%%%%%%%%
\begin{problem}[CS 1.2-6]
Assuming k ≤ n, in how many ways can we pass out k identical pieces of fruit to n children if each child may get at most one? What if k>n? Assume for both questions that we pass out all the fruit.
\end{problem}

\begin{solution}
  (a)$ k <= n$\\
  What we need to do is choose k children who can get the fruit piece from n children. So there are $C(k, n)$ ways to artribute the fruit pieces.\\
  (b)$k > n$\\
  If there can be no fruit piece left, then there is no way to pass out them. So accordingly, the answer is 0.\\
\end{solution}
%%%%%%%%%%%%%%%

%%%%%%%%%%%%%%%
\begin{problem}[CS 1.2-15]
A tennis club has 2n members. We want to pair up the members by twos for singles matches. In how many ways can we pair up all the members of the club? Suppose that in addition to specifying who plays whom, we also determine who serves first for each pairing. Now in how many ways can we specify our pairs?
\end{problem}

\begin{solution}
  (a)$\sum C_i^2, i = 2k(k = 1, 2, 3, ..., n)$\\
  (b)$A_{2n}^{2n} / A_n^n$
\end{solution}
%%%%%%%%%%%%%%%

%%%%%%%%%%%%%%%
\begin{problem}[CS 1.5-4]
Use multisets to determine the number of ways to pass out k identical apples to n children. Assume that a child may get more than one apple.
\end{problem}

\begin{solution}
  The number of k-element multisets chosen from an n-element set is$frac{(n + k - 1)!}{k!(n - 1)!} = \binom{n + k - 1}{k}$\\
  So there are $frac{(n + k - 1)!}{k!(n - 1)!} = \binom{n + k - 1}{k}$ ways to pass out the apples.
\end{solution}
%%%%%%%%%%%%%%%

%%%%%%%%%%%%%%%
\begin{problem}[CS 1.5-12]
A standard notation for the number of partitions of an n-element set into k classes is S(n, k). Because the empty family of subsets of the empty set is a partition of the empty set, S(0, 0) is 1. In addition, S(n, 0) is 0 for n > 0, because there are no partitions of a nonempty set into no parts. S(1, 1) is 1.\\
a. Explain why S(n, n) is 1 for all n > 0. Explain why S(n, 1) is 1 for all n > 0.\\
b. Explain why S(n, k) = S(n − 1, k − 1) + kS(n − 1, k) for $1 <k<n$.\\
c. Make a table like Table 1.1 that shows the values of S(n, k) for values of n and k ranging from 1 to 6.\\
\end{problem}

\begin{solution}
  a.\\
  (1)Why S(n, n) is 1 for all n > 0?\\
  There is an n-element set. We should attribute the n elements into n classes, and every class can't be empty. It's to say, there is at least 1 element in each class. So in the first step, we attribute 1 element to the n classes, then the n elements have been all passed out. So it is the only way to get S(n, n).\\
  It's like $\frac{n!}{n!} = 1$\\
  (2)Why S(n, 1) is 1 for all n > 0?\\
  There is n elements, and we need to artribute them into 1 class. That is, except from the class, there are no other class, where can contain any elements. So we should put all n elements into the only 1 class. That is like $C_n^n = 1$.\\
  b.\\
  We distribute the n elements into k classes. We choose an arbitary element which is noted as $a_1$. There are two cases.\\
  Case 1: $a_1$ is chosen as a separated class, then there are $n-1$ elements to be chosen into $k-1$ classes. So there are $S(n − 1, k − 1)$ possibilities.\\
  Case 2:$a_1$ is not chosen as a separated class. Then the class is different from other ones. So we have to multiply $k$ in the formula. Therefore, the answer is $k * S(n - 1, k)$.\\
  And the key to the original problem is $S(n, k)$, so we can get that $S(n, k) = S(n − 1, k − 1) + kS(n − 1, k)$ for $1 <k<n$.\\
  c.答案在上面
  \begin{table}[]
    \begin{tabular}{|l|l|l|l|l|l|l|}
      \hline
      k$\backslash$n & 1 & 2 & 3 & 4 & 5  & 6  \\ \hline
      1              & 1 & 1 & 1 & 1 & 1  & 1  \\ \hline
      2              & 0 & 1 & 3 & 7 & 15 & 31 \\ \hline
      3              & 0 & 0 & 1 & 6 & 25 & 90 \\ \hline
      4              & 0 & 0 & 0 & 1 & 11 & 69 \\ \hline
      5              & 0 & 0 & 0 & 0 & 1  & 16 \\ \hline
      6              & 0 & 0 & 0 & 0 & 0  & 1  \\ \hline
    \end{tabular}
  \end{table}
\end{solution}
%%%%%%%%%%%%%%%

%%%%%%%%%%%%%%%%%%%%
\beginoptional

%%%%%%%%%%%%%%%
\begin{problem}[Summation]
请计算如下代码段的返回值 $r$。

\input{algs/conundrum}
\end{problem}

\begin{solution}
  $\sum\frac{(n-i)(n-i-1)}{2}(i=2t<=n; t = 0,1,2,3...)$
\end{solution}
%%%%%%%%%%%%%%%

%%%%%%%%%%%%%%%%%%%%
\beginot

本周两个 OT 的目的是向大家介绍在算法分析中常用的数学基础。
阅读书籍~\cite{Book:GKP}:
%%%%%%%%%%%%%%%
\begin{ot}[Sums]
  第二章关于 ``Sums'' 的内容 (如前五节), 介绍你认为有用、有意思的求和技巧。
\end{ot}

% \begin{solution}
% \end{solution}
%%%%%%%%%%%%%%%
\vspace{0.50cm}
%%%%%%%%%%%%%%%
\begin{ot}[Binomial Coefficients]
  第五章关于 ``Binomial Coefficients'' 的内容 (如前两节或前三节), 介绍你认为有用、有意思的公式与技巧。
\end{ot}

% \begin{solution}
% \end{solution}
%%%%%%%%%%%%%%%

%%%%%%%%%%%%%%%%%%%%
% 如果没有需要订正的题目,可以把这部分删掉

% \begincorrection
%%%%%%%%%%%%%%%%%%%%

%%%%%%%%%%%%%%%%%%%%
% 如果没有反馈,可以把这部分删掉
\beginfb

% 你可以写
% ~\footnote{优先推荐 \href{problemoverflow.top}{ProblemOverflow}}:
% \begin{itemize}
%   \item 对课程及教师的建议与意见
%   \item 教材中不理解的内容
%   \item 希望深入了解的内容
%   \item $\cdots$
% \end{itemize}
%%%%%%%%%%%%%%%%%%%%
\bibliography{counting}
\bibliographystyle{plainnat}
%%%%%%%%%%%%%%%%%%%%
\end{document}