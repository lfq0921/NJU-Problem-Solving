% 2-13-bst.tex

%%%%%%%%%%%%%%%%%%%%
\documentclass[a4paper, justified]{tufte-handout}

\input{hw-preamble} % feel free to modify this file
%%%%%%%%%%%%%%%%%%%%
\title{第13讲: 搜索树}
\me{林凡琪 }{ 211240042}{}{}
\date{\zhtoday} % or like 2019年9月13日
%%%%%%%%%%%%%%%%%%%%
\begin{document}
\maketitle
%%%%%%%%%%%%%%%%%%%%
\noplagiarism % always keep this line
%%%%%%%%%%%%%%%%%%%%
\begin{abstract}
  % \begin{center}{\fcolorbox{blue}{yellow!60}{\parbox{0.65\textwidth}{\large 
  %   \begin{itemize}
  %     \item 
  %   \end{itemize}}}}
  % \end{center}
\end{abstract}
%%%%%%%%%%%%%%%%%%%%
\beginrequired

%%%%%%%%%%%%%%%
\begin{problem}[TC 12.1-5]
Argue that since sorting $n$ elements takes $\Omega(n\lg n)$ time in the worst case in the comparison model, any comparison-based algorithm for constructing a binary search tree from an arbitrary list of $n$ elements takes $\Omega(n\lg n)$ time in the worst case.
\end{problem}

\begin{solution}
  反证法:假设我们可以使用少于 $Omega(nlg n)$ 时间通过基于比较的算法构造二叉搜索树,因为有序树遍历是 $Theta(n)$,那么我们可以在小于 $Omega(nlg n)$ 的时间内获得排序的元素,这与$n$ 元素进行排序在最坏情况下需要 $Omega(nlg n)$ 时间的事实相矛盾。
\end{solution}
%%%%%%%%%%%%%%%

%%%%%%%%%%%%%%%
\begin{problem}[TC 12.2-9]
Let $T$ be a binary search tree whose keys are distinct, let $x$ be a leaf node, and let $y$ be its parent. Show that $y.key$ is either the smallest key in $T$ larger than $x.key$ or the largest key in $T$ smaller than $x.key$.
\end{problem}

\begin{solution}
  如果 $x = y.left$,则在 $x$ 上调用后续函数将导致 while 循环没有迭代,因此将返回 $y$。\\
  如果 $x = y.right$,则用于调用前置任务(参见练习 3)的 while 循环将不运行,因此将返回 $y$。
\end{solution}
%%%%%%%%%%%%%%%

%%%%%%%%%%%%%%%
\begin{problem}[TC 12.3-5]
Suppose that instead of each node $x$ keeping the attribute $x.p$, pointing to $x$'s parent, it keeps $x.succ$, pointing to $x$'s successor. Give pseudocode for $\text{SEARCH}$, $\text{INSERT}$, and $\text{DELETE}$ on a binary search tree $T$ using this representation. These procedures should operate in time $O(h)$, where $h$ is the height of the tree $T$. ($\textit{Hint:}$ You may wish to implement a subroutine that returns the parent of a node.)
\end{problem}

\begin{solution}
  我们不需要更改 $text{SEARCH}$,但是必须实现 $text{PARENT}$.
  \begin{algorithm}[H]
    \begin{algorithmic}
      \Function{PARENT}{$T,x$}
      \If {$x == T.root$}
      \State \Return $NIL$
      \EndIf
      \State $y =  TREE-MAXIMUM(x).succ$
      \If {$y==NIL$}
      \State $y = T.root$
      \Else
      \If {$y.left==x$}
      \State \Return $y$
      \EndIf
      \State $y=y.left$
      \EndIf
      \While {$y.right != x$}
      \State $y = y.right$
      \EndWhile
      \State \Return $y$
      \EndFunction
      \Procedure{INSERT}{T,z}
      \State $y = NIL$
      \State $x = T.root$
      \State $pred = NIL$
      \While {$x \neq NIL$}
      \State $y = x$
      \If {$z.key < x.key$}
      \State $x = x.left$
      \Else
      \State $pred = x$
      \State $x = x.right$
      \EndIf
      \EndWhile
      \If {$y == NIL$}
      \State $T.root = z$
      \State $z.succ = NIL$
      \ElsIf {$z.key <y.key$}
      \State $y.left = z$
      \State $z.succ = y$
      \If {$pred \neq NIL$}
      \State $pred.succc = z$
      \EndIf
      \Else
      \State $y.right = z$
      \State $z.succ = y.succ$
      \State $y.succ = z$
      \EndIf
      \EndProcedure
    \end{algorithmic}
  \end{algorithm}
  We modify $\text{TRANSPLANT}$ a bit since we no longer have to keep the pointer of $p$.
  \begin{algorithm}
    \begin{algorithmic}
      \Procedure{TRANSPLANT}{$T,u,v$}
      \State $p = PARENT(T,u)$
      \If {$p==NIL$}
      \State $T.root = v$
      \ElsIf $u == p.left$
      \State $p.left = v$
      \Else \State $p.right = v$
      \EndIf
      \EndProcedure
    \end{algorithmic}
  \end{algorithm}
  此外,我们必须实现 $\text{TREE-PREDECESSOR}$,这有助于我们在 $\text{DELETE}$ 的第 2 行轻松找到predecessor。
  \begin{algorithm}
    \begin{algorithmic}
      \Function {TREE-PREDECESSOR}{$T, x$}
      \If {$x.left != NIL$}
      \State \Return $TREE-MAXIMUM(x.left)$
      \EndIf
      \State $y = T.root$
      \State $pred = NIL$
      \While {$y \neq NIL$}
      \If {$y.key == x.key$}
      \State $break$
      \EndIf
      \If {$y.key < x.key$}
      \State $pred = y$
      \State $y.right$
      \Else \State $y = y.left$
      \EndIf
      \EndWhile
      \State \Return $pred$
      \EndFunction
      \Procedure{DELETE}{$T,z$}
      \State $pred = TREE-PREDECESSOR(T, z)$
      \State $pred.succ = z.succ$
      \If {$z.left== NIL$}
      \State $TRANSPLANT(T, z, z.right)$
      \ElsIf {$z.right == NIL$}
      \State $TRANSPLANT(T, z, z.left)$
      \Else
      \State $ y = TREE-MIMIMUM(z.right)$
      \If {$PARENT(T, y) != z$}
      \State $TRANSPLANT(T, y, y.right)$
      \State $y.right = z.right$
      \EndIf
      \State $TRANSPLANT(T, z, y)$
      \State $y.left = z.left$
      \EndIf
      \EndProcedure
    \end{algorithmic}
  \end{algorithm}


\end{solution}
%%%%%%%%%%%%%%%

%%%%%%%%%%%%%%%%%%%%%
%\beginoptional
%
%%%%%%%%%%%%%%%%
%\begin{problem}[]
%\end{problem}
%
%\begin{solution}
%\end{solution}
%%%%%%%%%%%%%%%%

%%%%%%%%%%%%%%%%%%%%
\beginot

%%%%%%%%%%%%%%%
\begin{ot}[Interval tree]
  介绍 Interval tree

  \noindent 参考资料:
  \begin{itemize}
    \item \href{https://en.wikipedia.org/wiki/Interval_tree}{Interval tree}
  \end{itemize}
\end{ot}
\begin{ot}[Splay tree]
  介绍 Splay tree (平摊分析部分可仅介绍基本思想)。

  \noindent 参考资料:
  \begin{itemize}
    \item \href{https://en.wikipedia.org/wiki/Splay\_tree}{Splay tree}
    \item \href{https://www.cs.princeton.edu/courses/archive/spring06/cos423/Handouts/self\%20adjusting.pdf}{Self-Adjusting Binary Search Trees (Rober Tarjan, JACM85).pdf}
  \end{itemize}
\end{ot}

% \begin{solution}
% \end{solution}
%%%%%%%%%%%%%%%
% \vspace{0.50cm}
%%%%%%%%%%%%%%%
% \begin{ot}[]
% 
%   \noindent 参考资料:
%   \begin{itemize}
%     \item 
%   \end{itemize}
% \end{ot}

% \begin{solution}
% \end{solution}
%%%%%%%%%%%%%%%

%%%%%%%%%%%%%%%%%%%%
% 如果没有需要订正的题目,可以把这部分删掉

% \begincorrection
%%%%%%%%%%%%%%%%%%%%

%%%%%%%%%%%%%%%%%%%%
% 如果没有反馈,可以把这部分删掉
\beginfb

% 你可以写
% ~\footnote{优先推荐 \href{problemoverflow.top}{ProblemOverflow}}:
% \begin{itemize}
%   \item 对课程及教师的建议与意见
%   \item 教材中不理解的内容
%   \item 希望深入了解的内容
%   \item $\cdots$
% \end{itemize}
%%%%%%%%%%%%%%%%%%%%
% \bibliography{2-5-solving-recurrence}
% \bibliographystyle{plainnat}
%%%%%%%%%%%%%%%%%%%%
\end{document}