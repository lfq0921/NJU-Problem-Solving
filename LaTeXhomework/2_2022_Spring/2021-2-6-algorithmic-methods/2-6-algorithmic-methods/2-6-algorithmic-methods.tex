% 2-6-algorithmic-methods.tex

%%%%%%%%%%%%%%%%%%%%
\documentclass[a4paper, justified]{tufte-handout}

\input{hw-preamble} % feel free to modify this file
%%%%%%%%%%%%%%%%%%%%
\title{第6讲: 算法方法}
\me{林凡琪}{211240042}{}{}
\date{\zhtoday} % or like 2019年9月13日
%%%%%%%%%%%%%%%%%%%%
\begin{document}
\maketitle
%%%%%%%%%%%%%%%%%%%%
\noplagiarism % always keep this line
%%%%%%%%%%%%%%%%%%%%
\begin{abstract}
  % \begin{center}{\fcolorbox{blue}{yellow!60}{\parbox{0.65\textwidth}{\large 
  %   \begin{itemize}
  %     \item 
  %   \end{itemize}}}}
  % \end{center}
\end{abstract}
%%%%%%%%%%%%%%%%%%%%
\beginrequired

%%%%%%%%%%%%%%%
\begin{problem}[DH 4-8]
Prove that the maximal distance between any two points on a polygon occurs between two of the vertices.
\end{problem}

\begin{solution}
  Assume that the maximum distance between any two points on a polygon will not occur between two vertices\\
  The maximum distance between a and b of a polygon with y sides is assumed to have x sides.\\
  We know that the maximum distance between any two points on a polygon will not occur between two vertices.\\
  So we can conclude that x$\to$b can form a polygon with (y - x + 1) sides.\\
  Then the problem can be described as the minimum distance of two points is not a straight line, we know that this does not work in all cases.\\
  Therefore, we can show that the maximum distance between any two points on a polygon occurs between two vertices.
\end{solution}
%%%%%%%%%%%%%%%

%%%%%%%%%%%%%%%
\begin{problem}[DH 4-9]
Write a program implementing the maximal polygonal distance algorithm
\end{problem}

\begin{solution}
  \noindent
  \begin{algorithm}
    \begin{algorithmic}[1]
      \Procedure{max}{$P = \{p_1,...,p_n\}$}
      \State p0 = pn;
      \State q = NEXT[p];
      \While {Area(p, NEXT[p], NEXT[q]) > Area(p, NEXT[p], q)}
      \State q = NEXT[q];
      \State q0 = q;
      \While {q != p0}
      \State p = NEXT[p];
      \State print(p, q);
      \While {Area(p, NEXT[p], NEXT[q]) > Area(p, NEXT[p], q)}
      \State q = NEXT[q];
      \If{(p, q) != (q0, p0)}
      \State print(p, q);
      \Else return;
      \EndIf
      \EndWhile
      \If{Area(p, NEXT[p], NEXT[q]) = Area(p, NEXT[p], q)}
      \State \If{(p,q)!=(q0,p0)}
      \State print(p, NEXT[q]);
      \Else print(NEXT[p],q)
      \EndIf
      \EndIf
      \EndWhile
      \EndWhile
      \EndProcedure
    \end{algorithmic}
  \end{algorithm}
  The input is a polygon $P = \{p_1,...,p_n\}$.
  致谢csdn博主(伪代码在下一页)
\end{solution}
%%%%%%%%%%%%%%%

%%%%%%%%%%%%%%%
\begin{problem}[DH 4-12]
Write high-level pseudocode of the greedy algorithm described in the text
for finding a minimal spanning tree.
\end{problem}

\begin{solution}
  \noindent
  \begin{algorithm}
    \begin{algorithmic}[2]
      \Procedure{greedy}{$C, Q[], W[], P[]$}
      \State profit$\gets$0
      \While{C $\neq$0}
      \State max$\gets$0
      \State I$\gets$0
      \For{i from 1 to N}
      \If{P[i] / W[i] > max and Q[i] $\neq$ 0}
      \State max = P[i] / W[i]
      \State I = i
      \EndIf
      \EndFor
      \State C = C - W[I]
      \State profit = profit + P[I]
      \EndWhile
      \State \Return profit
      \EndProcedure
    \end{algorithmic}
  \end{algorithm}
\end{solution}
%%%%%%%%%%%%%%%

%%%%%%%%%%%%%%%
\begin{problem}[DH 4-13]
\end{problem}

\begin{solution}
  (a)\\
  \noindent
  \begin{algorithm}
    \begin{algorithmic}[1]
      \Procedure{DP}{$C, N, Q[], w[], p[]$}
      \State dp[0,...,C] = 0
      \For{$i \gets 1, N$}
      \For{$j \gets C, w[i]$}
      \For{$k \gets 0, min(q[i], j/w[i])$}
      \State dp[j] = max(dp[j], dp[j - k * w[i]] + k*p[i])
      \EndFor
      \EndFor
      \EndFor
      \State print(dp[C])
      \EndProcedure
    \end{algorithmic}
  \end{algorithm}
  (b)The maximal profit is 194.
\end{solution}
%%%%%%%%%%%%%%%

%%%%%%%%%%%%%%%%%%%%
\beginoptional

%%%%%%%%%%%%%%%
\begin{problem}[DH 4-10]
\end{problem}

\begin{solution}
\end{solution}
%%%%%%%%%%%%%%%

%%%%%%%%%%%%%%%%%%%%
\beginot

%%%%%%%%%%%%%%%
本周 OT 关注搜索技术。

\begin{ot}[Alpha–Beta Pruning]
  请介绍 Alpha-Beta 剪枝技术,包括概念、方法、应用 (比如在双人游戏中) 等。

  \noindent 参考资料:
  \begin{itemize}
    \item \href{https://en.wikipedia.org/wiki/Alpha\%E2\%80\%93beta\_pruning}{Alpha–beta pruning @ wiki}
    \item \href{https://github.com/hengxin/problem-solving-class-paperswelove/blob/master/2nd-semester/Knuth\%20(AI\%2C\%201975)\%20An\%20Analysis\%20of\%20Alpha-Beta\%20Pruning.pdf}{``An Analysis of Alpha-Beta Pruning'' @ AI'1975} (可选)
  \end{itemize}
\end{ot}

\begin{solution}
\end{solution}
%%%%%%%%%%%%%%%
% \vspace{0.50cm}
%%%%%%%%%%%%%%%
\begin{ot}[SAT Solver]
  请介绍 \href{https://en.wikipedia.org/wiki/SAT\_solver}{SAT} 的求解算法。

  \noindent 参考资料
  \begin{itemize}
    \item \href{https://en.wikipedia.org/wiki/SAT\_solver#Algorithms\_for\_solving\_SAT}{Solving SAT @ wiki}
    \item \href{https://en.wikipedia.org/wiki/DPLL\_algorithm}{DPLL algorithm @ wiki}
    \item \href{https://yurichev.com/writings/SAT\_SMT\_by\_example.pdf}{Examples} (可选)
  \end{itemize}
\end{ot}

% \begin{solution}
% \end{solution}
%%%%%%%%%%%%%%%

%%%%%%%%%%%%%%%%%%%%
% 如果没有需要订正的题目,可以把这部分删掉

% \begincorrection
%%%%%%%%%%%%%%%%%%%%

%%%%%%%%%%%%%%%%%%%%
% 如果没有反馈,可以把这部分删掉
\beginfb

% 你可以写
% ~\footnote{优先推荐 \href{problemoverflow.top}{ProblemOverflow}}:
% \begin{itemize}
%   \item 对课程及教师的建议与意见
%   \item 教材中不理解的内容
%   \item 希望深入了解的内容
%   \item $\cdots$
% \end{itemize}
%%%%%%%%%%%%%%%%%%%%
\bibliography{2-5-solving-recurrence}
\bibliographystyle{plainnat}
%%%%%%%%%%%%%%%%%%%%
\end{document}