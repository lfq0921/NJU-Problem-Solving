% 2-12-hashing.tex

%%%%%%%%%%%%%%%%%%%%
\documentclass[a4paper, justified]{tufte-handout}

\input{hw-preamble} % feel free to modify this file
%%%%%%%%%%%%%%%%%%%%
\title{第12讲: Hashing 方法}
\me{ 林凡琪}{211240042 }{}{}
\date{\zhtoday} % or like 2019年9月13日
%%%%%%%%%%%%%%%%%%%%
\begin{document}
\maketitle
%%%%%%%%%%%%%%%%%%%%
\noplagiarism % always keep this line
%%%%%%%%%%%%%%%%%%%%
\begin{abstract}
  % \begin{center}{\fcolorbox{blue}{yellow!60}{\parbox{0.65\textwidth}{\large 
  %   \begin{itemize}
  %     \item 
  %   \end{itemize}}}}
  % \end{center}
\end{abstract}
%%%%%%%%%%%%%%%%%%%%
\beginrequired

%%%%%%%%%%%%%%%
\begin{problem}[CS 5.5-8 ($a, b, c$)]
Suppose you hash n items into k locations.\\
a. What is the probability that all n items hash to different
locations?\\
b. What is the probability that the ith item is the first collision?\\
c. What is the expected number of items you hash until the first
collision?
\end{problem}

\begin{solution}
  (a)如果k<n,那么概率为0\\
  否则如果k>=n,那么$P=\frac{k^{\underline{n}}}{k^n}$\\
  (b)如果$i-1>k$那么概率为0;\\
  否则如果$i-1\leq k$,那么P[i-1]=$\frac{k^{\underline{i-1}}}{k^{i-1}}$\\
  所以P[i]=$\frac{k^{\underline{i-1}}}{k^{i-1}} \cdot \frac{i-1}{k}$\\
  (c)$E[X]=\sum _{i=1}^n(i-1)\cdot \frac{k^{\underline{i-1}}}{k^{i-1}}\cdot \frac{i-1}{k}$
\end{solution}
%%%%%%%%%%%%%%%


%%%%%%%%%%%%%%%
\begin{problem}[TC 11.2-3]
\end{problem}

\begin{solution}
  成功搜索: 没有区别, $\Theta(1 + \alpha)Θ(1+α)$.\\
  不成功搜索: 更快但仍然是 $\Theta(1 + \alpha)Θ(1+α)$.\\
  插入: 和成功搜索一样, $\Theta(1 + \alpha)Θ(1+α)$.\\
  删除: 如果我们使用双向链表,则与以前相同, $\Theta(1)Θ(1)$.
\end{solution}
%%%%%%%%%%%%%%%

%%%%%%%%%%%%%%%
\begin{problem}[TC 11.3-3]
\end{problem}

\begin{solution}
  我们将证明每个字符串散列到它的数字之和 $\mod 2^p - 1$.\\
  我们将通过对字符串长度的归纳来做到这一点。\\
  Base case\\
  假设字符串是单个字符,那么该字符的值就是 k 的值,然后取 $\mod$ m。\\
  Inductive step\\
  令 $w = w_1w_2$​ 其中 $|w_1| \ge 1$ 和 $|w_2| = 1$。假设 $h(w_1) = k_1$​。 那么,$h(w) = h(w_1)2^p + h(w_2) \mod 2^p − 1 = h(w_1) + h(w_2) \mod 2^p − 1$。所以,由于 $ h(w_1)$ 是除了最后一个数字 $\mod m$ 之外的所有数字的总和,我们添加最后一个数字 $\mod m$,就能得到想要的结论。
\end{solution}
%%%%%%%%%%%%%%%

%%%%%%%%%%%%%%%
\begin{problem}[TC 11.4-3]
\end{problem}

\begin{solution}
  $\alpha =3/4$,\\
  不成功: $\frac{1}{1 - \frac{3}{4}} = 4$ probes,
  成功: $\frac{1}{\frac{3}{4}} \ln\frac{1}{1-\frac{3}{4}} \approx 1.848$ probes.\\
  $\alpha = 7 / 8$\\
  不成功: $\frac{1}{1 - \frac{7}{8}} = 8 $ probes,\\
  成功: $\frac{1}{\frac{7}{8}} \ln\frac{1}{1 - \frac{7}{8}} \approx 2.377$ probes.\\
\end{solution}
%%%%%%%%%%%%%%%

%%%%%%%%%%%%%%%
\begin{problem}[TC Problem 11-1]
\end{problem}

\begin{solution}
  (a)$X_i$:probe 的数量\\
  我们想要计算$Pr(X_i > k)$\\
  $A_j$是第j个probe发生在已占用的slot\\
  $X_i>k=A_1\cap A_2 \cap ... \cap A_k$\\
  $$
    \begin{aligned}
      Pr(X_i>k) & = Pr(A_1\cap A_2 \cap ... \cap A_k)                                                                    \\
                & =Pr(A_1)\cdot Pr(A_2|A_1)\cdot Pr(A_3|A_1\cap A_2)...Pr(A_k|A_1\cap A_2 \cap ... \cap A_{k-1})         \\
                & = \frac{n}{m}\cdot \frac{n-1}{m-1} ... \frac{n-k+1}{m-k+1}\leq (\frac{n}{m})^k = \alpha ^k \leq 2^{-k}
    \end{aligned}
  $$
  (b)$(X>2 \lg n) = \bigvee ^n _{i=1}(X_i > 2 \lg n)$\\
  $Pr(X>2 \lg n) = Pr(\bigvee ^n _{i=1}(X_i > 2 \lg n))\leq n \cdot O(\frac{1}{n^2})=O(1/n)$\\
  (c)$$
    \begin{aligned}
      E[X] & = \sum^n_{i=1} i\cdot Pr(X = i)                                                                                              \\
           & =\sum^{\lceil 2 \lg n\rceil }_{i=1}i\cdot Pr(X = i)+\sum^n_{i = \lceil 2 \lg n\rceil + 1}Pr(X = i)                           \\
           & \leq \lceil 2 \lg n\rceil \sum^{\lceil 2 \lg n\rceil }_{i=1}i\cdot Pr(X = i)+n\sum^n_{i = \lceil 2 \lg n\rceil + 1}Pr(X = i) \\
           & =\lceil 2 \lg n\rceil Pr[X\leq \lceil 2 \lg n\rceil]+nPr(X>\lceil 2 \lg n\rceil)                                             \\
           & \leq \lceil 2 \lg n\rceil + nO(1/n)                                                                                          \\
           & = O(\lg n)
    \end{aligned}
  $$
\end{solution}
%%%%%%%%%%%%%%%

%%%%%%%%%%%%%%%
\begin{problem}[TC Problem 11-2]
\end{problem}

\begin{solution}
  (a) 一个特定的密钥以概率 $\frac{1}{n}$ 散列到一个特定的槽,假设我们选择一组特定的 k 个密钥。这些 k 个密钥被插入到相关槽中的概率以及所有其他密钥的概率 插入其他地方是 $(^1_n)^k$$(^1_{n - 1})^{n - k}$
        因为有 $(^n_k$) 种方法来选择我们的 k 个密钥,所以我们得到
      $Q_k$ = $(^1_n)^k$$(^1_{n - 1})^{n - k}$$(^n_k$)\\
        b.对于 i = 1, 2, ...,n,令 $X_i$ 是一个随机变量,表示散列到插槽 i 的键的数量,令 $A_i$ 是 $X_i$ = k 的事件,即 k 密钥哈希到插槽 i。我们知道 Pr$\{$A$\}$ = $Q_k$,$\therefore$\\
      $P_k$ = Pr$\{$M = k$\}$

        = Pr$\{$($_{1\le i\le n}^{max}X_i$) = k$\}$

        = Pr$\{$there exists i such that $X_i$ = k and that $X_i$$\le$k for i = 1,2,…,n$\}$

  $\le$Pr$\{$there exists i such that $X_i$ = k$\}$

      = Pr$\{$$A_1\bigcup A_2\bigcup …\bigcup A_n$$\}$

  $\le$Pr$\{$$A_1$$\}$ + Pr$\{$$A_2$$\}$ + … + Pr$\{$$A_n$$\}$

      = n$Q_k$
      c.首先,1-$\frac{1}{n}$ < 1,这意味着 (1 - $\frac{1}{n}$)$^{n - k}$ < 1。其次,$\ frac{n!}{(n - k)!}$ = n(n - 1)(n - 2)...(n - k + 1) < $n^k$.使用这些事实和 k! > ($\frac{k}{e}$)$^k$,我们有

    $Q_k$ < $n^k$k!(n - k)!((1 - $\frac{1}{n}$)$^{n - k}$ < 1)

      <$k^k$(k! > ($\frac{k}{e}$)$^k$)

      d.当 n = 2 时,lglgn = 0,所以准确地说,我们需要假设 n$\ge$3

      对于任何 k,我们都有 $Q_k$ <$\frac{e^k}{k^k}$,特别是,这个不等式对 $k_0$ 成立。因此足以证明 $\frac{e_0^{k_0 }}{k_0^{k_0}}$,或 $n^3$ < $\frac{k_0^{k_0}}{e_0^{k_0}}$
  $\therefore$3lgn = lglgn(lgc + glgn - lglglgn - lge)

      3 < lglgn(lgc + glgn - lglglgn - lge)

      = c$(_{1 + lglgn - lglgn}^{lgc - lge lglglgn})$

      所以 x = c$(_{1 + lglgn - lglgn}^{lgc - lge lglglgn})$
      我们需要证明存在一个常数 c > 1 使得 3 <cx

      注意到 lim$_{n\to ∞}$x = 1,我们看到存在 $n_0$ 使得 x$\ge$$\frac{1}{2}$ 对于所有 n $\ge$$n_0 $,因此任何常数 c > 6 适用于 n $\ge$$n_0$\\
        我们处理较小的 n 值,特别是 3$\le$n < $n_0$-,如下所示。由于 n 被约束为整数,因此在 3 $\le$n < $n_0$。我们可以为每个这样的 n 值计算表达式 x,并确定 c 的值,对于所有 n 值,3 < cx。我们使用的 c 的最终值是 6 中的较大者,它适用于所有 n > $n_0$,我们为范围 3 $\le$n < $n_0$ 选择的 c 的最大值\\
        因此,我们知道 $Q_{k_0}$ < $\frac{1}{n^3}$

        选择 k = $k_0$ 给出 $P_{k_0}$$\le$n$Q_{k_0}$ < n, ($\frac{1}{n^3}$) = $\frac{1}{n ^2}$。\\
    $\therefore$$Q_k$ < $\frac{e^k}{k^k}$

  $\le$$\frac{e^{k_0}}{k^{k_0}}$

      < $\frac{1}{n^3}$ for k $\ge$$k_0$

        e.E[M] = $\sum$k * Pr$\{$M = k$\}$

  $\le$$\sum$$k_0$Pr$\{$M = k$\}$ + $\sum$n * Pr$\{$M = k$\}$

  $\le$$k_0$$\sum$Pr$\{$M = k$\}$ + n$\sum$Pr$\{$M = k$\}$

        = $k_0$ * Pr$\{$M$\le$$k_0$$\}$ + n * Pr$\{$M > $k_0$$\}$\\
      因为$k_0$ =$\frac{clgn}{lglgn}$

      Pr$\{$M > $k_0$$\}$ = $\sum$Pr$\{$M = k$\}$

        =$\sum$$P_k$

      <$\sum$$\frac{1}{n^2}$

        < n * ($\frac{1}{n^2}$)

        = $\frac{1}{n}$

  $\therefore$E[M]$\le$$k_0$ * 1 + n ($\frac{1}{n}$)

      = $k_0$ + 1

      = O($\frac{lgn}{lglgn}$)
\end{solution}
%%%%%%%%%%%%%%%

%%%%%%%%%%%%%%%%%%%%
\beginoptional

%%%%%%%%%%%%%%%
\begin{problem}[TC 11.2-6]
\end{problem}

\begin{solution}
\end{solution}
%%%%%%%%%%%%%%%

%%%%%%%%%%%%%%%%%%%%
\beginot

%%%%%%%%%%%%%%%
\begin{ot}[Perfect Hashing]
  介绍 Perfect hashing。

  \noindent 参考资料:
  \begin{itemize}
    \item Section 11.5 of CLRS
  \end{itemize}
\end{ot}

% \begin{solution}
% \end{solution}
%%%%%%%%%%%%%%%
% \vspace{0.50cm}
%%%%%%%%%%%%%%%
\begin{ot}[Bloom filter]
  介绍 Bloom filter。

  \noindent 参考资料:
  \begin{itemize}
    \item \href{https://en.wikipedia.org/wiki/Bloom\_filter}{Bloom filter @ wiki}
  \end{itemize}
\end{ot}

% \begin{solution}
% \end{solution}
%%%%%%%%%%%%%%%

%%%%%%%%%%%%%%%%%%%%
% 如果没有需要订正的题目,可以把这部分删掉

% \begincorrection
%%%%%%%%%%%%%%%%%%%%

%%%%%%%%%%%%%%%%%%%%
% 如果没有反馈,可以把这部分删掉
\beginfb

% 你可以写
% ~\footnote{优先推荐 \href{problemoverflow.top}{ProblemOverflow}}:
% \begin{itemize}
%   \item 对课程及教师的建议与意见
%   \item 教材中不理解的内容
%   \item 希望深入了解的内容
%   \item $\cdots$
% \end{itemize}
%%%%%%%%%%%%%%%%%%%%
% \bibliography{2-5-solving-recurrence}
% \bibliographystyle{plainnat}
%%%%%%%%%%%%%%%%%%%%
\end{document}