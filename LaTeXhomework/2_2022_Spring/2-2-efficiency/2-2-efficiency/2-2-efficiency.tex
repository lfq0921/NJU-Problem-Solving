% 2-2-efficiency.tex

%%%%%%%%%%%%%%%%%%%%
\documentclass[a4paper, justified]{tufte-handout}

\input{hw-preamble} % feel free to modify this file
%%%%%%%%%%%%%%%%%%%%
\title{第2讲: 算法的效率}
\me{林凡琪}{211240042}{}{}
\date{\zhtoday} % or like 2019年9月13日
%%%%%%%%%%%%%%%%%%%%
\begin{document}
\maketitle
%%%%%%%%%%%%%%%%%%%%
\noplagiarism % always keep this line
%%%%%%%%%%%%%%%%%%%%
\begin{abstract}
  % \mfigcap{width = 0.85\textwidth}{figs/George-Boole}{George Boole}
  % \begin{center}{\fcolorbox{blue}{yellow!60}{\parbox{0.65\textwidth}{\large 
  %   \begin{itemize}
  %     \item 
  %   \end{itemize}}}}
  % \end{center}
\end{abstract}
%%%%%%%%%%%%%%%%%%%%
\beginrequired

%%%%%%%%%%%%%%%
\begin{problem}[DH Problem 6.18: $\log_{m} n$]
\end{problem}

\begin{solution}
  \begin{algorithm}
    \caption{$\log_{m} n$}
    \label{alg:sum}
    \begin{algorithmic}[1]
      \Function{LG1}{m, n}
      \State $k = 0$
      \State $res = 1$
      \While{$res * m <= n$}
      \State $res = res * m$
      \State $k = k + 1$
      \EndWhile
      \State return $res$
      \EndFunction
    \end{algorithmic}
  \end{algorithm}

  Time complexity: $O (\log_m n)$\\
  Space complexity: $O(1)$\\
\end{solution}
%%%%%%%%%%%%%%%

%%%%%%%%%%%%%%%
\begin{problem}[DH Problem 6.19: $\log_{m} n$]
\end{problem}

\begin{solution}
  \begin{algorithm}
    \caption{$\log_{m} n$}
    \label{alg:sum}
    \begin{algorithmic}[1]
      \Function{LG2}{m, n}
      \State $k = 0$
      \State $a = 1$
      \While{$m * a <= n$}
      \State $k_1 = 1 $
      \State $b = m * m$
      \State $c = m * a$
      \While{$a * b <= n$}
      \State $c = a$
      \State $a = b * c$
      \State $k_1 = 2k_1$
      \State $b = b * b$
      \EndWhile
      \State a = c
      \State $k = k + k_1$
      \EndWhile
      \State return $k$
      \EndFunction
    \end{algorithmic}
  \end{algorithm}
  Time complexity:$O ((\log \log n)^2)$\\
  Space complexity: $O(1)$
\end{solution}


%%%%%%%%%%%%%%%

%%%%%%%%%%%%%%%
\begin{problem}[DH Problem 6.20 (a): $\log_{m} n$]
\end{problem}

\begin{solution}
  \begin{algorithm}
    \caption{LG2}
    \label{alg:sum}
    \begin{algorithmic}[1]
      \Procedure{LG2}{m,n}
      \State $res[0] = m$
      \State $count = 0$
      \While{$res[count] * res[count] <= n$}
      \State $res[count + 1] = res[count] * res[count]$
      \State $count = count + 1$
      \EndWhile
      \State $sum = res[count]$
      \State $ans = pow(2, count)$
      \While{$count >= 1$}
      \If {$sum * res[count - 1] <= n$ AND $sum * res[count - 1] * m * m > n$}
      \State $ans = ans + pow(2, count - 1)$
      \State $sum = sum * res[count - 1]$
      \EndIf
      \State $count = count - 1$
      \EndWhile
      \State return ans
      \EndProcedure
    \end{algorithmic}
  \end{algorithm}
  Time complexity:$O (\log \log n)$\\
  Space complexity: $O(\log n)$
\end{solution}
%%%%%%%%%%%%%%%

%%%%%%%%%%%%%%%
\begin{problem}[TC Exercise 3.1-6]
\end{problem}

\begin{solution}
  充分性:\\
  设该算法的运行时间为$T(n) = \Theta(n) \rightarrow \exists c_1, c_2, n_0, \forall n > n_0, c_1 n <= T(n) <= c_2 n$ \\
  所以$\exists c_1, n_0, \forall n > n_0, c_1 n <= T(n) \rightarrow T(n) = \Omega (n)$即最好情况运行时间是$\Omega (n)$\\
  同样,$\exists c_2, n_0, \forall n > n_0,T(n) <= c_2 n,\rightarrow T(n) = O(n)$即最坏情况运行时间为$O(n)$\\
  必要性:\\
  设该算法运行时间为$T(n) \in O(n)$并且$T(n) \in \Omega(n)$\\
  即$\exists c_1, n_0, \forall n > n_0, c_1 n <= T(n) \rightarrow T(n) = \Omega (n)$且$\exists c_2, n_0, \forall n > n_0,T(n) <= c_2 n,\rightarrow T(n) = O(n)$.\\
  综合两点可得$T(n) = \Theta(n) \rightarrow \exists c_1, c_2, n_0, \forall n > n_0, c_1 n <= T(n) <= c_2 n$\\
  综上,证毕.
\end{solution}
%%%%%%%%%%%%%%%

%%%%%%%%%%%%%%%
\begin{problem}[TC Exercise 3.1-7]
\end{problem}

\begin{solution}
  $o(g(n)) = \{f(n)|\lim_{n \to \infty} \frac{f(n)}{g(n)} = 0\}$\\
  $\omega (g(n)) = \{f(n) | \lim_{n \to \infty} \frac{f(n)}{g(n)} \rightarrow \infty\}$\\
  $\Rightarrow o(g(n)) \cap \omega (g(n)) = \{f(n) | \lim_{n \to \infty} \frac{f(n)}{g(n)} \rightarrow \infty, \lim_{n \to \infty} \frac{f(n)}{g(n)} = 0\}$\\
  显然,$0 \neq \infty$.所以一定不存在符合条件的f(n),所以$o(g(n)) \cap \omega (g(n))$为空集
\end{solution}
%%%%%%%%%%%%%%%

%%%%%%%%%%%%%%%
\begin{problem}[TC Problem 3-3 (a)]
\end{problem}

\begin{solution}

  答案被移到下面去了...
  \begin{table}[]
    \begin{tabular}{|l|l|}
      \hline
      等价类序号 & 函数                                          \\ \hline
      1          & $g_1 = 2^{2^{n + 1}}$                         \\ \hline
      2          & $g_2 = 2^{2^{n}}$                             \\ \hline
      3          & $g_3 = (n + 1)!$                              \\ \hline
      4          & $g_4 = n!$                                    \\ \hline
      5          & $g_5 = e^n$                                   \\ \hline
      6          & $g_6 = n2^n$                                  \\ \hline
      7          & $g_7 = 2 ^n$                                  \\ \hline
      8          & $g_8 = (\frac{3}{2})^n$                       \\ \hline
      9          & $g_9 = n^{\lg \lg n},g_10 = (\lg n) ^{\lg n}$ \\ \hline
      10         & $g_{11} = (\lg n)!$                           \\ \hline
      11         & $g_{12} = n^3$                                \\ \hline
      12         & $g_{13} = n^2,g_{14} = 4^{\lg n}$             \\ \hline
      13         & $g_15 = n\lg n, g_{16} = lg(n!)$              \\ \hline
      14         & $g_{17} = n,g_{18} = 2^{\lg n}$               \\ \hline
      15         & $g_{19} = (\sqrt{2})^{\lg n}$                 \\ \hline
      16         & $g_{20} = 2^{\sqrt{2 \lg n}}$                 \\ \hline
      17         & $g_{21} = \lg^2n$                             \\ \hline
      18         & $g_{22} = \ln n$                              \\ \hline
      19         & $g_{23} = \sqrt{\lg n}$                       \\ \hline
      20         & $g_{24} = \ln \ln n$                          \\ \hline
      21         & $g_{25} = 2^{\lg^*n}$                         \\ \hline
      22         & $g_{26} = \lg^*(\lg n), g_{27} = \lg^*n$      \\ \hline
      23         & $g_{28} = \lg(\lg^*n)$                        \\ \hline
      24         & $g_{29} = n^{1/\lg n}$                        \\ \hline
    \end{tabular}
  \end{table}

\end{solution}
%%%%%%%%%%%%%%%

%%%%%%%%%%%%%%%%%%%%
\beginoptional

%%%%%%%%%%%%%%%
\begin{problem}[DH Problem 6.13]
\end{problem}

\begin{solution}
\end{solution}
%%%%%%%%%%%%%%%

%%%%%%%%%%%%%%%%%%%%
\beginot

%%%%%%%%%%%%%%%
本次 OT 介绍两种证明问题下界的常用技术。

\begin{ot}[Decision Trees]
  介绍 Decison Trees (决策树) 的概念以及在证明问题下界时的应用
  (包括但不限于本次选做题 DH 6.13)。

  参考资料:
  \begin{itemize}
    \item \href{https://en.wikipedia.org/wiki/Decision\_tree\_model}{Decision tree model @ wiki}
    \item \href{http://jeffe.cs.illinois.edu/teaching/algorithms/notes/12-lowerbounds.pdf}{lecture-note by jeffe}
  \end{itemize}
\end{ot}

% \begin{solution}
% \end{solution}
%%%%%%%%%%%%%%%
\vspace{0.50cm}
%%%%%%%%%%%%%%%
\begin{ot}[Adversary Argument]
  介绍 Adversary Argument (对手论证) 的概念
  以及在证明问题下界时的应用。

  \begin{itemize}
    \item \href{http://jeffe.cs.illinois.edu/teaching/algorithms/notes/13-adversary.pdf}{lecture-note by jeffe}
  \end{itemize}
\end{ot}

% \begin{solution}
% \end{solution}
%%%%%%%%%%%%%%%

%%%%%%%%%%%%%%%%%%%%
% 如果没有需要订正的题目,可以把这部分删掉

% \begincorrection
%%%%%%%%%%%%%%%%%%%%

%%%%%%%%%%%%%%%%%%%%
% 如果没有反馈,可以把这部分删掉
\beginfb

% 你可以写
% ~\footnote{优先推荐 \href{problemoverflow.top}{ProblemOverflow}}:
% \begin{itemize}
%   \item 对课程及教师的建议与意见
%   \item 教材中不理解的内容
%   \item 希望深入了解的内容
%   \item $\cdots$
% \end{itemize}
%%%%%%%%%%%%%%%%%%%%
\end{document}