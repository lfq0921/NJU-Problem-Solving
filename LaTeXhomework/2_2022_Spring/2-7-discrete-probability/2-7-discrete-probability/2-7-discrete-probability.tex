% 2-7-discrete-probability.tex

%%%%%%%%%%%%%%%%%%%%
\documentclass[a4paper, justified]{tufte-handout}

\input{hw-preamble} % feel free to modify this file
%%%%%%%%%%%%%%%%%%%%
\title{第7讲: 离散概率基础}
\me{林凡琪 }{ 211240042}{}{}
\date{\zhtoday} % or like 2019年9月13日
%%%%%%%%%%%%%%%%%%%%
\begin{document}
\maketitle
%%%%%%%%%%%%%%%%%%%%
\noplagiarism % always keep this line
%%%%%%%%%%%%%%%%%%%%
\begin{abstract}
  % \begin{center}{\fcolorbox{blue}{yellow!60}{\parbox{0.65\textwidth}{\large 
  %   \begin{itemize}
  %     \item 
  %   \end{itemize}}}}
  % \end{center}
\end{abstract}
%%%%%%%%%%%%%%%%%%%%
\beginrequired

%%%%%%%%%%%%%%%
\begin{problem}[CS 5.1-10]
How many five-card hands chosen from a standard deck of playing cards consist of five cards in a row (such as the nine of diamonds, ten of clubs, jack of clubs, queen of hearts, and king of spades)? Such a hand is called a straight. What is the probability that a five-card hand is a straight? Explore whether you get the same answer by using five-element sets as your model of hands or five-element permutations as your model of hands.
\end{problem}

\begin{solution}
  (a)组合:抽五张牌一共有C(5, 54)种,顺子一共有10种,抽出顺子的组合种类有$4^5 \times 10$.\\
  所以概率$P = \frac{4^5 \times 10}{C(5, 54)} = 0.00324$.\\
  (b)排列:$P= \frac{4^5 \times 10 \times 5!}{C(5, 54)\times 5!} = 0.00324$
\end{solution}
%%%%%%%%%%%%%%%

%%%%%%%%%%%%%%%
\begin{problem}[CS 5.1-12]
A die is made of a cube with a square painted on one side, a circle on two sides, and a triangle on three sides. If the die is rolled twice, what is the probability that the two shapes you see on top are the same?
\end{problem}

\begin{solution}
  两次都是正方形:$P_1 = \frac{1}{6} ^ 2$\\
  两次都是圆形:$P_2 = \frac{1}{3} ^2$\\
  两次都是三角形:$P_3 = \frac{1}{2} ^ 2$\\
  所以两次都形状相同,概率$P = P_1 + P_2 + P_3 = \frac{14}{36} = \frac{7}{18}$
\end{solution}
%%%%%%%%%%%%%%%

%%%%%%%%%%%%%%%
\begin{problem}[CS 5.2-4]
A bowl contains two red, two white, and two blue balls. If you remove two balls, what is the probability that at least one is red or white? Compute the probability that at least one is red.
\end{problem}

\begin{solution}
  (a)至少有一个白球或红球,即只要不是全蓝的就行。\\
  $P = 1 - \frac{C(2, 2)}{C(2, 6)} = \frac{14}{15}$\\
  (b)至少有一个是红球的对立事件是:一个红球都没有,即两个都是白球或蓝球。\\
  $P_0 = \frac{C(2,4)}{C(2, 6)} = 0.4$\\
  所以至少有一个红球的概率是$P_1=1-P_0 = 0.6$\\
\end{solution}
%%%%%%%%%%%%%%%

%%%%%%%%%%%%%%%
\begin{problem}[CS 5.2-10]
If you are hashing n keys into a hash table with k locations, what is the probability that every location gets at least one key?
\end{problem}

\begin{solution}
  1.如果n<k,则概率为0.\\
  2.如果n>=k, 根据TH5.4,可知%$|\bigcup^n_{i=1}E_i| = \sum ^n _{k = 1}(-1)^{k+1}\sum_{i_1,i_2,...,i_k:1\leq i_1 < i_2 < ... < i_k \leq n}|E_{i_1} \cap E_{i_2} \cap ... \cap E_{i_k}|$.%
  每个位置都有钥匙的可能有$\sum^k_{i = 0}(-1)^i\binom{k}{i} (k-i)^n$种,所以每个位置都有钥匙的概率为$\frac{\sum^k_{i = 0}(-1)^i\binom{k}{i} (k-i)^n}{k^n}$

\end{solution}
%%%%%%%%%%%%%%%

%%%%%%%%%%%%%%%
\begin{problem}[CS 5.3-2]
In three flips of a coin, is the event that two flips in a row are heads independent of the event that there is an even number of heads?

\end{problem}

\begin{solution}
  两次朝上(E)的概率为$P(E) = C(2,3) * \frac{1}{2}^3$,偶数次朝上(F)的概率为也为$P(F) = C(2,3) * \frac{1}{2}^3$\\
  而条件概率下,$P(E|F) = \frac{C(2,3) * \frac{1}{2}^3}{C(2,3) * \frac{1}{2}^3} = 1 \neq P(E)$所以两次正面和偶数次正面不是独立事件。
\end{solution}
%%%%%%%%%%%%%%%

%%%%%%%%%%%%%%%
\begin{problem}[CS 5.3-12]
In a family consisting of a mother, father, and two children of different ages, what is the probability that the family has two girls, given that one of the children is a girl? What is the probability that the children are both boys, given that the older child is a boy?
\end{problem}

\begin{solution}
  根据高中生物题,至少有一个女孩的概率$P_1 = 1 - \frac{1}{4} = \frac{3}{4}$\\
  有两个女孩的概率是$P_3 = \frac{1}{4}$\\
  所以在已经有一个女孩的条件下,有两个女孩的概率是$P_5 = \frac{1}{3}$\\
  更老的那个是男孩的概率为$P_0 = \frac{1}{2}$\\
  有两个男孩的概率是$P_2 = \frac{1}{4}$\\
  所以在更大的孩子是男孩的条件下有两个男孩的概率是$P_4 = \frac{1}{2}$
\end{solution}
%%%%%%%%%%%%%%%

%%%%%%%%%%%%%%%
\begin{problem}[CS 5.4-10]
What is the expected value of the constant random variable X that has X(s) = c for every member s of the sample space? (We frequently use c to stand for this random variable. Thus, this question is asking for E(c).)
\end{problem}

\begin{solution}
  $$
    \begin{aligned}
      E(c) & = \sum_{s:s\in S}X(s)P(s) \\
           & =\sum_{s:s\in S}cP(s)     \\
           & =c\sum_{s:s\in S}P(s)     \\
           & =c
    \end{aligned}
  $$
  其中X(s) = c
\end{solution}
%%%%%%%%%%%%%%%

%%%%%%%%%%%%%%%
\begin{problem}[CS 5.4-15]
prove $E(cX) = cE(X)$
\end{problem}

\begin{solution}
  使用数学归纳法。\\
  i)当c = 1时, $E(1\cdot X) = 1\cdot E(X)$.\\
  ii)假设当c = k - 1时,$E((k-1)X) = (k-1)E(X)$成立\\
  则可由定理5.10得出\\
  $$
    \begin{aligned}
      E(kX) & = E((k-1)X + X)    \\
            & = E((k-1)X) + E(X) \\
            & = (k-1)E(X) + E(X) \\
            & =kE(X)
    \end{aligned}
  $$
  所以当c为任意大于等于1的自然数时,该等式成立。\\
  证毕

\end{solution}
%%%%%%%%%%%%%%%

%%%%%%%%%%%%%%%%%%%%
\beginoptional

%%%%%%%%%%%%%%%
\begin{problem}[The Ballot Problem]
In an election, candidate $A$ receives $n$ votes,
and candidate $B$ receives $m$ votes where $n > m$.
Assuming that all orderings are equally likely,
what is the probability that $A$ is always ahead in the count of votes?
\end{problem}

\begin{solution}
\end{solution}
%%%%%%%%%%%%%%%

%%%%%%%%%%%%%%%%%%%%
\beginot

%%%%%%%%%%%%%%%
\begin{ot}[Monty Hall Problem]
  请介绍 Monty Hall Problem, 尽量讲清楚各种版本背后的概率解释。

  \noindent 参考资料:
  \begin{itemize}
    \item \href{https://en.wikipedia.org/wiki/Monty\_Hall\_problem}{Monty Hall problem @ wiki}
    \item \href{https://www.youtube.com/watch?v=cXqDIFUB7YU}{``21'' Movie @ Youtube}
  \end{itemize}
\end{ot}

% \begin{solution}
% \end{solution}
%%%%%%%%%%%%%%%
% \vspace{0.50cm}
%%%%%%%%%%%%%%%
\begin{ot}[Shuffling Cards]
  \noindent 请参考下列资料介绍``洗牌''中的数学。
  (不必追求严格推导, 主要介绍基本思想。)

  \begin{quote}
    ``How often does one have to shuffle a deck of cards until it is random?''
  \end{quote}

  \noindent 参考资料:
  \begin{itemize}
    \item Section ``Top-in-at-random shuffles'' of Chapter 30
          of Book: ``Proofs from THE BOOK'' (见课程网站)
  \end{itemize}
\end{ot}

% \begin{solution}
% \end{solution}
%%%%%%%%%%%%%%%

%%%%%%%%%%%%%%%%%%%%
% 如果没有需要订正的题目,可以把这部分删掉

% \begincorrection
%%%%%%%%%%%%%%%%%%%%

%%%%%%%%%%%%%%%%%%%%
% 如果没有反馈,可以把这部分删掉
\beginfb

% 你可以写
% ~\footnote{优先推荐 \href{problemoverflow.top}{ProblemOverflow}}:
% \begin{itemize}
%   \item 对课程及教师的建议与意见
%   \item 教材中不理解的内容
%   \item 希望深入了解的内容
%   \item $\cdots$
% \end{itemize}
%%%%%%%%%%%%%%%%%%%%
% \bibliography{2-5-solving-recurrence}
% \bibliographystyle{plainnat}
%%%%%%%%%%%%%%%%%%%%
\end{document}