% 2-14-b-tree.tex

%%%%%%%%%%%%%%%%%%%%
\documentclass[a4paper, justified]{tufte-handout}

\input{hw-preamble} % feel free to modify this file
%%%%%%%%%%%%%%%%%%%%
\title{第14讲: B 树}
\me{林凡琪 }{211240042}{}{}
\date{\zhtoday} % or like 2019年9月13日
%%%%%%%%%%%%%%%%%%%%
\begin{document}
\maketitle
%%%%%%%%%%%%%%%%%%%%
\noplagiarism % always keep this line
%%%%%%%%%%%%%%%%%%%%
\begin{abstract}
  % \begin{center}{\fcolorbox{blue}{yellow!60}{\parbox{0.65\textwidth}{\large 
  %   \begin{itemize}
  %     \item 
  %   \end{itemize}}}}
  % \end{center}
\end{abstract}
%%%%%%%%%%%%%%%%%%%%
\beginrequired

%%%%%%%%%%%%%%%
\begin{problem}[TC 18.1-1]
\end{problem}

\begin{solution}
  根据定义,最小度 tt 意味着除根以外的每个节点都必须至少有 t - 1 个key,因此除根以外的每个内部节点都至少有 tt 个子节点。 因此,当 t = 1 时,意味着除根之外的每个节点都必须至少有 t - 1 = 0 个key,因此除根之外的每个内部节点都至少有 t = 1 个子节点。\\
  因此,我们可以看到最小情况不存在,因为在 B 树中不存在具有 0 个key的节点,并且不存在只有 1 个子节点的节点.
\end{solution}
%%%%%%%%%%%%%%%

%%%%%%%%%%%%%%%
\begin{problem}[TC 18.1-4]
\end{problem}

\begin{solution}
  $$
    \begin{aligned}
      n & = (1 + 2t + (2t)^2 + … + (2t)^h) * (2t - 1) \\
        & = (2t)^{h + 1} - 1
    \end{aligned}
  $$
\end{solution}
%%%%%%%%%%%%%%%

%%%%%%%%%%%%%%%
\begin{problem}[TC 18.2-3]
\end{problem}

\begin{solution}
  在 B-tree 中找到最小值与在二叉搜索树中找到最小值非常相似。 我们需要为给定的根找到最左边的叶子,并返回第一个键\\
  x 是 B-tree T 上的一个节点。顶层调用是 B-TREE-FIND-MIN(T.root)\\
  FCTVAL 是存储在以 x 为根的子树中的最小key
  \begin{algorithm}
    \begin{algorithmic}[1]
      \Procedure{b-tree-find-min}{x}
      \If{x = NIL}
      \Return NIL
      \ElsIf{x.leaf}
      \Return x.key[1]
      \Else
      \State DISK-READ(x.c[1])
      \Return B-TREE-FIND-MIN(x.c[1])
      \EndIf
      \EndProcedure
    \end{algorithmic}
  \end{algorithm}
  根据以下规则查找给定密钥 $x.key_i$ 的前继:

  如果 x 不是叶节点,则返回 x 的第 i 个子节点中的最大键,这也是以 $x.c_i$ 为根的子树的最大key

  如果 x 是叶节点并且 i > 1,则返回 x 的第 (i - 1) 个key,即 $x.key_{i - 1}$

  否则,查找最后一个节点 y(从下向上)且 j > 0,使得 $x.key_i$ 是 $y.c_j$ 中最左边的key; 如果 j = 1,则返回 $\text{NIL}$,因为 $x.key_i$x.key 是树中的最小key; 否则我们返回 $y.key_{j - 1}$

  x 是 B 树 T 上的一个节点。 i 是键的索引。

  FCTVAL 是 $x.key_i$ 的前身

  \begin{algorithm}
    \begin{algorithmic}[2]
      \Procedure{b-tree-predecessor}{x, i}
      \If{!x.leaf}
      \State DISK-READ(x.c[i])
      \Return B-TREE-FIND-MAX(x.c[i])
      \ElsIf{i > 1}
      \Return x.key[i - 1]
      \Else
      \State z$\gets$x
      \While{true}
      \If{z.p = NIL}
      \Return NIL
      \EndIf
      \State y$\gets$z.p
      \State j$\gets$1
      \State DISK-READ(y.c[1])
      \While{y.c[j] != x}
      \State j$\gets$j + 1
      \State DISK-READ(y.c[j])
      \EndWhile
      \If{j = 1}
      \State z$\gets$y
      \Else
      \Return y.key[j - 1]
      \EndIf
      \EndWhile
      \EndIf
      \EndProcedure
    \end{algorithmic}
  \end{algorithm}

  x 是 B-tree TT 上的一个节点。 顶层调用是 $\text{B-TREE-FIND-MAX}(T.root)$

  $\text{FCTVAL}$ 是存储在以 x 为根的子树中的最大key。\\
  \begin{algorithm}
    \begin{algorithmic}[3]
      \Procedure{b-tree-find-max}{x}
      \If{x = NIL}
      \Return NIL
      \ElsIf{x.leaf}
      \Return x.[x.n]
      \Else
      \State DISK-READ(x.c[x.n + 1])
      \Return B-TREE-FIND-MAX(x.c[x.n + 1])
      \EndIf
      \EndProcedure
    \end{algorithmic}
  \end{algorithm}
\end{solution}
%%%%%%%%%%%%%%%

%%%%%%%%%%%%%%%
\begin{problem}[TC 18.3-1]
\end{problem}

\begin{solution}
  \begin{figure}
    \centering
    \includegraphics[width=4cm,height=5cm]{pic.jpg}
  \end{figure}
\end{solution}
%%%%%%%%%%%%%%%

%%%%%%%%%%%%%%%%%%%%
\beginoptional

%%%%%%%%%%%%%%%
\begin{problem}[TC 18.2-4\red{$^{\star}$}]
\end{problem}

\begin{solution}
\end{solution}
%%%%%%%%%%%%%%%

%%%%%%%%%%%%%%%%%%%%
\beginot

%%%%%%%%%%%%%%%
\begin{ot}[T-tree]
  介绍 T-tree (具体的rotation过程可以不用详细介绍)。

  \noindent 参考资料:
  \begin{itemize}
    \item \href{https://en.wikipedia.org/wiki/T-tree}{https://en.wikipedia.org/wiki/T-tree}
  \end{itemize}
\end{ot}
%%%%%%%%%%%%%%


\begin{ot}[R-tree]
  介绍 R-tree。

  \noindent 参考资料:
  \begin{itemize}
    \item \href{https://en.wikipedia.org/wiki/R-tree}{https://en.wikipedia.org/wiki/R-tree}
  \end{itemize}
\end{ot}
%%%%%%%%%%%%%%
% \begin{solution}
% \end{solution}
%%%%%%%%%%%%%%%
% \vspace{0.50cm}
%%%%%%%%%%%%%%%
% \begin{ot}[]
% 
%   \noindent 参考资料:
%   \begin{itemize}
%     \item 
%   \end{itemize}
% \end{ot}

% \begin{solution}
% \end{solution}
%%%%%%%%%%%%%%%

%%%%%%%%%%%%%%%%%%%%
% 如果没有需要订正的题目,可以把这部分删掉

% \begincorrection
%%%%%%%%%%%%%%%%%%%%

%%%%%%%%%%%%%%%%%%%%
% 如果没有反馈,可以把这部分删掉
\beginfb

% 你可以写
% ~\footnote{优先推荐 \href{problemoverflow.top}{ProblemOverflow}}:
% \begin{itemize}
%   \item 对课程及教师的建议与意见
%   \item 教材中不理解的内容
%   \item 希望深入了解的内容
%   \item $\cdots$
% \end{itemize}
%%%%%%%%%%%%%%%%%%%%
% \bibliography{2-5-solving-recurrence}
% \bibliographystyle{plainnat}
%%%%%%%%%%%%%%%%%%%%
\end{document}