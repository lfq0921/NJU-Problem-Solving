% 2-11-heapsort.tex

%%%%%%%%%%%%%%%%%%%%
\documentclass[a4paper, justified]{tufte-handout}

\input{hw-preamble} % feel free to modify this file
%%%%%%%%%%%%%%%%%%%%
\title{第11讲: 堆与堆排序}
\me{林凡琪 }{211240042 }{}{}
\date{\zhtoday} % or like 2019年9月13日
%%%%%%%%%%%%%%%%%%%%
\begin{document}
\maketitle
%%%%%%%%%%%%%%%%%%%%
\noplagiarism % always keep this line
%%%%%%%%%%%%%%%%%%%%
\begin{abstract}
  % \begin{center}{\fcolorbox{blue}{yellow!60}{\parbox{0.65\textwidth}{\large 
  %   \begin{itemize}
  %     \item 
  %   \end{itemize}}}}
  % \end{center}
\end{abstract}
%%%%%%%%%%%%%%%%%%%%
\beginrequired

%%%%%%%%%%%%%%%
\begin{problem}[TC 6.1-2]
证明有n个元素的堆高度为$\lfloor \lg n \rfloor$
\end{problem}

\begin{solution}
  令$n = 2^m − 1 + k$,其中$m$很大.\\
  堆由一个高度为m-1的完全二叉树组成,底部有k个附加的叶节点.\\
  根的高度是通往这k个叶节点之一的最长的简单路径的长度m.\\
  从定义可得, $m = \lfloor \lg n \rfloor$

\end{solution}
%%%%%%%%%%%%%%%


%%%%%%%%%%%%%%%
\begin{problem}[TC 6.1-7]
\end{problem}

\begin{solution}
  取序号为$\lfloor n/2 \rfloor + 1$的子节点.\\
  $$
    \begin{aligned}
      \text{LEFT}(\lfloor n / 2 \rfloor + 1) & = 2(\lfloor n / 2 \rfloor + 1) \\
                                             & > 2(n / 2 - 1) + 2             \\
                                             & = n - 2 + 2                    \\
                                             & = n.
    \end{aligned}
  $$
  由于左侧子节点的序号大于堆中元素的个数,所以该节点没有子节点,所以是叶节点.其他有更大区号的节点同理也是叶节点.
  \\
  但是对于序号为$\lfloor n/2 \rfloor$的节点,它不是叶节点.在节点数为偶数的情况下它将有一个序号为n的左节点,在节点数为奇数的情况下,它将有一个索引为n-1的左节点和一个索引为n的右节点.\\
  这使得n大小的堆中的叶子数量为$\lceil n / 2\rceil$
\end{solution}
%%%%%%%%%%%%%%%

%%%%%%%%%%%%%%%
\begin{problem}[TC 6.2-5]
\end{problem}

\begin{solution}
  \begin{algorithm}[H]
    \caption{MAX-HEAPIFY}
    \begin{algorithmic}[1]
      \Procedure{MAX}{$A,i$}
      \While {1}
      \State $l = LEET(i)$
      \State $r = RIGHT(i)$
      \If {$l \leq A.heap-size and A[l] > A[i]$}
      \State $largest = l$
      \Else $largest = i$
      \EndIf
      \If {$r \leq  A.heap-size and A[r] > A[largest]$}
      \State $largest = r$
      \EndIf
      \If {$largest == i$}
      \State \Return
      \EndIf
      \State $swap(A[i], A[largest]$)
      \State $i = largest$
      \EndWhile
      \EndProcedure
    \end{algorithmic}
  \end{algorithm}
\end{solution}
%%%%%%%%%%%%%%%

%%%%%%%%%%%%%%%
\begin{problem}[TC 6.2-6]
\end{problem}

\begin{solution}
  考虑由A产生的堆,其中$A[1]=1,A[i]=2(2 \leq i \leq n)$\\
  由于1是堆中最小的元素,它必须通过堆的每一层进行交换,直到它成为一个叶子节点。\\
  由于堆的高度是$lfloor \lg n\rfloor$因此\text{MAX-HEAPIFY}的最坏情况时间是$\Omega(\lg n)Ω(lgn)$。
\end{solution}
%%%%%%%%%%%%%%%

%%%%%%%%%%%%%%%
\begin{problem}[TC 6.3-3]
\end{problem}

\begin{solution}
  从6.1-7中,我们知道,堆的叶子节点序号如下

  $\left\lfloor n / 2 \right\rfloor + 1, \left\lfloor n / 2 \right\rfloor + 2, \dots, n.$

  注意,这些元素对应于堆数组的后半部分(如果nn是奇数,则加上中间的元素)。因此,任何大小为nn的堆中的叶子数量是$\left\lceil n / 2\right\rceil⌈n/2⌉$。\\
  让我们通过归纳法来证明。\\
  让$n_h$表示高度为h的节点数.由于$n_0 = \left\lceil n / 2 \right\rceil$所以上界对base成立.\\
  现在假设它对$h-1$成立.我们已经证明它对h成立.\\
  如果$n_{h-1}$是偶数,则高度为h的每个节点正好有两个子节点,这意味着$n_h = n_{h - 1} / 2 = \left\lfloor n_{h - 1} / 2 \right\rfloor$\\
  如果$n_{h-1}$是奇数,高度为h的一个节点有一个孩子,其余的有两个孩子,这也意味着$n_h = \left\lfloor n_{h - 1} / 2 \right\rfloor + 1 = \left\lceil n_{h - 1} / 2 \right\rceil$\\
  因此,我们有
  $$
    \begin{aligned}
      n_h & = \left\lceil \frac{n_{h - 1}}{2} \right\rceil                                                    \\
          & \le \left\lceil \frac{1}{2} \cdot \left\lceil \frac{n}{2^{(h - 1) + 1}} \right\rceil \right\rceil \\
          & = \left\lceil \frac{1}{2} \cdot \left\lceil \frac{n}{2^h} \right\rceil \right\rceil               \\
          & = \left\lceil \frac{n}{2^{h + 1}} \right\rceil
    \end{aligned}
  $$

\end{solution}
%%%%%%%%%%%%%%%

%%%%%%%%%%%%%%%
\begin{problem}[TC 6.4-2]
\end{problem}

\begin{solution}
  Initialization:数组A[i+1...n]是空的,所以不变式成立\\
  Maintenance:\\
  $A[1]$是A[1...i]中最大的元素,它比A[i + 1...n]A[i+1...n]中的元素小。当我们把它放在第i个位置的时候,那么$A[i..n]$含有已被排序的最大的元素.\\
  减少堆的大小并调用 text{MAX-HEAPIFY}将A[1...i-1]变成一个最大堆。\\
  递减i为下一次迭代奠定了不变性。\\
  Termination:\\
  循环后i=1。这意味着A[2...n]被排序,A[1]是数组中最小的元素\\
\end{solution}
%%%%%%%%%%%%%%%

%%%%%%%%%%%%%%%
\begin{problem}[TC 6.4-4]
\end{problem}

\begin{solution}
  花费线性的时间把它转换为max-heap,然后用$n \lg n$的时间排序
\end{solution}
%%%%%%%%%%%%%%%

%%%%%%%%%%%%%%%
\begin{problem}[TC 6.4-5 (\red{$^{\ast}$})]
\end{problem}

\begin{solution}
  假设这个堆是个满二叉树,$n=2^k - 1$,有$2^{k-1}$个叶子节点和$2^{k-1}-1$个内部节点.\\
  考虑把堆前面的$2^{k-1}$个元素排序.我们考虑它们在堆中的排列,把叶子节点涂成红色,内部节点涂成蓝色,有色节点是堆的一个子树.因为有$2^{k-1}$个有色节点,所以最多只有$2^{k - 2}$个是红色的,这意味着最少有$2^{k - 2} - 1$个节点是蓝色的.\\
  虽然红色节点可以直接跳到根部,但蓝色节点在被移除之前需要向上移动。让我们计算一下将蓝色节点移到根部的最少交换次数。\\
  此时有$2^{k - 2} - 1$个节点是蓝色的,并且它们被排列在二叉树中.\\
  如果有d个这样的蓝色节点,那么就会有$i=\lg d$层,每个层包含$2^i$个长度为i的节点。因此,互换的次数是
  $$
    \sum_{i = 0}^{\lg d}i2^i = 2 + (\lg d - 2)2^{\lg d} = \Omega(d\lg d).$$
  有这样的递推关系
  $$T(n) = T(n / 2) + \Omega(n\lg n).$$
  利用master method 我们得到$T(n) = \Omega(n\lg n)$
\end{solution}
%%%%%%%%%%%%%%%

%%%%%%%%%%%%%%%
\begin{problem}[TC 6.5-5]
\end{problem}

\begin{solution}
  Initialiazation:\\
  A是一个堆,只是A[i]可能比它的父代大,因为它已经被修改了。A[i]比它的子代大,否则guard clause会失败,不会进入循环(新值比旧值大,旧值比子代大)。\\
  Maintenance:\\
  当我们将A[i]与其父本交换时,max-heap属性得到了满足,只是现在A[text{PARENT}(i)]可能比其父本大。把i改成它的父本,就能保持不变性。\\
  Termination:\\
  只要堆被耗尽或者A[i]及其父本的max-heap属性被保留,循环就会终止。在循环终止时,A是一个max-heap。
\end{solution}
%%%%%%%%%%%%%%%

%%%%%%%%%%%%%%%
\begin{problem}[TC 6.5-9]
\end{problem}

\begin{solution}
  \begin{algorithm}
    \begin{algorithmic}
      \Procedure{merge-sorted-lists}{lists[][]}
      \State n$\gets$lists.length
      \State let lowest-from-each be an empty array
      \For{$i \gets 1 to n$}
      \State add (lists[i][0], i) to lowest-from-each
      \State delete lists[i][0]
      \EndFor
      \State A$\gets$MIN-HEAP(lowest-from-each)
      \State let merged-lists be an empty array
      \While{$!min-heap.EMPTY()$}
      \State element-value, index-of-list $\gets$ HEAP-EXTRACT-MIN(A)
      \State add element-value to merged-lists
      \If{$lists[index-of-list].length > 0$}
      \State MIN-HEAP-INSERT(A, lists[index-of-list][0], inde-of-list))
      \State delete lists[index-of-list][0]
      \EndIf
      \EndWhile
      \Return merged-lists
      \EndProcedure
    \end{algorithmic}
  \end{algorithm}
\end{solution}
%%%%%%%%%%%%%%%

%%%%%%%%%%%%%%%%%%%%
\beginoptional

%%%%%%%%%%%%%%%
\begin{problem}[Heap Equality]
Prove that
\[
  \forall h \in \mathbb{Z}^{+}: \lceil \log(\lfloor \frac{1}{2}h \rfloor + 1) \rceil + 1 = \lceil \log (h + 1) \rceil.
\]
\end{problem}

\begin{solution}
\end{solution}
%%%%%%%%%%%%%%%

%%%%%%%%%%%%%%%%%%%%
\beginot

%%%%%%%%%%%%%%%
\begin{ot}[Binomial Heaps]
  介绍 Binomial Heap 数据结构。

  \noindent 参考资料:
  \begin{itemize}
    \item Problem 19-2 of《算法导论》
    \item \href{https://en.wikipedia.org/wiki/Binomial\_heap}{Binomial heap @ wiki}
  \end{itemize}
\end{ot}

% \begin{solution}
% \end{solution}
%%%%%%%%%%%%%%%
% \vspace{0.50cm}
%%%%%%%%%%%%%%%
\begin{ot}[Fibonacci Heaps]
  介绍 Fibonacci Heap 数据结构。可不介绍其中的平摊分析内容。

  \noindent 参考资料:
  \begin{itemize}
    \item Chapter 19 of《算法导论》
    \item \href{https://en.wikipedia.org/wiki/Fibonacci\_heap}{Fibonacci heap @ wiki}
  \end{itemize}
\end{ot}

% \begin{solution}
% \end{solution}
%%%%%%%%%%%%%%%

%%%%%%%%%%%%%%%%%%%%
% 如果没有需要订正的题目,可以把这部分删掉

% \begincorrection
%%%%%%%%%%%%%%%%%%%%

%%%%%%%%%%%%%%%%%%%%
% 如果没有反馈,可以把这部分删掉
\beginfb

% 你可以写
% ~\footnote{优先推荐 \href{problemoverflow.top}{ProblemOverflow}}:
% \begin{itemize}
%   \item 对课程及教师的建议与意见
%   \item 教材中不理解的内容
%   \item 希望深入了解的内容
%   \item $\cdots$
% \end{itemize}
%%%%%%%%%%%%%%%%%%%%
% \bibliography{2-5-solving-recurrence}
% \bibliographystyle{plainnat}
%%%%%%%%%%%%%%%%%%%%
\end{document}